\section{Classification for the Larson--Wogen $M$-basis}
  \label{section:lw-sys}
  In this section we study Example~\ref{lw-sys} of Larson--Wogen vector system $\fsys_{LW}$.
  Namely, we prove a theorem similar to Theorem 2.2 of~\cite{katavolos}.
  Up until now there existed two different techniques in studying $k$ point density for $k=1$ (strong $M$-bases) and for $k\geq2$.
  Here we demonstrate a universal method for the analysis of the $k$ point density property.
  \begin{theorem}[\cite{katavolos}, Theorem 2.2]
    \label{thm:katavolos}
    The sequences $\fsys_{LW}$ and $\fsys_{LW}^*$ are biorthogonal and both are complete in $\cal{H}$.
    Moreover, the following is true.
    \begin{itemize}
      \item  The system $\fsys_{LW}$ is one point dense (a strong $M$-basis) if and only if the sequence
        \begin{equation}
          \mu_n = \frac{a_{n-1} a_{n-3} \dots}{a_{n} a_{n-2} \dots }
        \end{equation}
        does not belong to $\ell^2$.
      \item The system $\fsys_{LW}$ is $k$ point dense ($k > 1$) if and only if the sequence $\{1/a_n\}_{n=1}^\infty$ does not belong to $\ell^1$.
    \end{itemize}
  \end{theorem}
  \begin{proof}
    Consider a $k$-dimensional operator $T = \sum_{s=1}^k y^s \otimes x^s$ which annihilates $R_1(\cal{A})$.
    We reproduce the logic from the previous section and define $\Xi_n$, $u_n$, $v_n$ exactly in the way it is described there.
    For the given $M$-basis $\fsys = \fsys_{LW}$ we calculate $\Xi_n$ precisely:
    \begin{align*}
      \Xi_{2n-1} &= a_{2n}T_{2n - 1, 2n},\\
      \Xi_{2n} &= a_{2n + 1}T_{2n + 1, 2n},
    \end{align*}
      where $T_{ij} = \inner{Te_j}{e_i}$.

    Since $T_{ij} = \inner{u_j}{v_i}$, we can deduce the expressions for $\Xi_n$ via the vectors $u_n$ and $v_n$:
    \begin{align*}
      \Xi_{2n-1} &= a_{2n} \sum_{s=1}^k y^s_{2n} x^s_{2n-1} = a_{2n} \inner{u_{2n}}{v_{2n-1}},\\
      \Xi_{2n} &= a_{2n+1} \sum_{s=1}^k y^s_{2n} x^s_{2n+1} = a_{2n+1} \inner{u_{2n}}{v_{2n+1}},
    \end{align*}
      where $\langle\cdot, \cdot\rangle$ denotes the scalar product in $\mathbb{R}^k$.

    For the convenience of the reader we introduce the sequences of vectors $w_n$ and $w^*_n$.
    \begin{align*}
      w_{2n} &= u_{2n} \quad w^*_{2n} = v_{2n},\\
      w_{2n+1} &= v_{2n+1} \quad w^*_{2n+1} = u_{2n+1}.
    \end{align*}
    In view of this notation $\Xi_n = a_{n+1} \inner{w_n}{w_{n+1}}$.

    It follows from Proposition~\ref{prop:kreformulation} that $k$ point density of $\fsys$ is equivalent to the existence of
      such $k$-dimensional vectors $w_m$, $w^*_m$ lying in $\ell^2(\mathbb{R}^k)$ that
    \begin{equation}
      \label{eq:vector2}
      a_{n+1} \inner{w_n}{w_{n+1}} = -\sum_{m=0}^n \inner{w_m}{w^*_m},
    \end{equation}
      for any $m \geq 0$, such that $\sum_{m=0}^\infty \inner{w_m}{w^*_m} \neq 0$.

    In what follows we show that the conditions in the previous proposition can be simplified.
    \begin{prop}
      \label{prop:reformulation-lw}
      Such $k$-dimensional vectors $w_s$, $w^*_s$ do exist if and only if there exists such vectors $r_s$ in $\ell^2(\mathbb{R}^k)$
        satisfying the following.
      \begin{equation}
        \label{eq:vector3}
        a_{n+1} \inner{r_n}{r_{n+1}} = 1,
      \end{equation}
      for any $n \geq 0$.
    \end{prop}
    \begin{proof}
      Suppose we found such $r_n$.
      Then we solve~\eqref{eq:vector2} by putting $w^*_s$ to zero for any $s > 0$, $w_s$ to $r_s$ and
        choose the vector $w^*_0$ so that $\inner{w_0}{w^*_0} = -1$.

      Now we prove the converse.
      Suppose we found such $w_n$ that~\eqref{eq:vector2} holds.
      Given that the vectors $w_n$ lie in the $\mathbb{R}^k$, we rewrite the scalar product as
        the product of the vector lengths and the cosine of the angle between the vectors.
      Namely, we define $W_n = \lvert w_n\rvert$ and real $\theta_n$ that
        $\inner{w_{n}}{w_{n+1}} = W_n W_{n+1} \cos{\theta_n}.$

      The sequence $\Xi_n = -\sum_0^n \inner{w_m}{w^*_m}$ has a non-zero limit, so let us
        find the largest $N > 0$ such that $\Xi_N = 0$.
      Then we can modify the original sequence by setting $w_n$, $w^*_n$ to zero for any $0 \leq n \leq N$ so that~\eqref{eq:vector2}
        still holds.
      Therefore, without loss of generality we can assume that $\Xi_n \neq 0$ for any $n \geq 0$.
      %\begin{prop}
        %For a given system $k$ point density is equivalent to the existence such $0 < a'_n \leq a_n$ that
        %\[
          %\nu_n = \frac{a'_{n-1} a'_{n-3} \dots}{a'_{n} a'_{n-2} \dots }
        %\]
        %belongs to $\ell^2$.
      %\end{prop}
      Setting $a'_n = a_n \cos{\theta_n}$ we see that the sequence
      \[
        W_n = \frac{\Xi_{n-1}/a'_n}{\Xi_{n-2}/a'_{n-1}} \cdot \frac{\Xi_{n-3}/a'_{n-2}}{\Xi_{n-4}/a'_{n-3}} \cdots
      \]
        belongs to $\ell^2$.
      Now since $\Xi_n = -\sum_0^n \inner{w_s}{w^*_s}$, we discover that
      \[
        \frac{\Xi_n}{\Xi_{n-1}} = 1 + \eta_n,
      \]
        where $\{\eta_n\}_{n=1}^\infty \in \ell^1$.
      Thus the product of such $(1 + \eta_s)$ fractions is bounded by some constant above.
      It follows that
      \[
        W^\#_n = \frac{1/a'_n}{1/a'_{n-1}} \cdot \frac{1/a'_{n-2}}{1/a'_{n-3}} \cdots
      \]
        belongs to $\ell^2$.
      Now we set $r_n$ to $\frac{W^\#_n}{W_n}w_n$, and then~\eqref{eq:vector2} holds since
      \[
        a_{n+1} \inner{r_n}{r_{n+1}} = a_{n+1} \frac{1/a'_{n+1}}{\Xi_n/a'_{n+1}} \inner{w_n}{w_{n+1}} = 1.
      \]
      Since $\lvert r_n \rvert = \lvert W^\#_n \rvert$ and $\{\lvert W^\#_n \rvert\}_{n=1}^\infty$ belongs to $\ell^2$,
        then $r_n \in \ell^2(\mathbb{R}^k)$ as well.
    \end{proof}

    Now we are ready to finish the proof.
    First we prove the theorem for the case $k=1$.
    \begin{prop}
      The system $\fsys_{LW}$ is one point dense if and only if $\{\mu_n\}_{n=1}^\infty$ does not belong to $\ell^2$.
    \end{prop}
    \begin{proof}
      We apply Propositions~\ref{prop:kreformulation} and~\ref{prop:reformulation-lw}.
      The case $k=1$ has all the vectors $r_k$, $r^*_k$ lying on the same line.
      The formula for the multidimensional case reduces to the expression of $\mu_n$ whenever we set all the angles to zero $\theta_n = 0$,
        namely the lengths of the vectors $r_k$ are precisely $\mu_n$.
    \end{proof}

    Now we consider the case $k > 1$.
    \begin{prop}
      The system $\fsys_{LW}$ is $k$ point dense \textup($k > 1$\textup) if and only if the sequence $\{1/a_n\}_{n=1}^\infty$
        does not belong to $\ell^1$.
    \end{prop}
    \begin{proof}
      According to Propositions~\ref{prop:kreformulation} and~\ref{prop:reformulation-lw}, the system $\fsys_{LW}$ is $k$ point dense
        if and only if there is no such sequence $\seq{r}$ in $\ell^2(\mathbb{R}^k)$ which satisfy $a_n \inner{r_n}{r_{n-1}} = 1$.

      Obviously, if there are such vectors $r_n$, then $\{1/a_n\}_{n=1}^\infty$ belongs to $\ell^1$.

      Conversely, suppose $\{1/a_n\}_{n=1}^\infty$ belongs to $\ell^1$.
      Then consider $R_n$ defined as follows:
      \[
        R_n = \max(\lvert a_n \rvert^{-\frac{1}{2}}, \lvert a_{n+1} \rvert^{-\frac{1}{2}}).
      \]
      It is easy to see that $R_n$ belongs to $\ell^2$.

      Notice that the product $R_nR_{n-1} \geq 1/\lvert a_n\rvert$, and so it is always possible to choose the angle $\theta_n$ so that
      \[
        a_n \langle r_n, r_{n-1} \rangle = a_n R_n R_{n-1}\cos{\theta_n} = 1.
      \]
      Now we have defined the lengths for $r_n$ and the angles between two consecutive vectors $r_{n-1}$, $r_n$.
      Obviously, for any $k \geq 2$ we are able to choose the corresponding vectors $r_n$.
    \end{proof}
    Theorem~\ref{thm:katavolos} is now proved with the last proposition.
  \end{proof}
  %\begin{remark}
    %The difference between two cases $k=1$ and $k\geq 2$ is in $\mathbb{R}^1$ any two vectors
      %have $0$ or $\pi$ angle between them.
  %\end{remark}
  % \begin{remark}
  %     Right here we can deduce the necessary and sufficient condition for the $k$-completeness property (for $k>1$) having the condition
  %     for the strong approximation property discovered earlier.
  %     As we remember $f_n$ approximates strongly iff the $a_n^{-1}$ is summable. Also we know as a fact that
  %     a system approximates strongly iff for any $k$ it is $k$-complete. Taking the last proposition into account we
  %     understand that the $k$-completeness condition is the same for all $k > 1$ and is identical to the strong approximation condition: $a_n^{-1}$ must be summable.
  %o  \end{remark}
