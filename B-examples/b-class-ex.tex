\documentclass[12pt]{amsart}
\usepackage{cmap}
%\usepackage[cp866]{inputenc}
\usepackage[T2A]{fontenc}
\usepackage[utf8]{inputenc}
\usepackage[ngerman, english]{babel}
\usepackage[pdftex,unicode]{hyperref}
\usepackage{amsmath}
\usepackage{amssymb}
\usepackage{amsthm}
\usepackage{verbatim}
\usepackage{amsfonts}
\usepackage{graphicx}
\usepackage[normalem]{ulem}
\usepackage{extsizes}
\usepackage{float}
\usepackage{bbold}
\usepackage{dsfont}
\usepackage{calc}
\usepackage{bm}
\newlength{\widecommentlength}
\setlength{\widecommentlength}{5in}
% \newcommand{\widecommentbox}[2]{\def#1##1{\strut\newline\noindent\colorbox{#2}{\linespread{1}\parbox{.95\textwidth}{\small ##1}}\newline}}
% \usepackage{pgfplots}
\newcommand{\widecommentbox}[3]{\def#1##1{\strut\newline\noindent\colorbox{#3}{\linespread{1}\parbox{.95\textwidth}{\small {\bf [#2]} ##1}}\newline}}
\def\commentsep{\noindent\dotfill}

% To temporarily omit all comments, enable these two lines:
% \renewcommand{\widecommentbox}[3]{\def#1##1{}}
% \let\commentsep\relax

\widecommentbox{\alex}{Alex}{green!20!white}
\widecommentbox{\ad}{AD}{red!20!white}

\usepackage{enumitem}

\makeatletter
%\@addtoreset{theorem}{section}
%\@addtoreset{lemma}{section}
%\@addtoreset{prop}{section}
\makeatother

%\newcommand{\sectionbreak}{\clearpage}
\usepackage[square,numbers,sort&compress]{natbib}
\usepackage{mathtools}
\renewcommand{\bibnumfmt}[1]{#1.\hfill} % нумерация источников в самом списке - через точку
% \renewcommand{\bibsection}{\section*{Список литературы}} % заголовок специального раздела
\setlength{\bibsep}{0pt}
\newcommand*{\Scale}[2][4]{\scalebox{#1}{\ensuremath{#2}}}%

%\titleformat{\section}[block]{\Large\bfseries\centering}{}{1em}{}
%\titleformat{\subsection}[block]{\large\bfseries\centering}{}{1em}{}
\newcommand{\cal}[1]{\mathcal{#1}}
\renewcommand{\leq}{\leqslant}
\renewcommand{\geq}{\geqslant}
\renewcommand{\phi}{\varphi}
\newtheorem{theorem}{Theorem}
\newtheorem{prop}{Proposition}
\newtheorem{lemma}{Lemma}
\theoremstyle{definition}
\newtheorem{corol}{Corollary}
\newtheorem*{corol*}{Corollary}
\newtheorem{remark}{Remark}
\newtheorem*{remark*}{Remark}
\newtheorem*{note}{Note}
\newtheorem{definition}{Definition}
\newcommand\bigmatrixzero{\raisebox{-0.25\height}{\textnormal{\Huge 0}}}
\newcommand\bigzero{\makebox(10, 10){\text{\Huge 0}}}
\newcommand{\system}[1]{\{{#1}_k\}_{k=1}^\infty}
\newcommand\inner[2]{\langle #1, #2 \rangle}
\newcommand\bigmatrixzero{\raisebox{-0.25\height}{\textnormal{\Huge 0}}}
\newcommand\bigzero{\makebox(10, 10){\text{\Huge 0}}}
\newcommand{\seq}[1]{\{{#1}_n\}_{n=1}^\infty}
\newcommand{\fsys}{\mathfrak{F}}
\newcommand{\fstarsys}{\mathfrak{F^{*}}}
\newcommand{\lattice}{\operatorname{Lat}\cal{A}}
\newcommand{\len}{\cal{L}}
\newcommand{\scal}[2]{\langle {#1}, {#2} \rangle}
\newcommand{\vspan}[1]{span\left(#1\right)}
\newcommand{\cspan}[1]{\overline{span}\left(#1\right)}

\numberwithin{remark}{section}
\numberwithin{theorem}{section}
\numberwithin{prop}{section}
\numberwithin{equation}{section}
\numberwithin{lemma}{section}

\setlength{\emergencystretch}{2pt}
\newtheoremstyle{case}{}{}{}{}{}{:}{ }{}
\theoremstyle{case}
\newtheorem{case}{Case}

\begin{document}
\title{Band-diagonal $M$-bases and the $k$ point density}
\author{Alexey Pyshkin}
\begin{abstract}
  %In the early 1990s the works of Larson, Wogen and Argyros, Lambrou, Longstaff
    %disclosed an example of a strong tridiagonal $M$-basis that was not rank one dense.
  %Later Katavolos, Lambrou and Papadakis studied $k$ point density property of this example.
  %In this paper we present new methods for the analysis of $k$ point density
    %and rank one density properties for band-diagonal $M$-bases.
\end{abstract}
\keywords{biorthogonal system, $M$-basis, rank one density, two point density}
%\thanks{The author was supported by RFBR (the project~16-01-00674) and by «Native towns», a social investment program of PJSC «Gazprom Neft».}
\maketitle

\section{Introduction}
  \subsection{Density properties}
    Consider an infinite-dimensional real Hilbert space $\cal{H}$.
    Suppose that $\cal{H}$ has an orthonormal basis $\{e_j\}_{j=0}^\infty$.
    A sequence $\fsys=\seq{f}$ of vectors in $\cal{H}$ is called \emph{minimal} if none of its elements can be approximated by the linear combinations of the others: $f_n \notin
      \overline{span}\big(\{f_k\}_{k \ne n}\big)$ for any $n$, where $\overline{span}$ denotes the closed linear span.
    The sequence $\fsys$ is said to be \emph{complete} when it spans the whole space $\overline{span}\big(\{f_k\}\big)=\cal{H}$.
    The system $\fstarsys = \{f_l^*\}_{l=0}^\infty$ is \emph{biorthogonal} if for any $k,l \geq 0$ we have $\inner{f_k}{f^*_l} = \delta_{kl}$, where $\delta$ is the Kronecker delta.
    It is known that $\fsys$ is complete and minimal when and only when it possesses a unique biorthogonal system $\fstarsys$.
    We call the minimal system $\fsys$ \emph{band-diagonal} if there exists $L \in \mathbb{N}$ such that $\inner{f_t}{e_l} = \inner{f^*_t}{e_l} = 0$
      whenever $\lvert t - l \rvert > L$.
    We say that $\fsys$ is an $M$-basis if $\fsys$ is complete and $\fstarsys$ is complete as well.

    Consider the operator algebra $\cal{A} = \{T\in B(\cal{H}): Tf_n = \lambda_n f_n, \text{ for some } \lambda_n \in \mathbb{R}, n \geq 0\}$
      and the algebra $R_1(\cal{A})$ generated by rank one operators of $\cal{A}$.
    We are interested in the following properties of the algebra $\cal{A}$.
    \begin{definition}[$k$ point density property]
      \label{kpd}
      We say that the algebra $\cal{A}$ has \emph{$k$ point density property} if for any $x_1, x_2,\dots x_k \in \cal{H}$ and $\varepsilon > 0$
        there exists $R\in R_1(\cal{A})$ such that $||Rx_s - x_s|| < \varepsilon$ for any $1 \leq s \leq k$.
    \end{definition}
    The definition for $k=1$ is equivalent to $\fsys$ being a \emph{strong $M$-basis} (see~\cite{katavolos}):
      the system $\fsys$ is called a \emph{strong $M$-basis} if for any $x\in\cal{H}$ we have $x \in \overline{span}\Big(\big\{\inner{x}{f^*_n}f_n\big\}_{n=0}^\infty\Big)$.
    \begin{definition}[rank one density property]
      \label{r1d}
      We say that the algebra $\cal{A}$ has \emph{rank one density property} if the unit ball of rank one subalgebra $R_1(\cal{A})$
        is dense in the unit ball of $\cal{A}$ in the strong operator topology.
    \end{definition}
    By abuse of notation, we say that $\fsys$ is $k$ point dense (rank one dense)
      when the corresponding algebra $R_1(\cal{A})$ is $k$ point dense (rank one dense).

    Notice that rank one density property implies $k$ point density property for any $k$.

  %\subsection{Motivation}
    %Long\-staff in~\cite{longstaff} studied abstract subspace lattices and corresponding operator algebras.
    %In that paper Longstaff raised an important question: does one point density property always imply rank one density property?

    %The solution remained unknown until Larson and Wogen showed~\cite{larson} that the answer is negative.
    %They constructed an example of a vector system $\fsys$ such that it is one point dense but does not possess rank one density property.
    %\begin{example}[Larson--Wogen system $\fsys_{LW}$ parameterized with real $a_n$]
      %\label{lw-sys}
      %For any $j \geq 0$ we define
      %\begin{align*}
        %&f_{2j+1}=-a_{2j+1}e_{2j} + e_{2j+1} + a_{2j+2}e_{2j+2} \qquad &f_{2j}=e_{2j},\\
        %&f^*_{2j}=-a_{2j}e_{2j-1} + e_{2j} + a_{2j+1}e_{2j+1} \qquad &f^*_{2j+1}=e_{2j+1},
      %\end{align*}
      %where $a_n$ are nonzero real numbers for any $n > 0$ and $a_0 = 0$.
    %\end{example}
    %The construction presented by Larson and Wogen was remarkably simple and elementary,~--- notice that the matrices corresponding to
      %the vectors $\{f_j\}_{j=0}^\infty$ and $\{f^*_j\}_{j=0}^\infty$ are both tridiagonal.
    %Afterwards this example was also studied in~\cite{argyroslambrou} (see Addendum), by Azoff and Shehada in~\cite{azoff}, in~\cite{me1}.
    %In 1993 Katavolos, Lambrou and Papadakis in~\cite{katavolos} performed a deep analysis of the density properties
      %of this vector system and deduced that for $\fsys_{LW}$ one point density does not imply rank one density.
    %Moreover, they showed that for such system rank one density is equivalent to two point density.

    %We are going to consider band-diagonal systems similar to the one regarded by Larson and Wogen and to determine the exact conditions
      %for $k$ point density property of such vector systems.
    %In this paper we present a few new techniques for the analysis of $k$ point density and rank one density of band-diagonal vector systems.

\section{Chain of reflexive closures for B-class}
  %\label{sec:preliminaries}
  \begin{definition}
    Let $U$ be a linear subspace of operators.
    Then $ref_k(U) = (F_k \cap R_\perp)^{\perp}$ is called \emph{$k$-reflexive closure} of $U$.
    Here $F_k$ is a linear subspace of all rank $k$ operators, $U_\perp$ means the preannihilator of $U$ and
      $U^\perp$ denotes the annihilator of $U$ (both with respect to the weak star topology induced by trace).
  \end{definition}
  It is well-known that $ref_{k + 1}(U) \subseteq ref_k(U)$ for any linear subspace $U$.

  We will denote the weak star closure of $R$ with $w_0(R)$ and the weak star closure of $R$ and the identity operator with $w(R)$.
  Moreover for any linear subspace $U$ there is a chain of inclusions
  \[
    ref_1(U) \supseteq ref_2(U) \supseteq \dots \supseteq ref_k(U) \supseteq \dots \supseteq w_0(U),
  \]
  In their paper of 1993 Azoff and Shehada (TODO cite) suggested such algebra $U$, generated by
    mutually orthogonal idempotents that only one of inclusions was strict:
  \[
    ref_1(U) = ref_2(U) = \dots = ref_k(U) \supset ref_{k + 1}(U) = ref_{k+2}(U) = \dots = w_0(U),
  \]
    for any given $k > 0$.
  Also they provided an example where all the inclusions were strict.
  This way it was shown that this chain of inclusions can be very different for various $U$.
  The main idea of that paper is tightly connected to the technique in the famous paper of Wogen (TODO cite, "Some counterexamples..."),
    exploiting the infinite matrix and the shift operator.

  For us the first important thing is that the idempotents in their construction were not necessarily rank one operators.
  The second thing is that in their construction the identity operator did not belong even to $ref_1(U)$;
    that is the most intriguing question for us.
  Indeed, the rank one density property is the same as $I \in w_0(R)$,
    while the $k$ point density property is equivalent to having $I$ in $ref_k(R)$.

  Suppose that $\fsys=\{f_n\}_{n=0}^\infty$ is an arbitrary B-class $M$-basis and
    $\fstarsys = \{f^*_n\}_{n=0}^\infty$ is its biorthogonal sequence.
  Again, we consider the operator algebra $\cal{A} = \{T\in B(\cal{H}): Tf_n = \lambda_n f_n,
    \text{ for some } \lambda_n \in \mathbb{R}, n \geq 0\}$
    and the algebra $R(\cal{A})$ generated by rank one operators of $\cal{A}$.


  The next proposition shows that for B-class the chain of inclusion always contains one strong inclusion.
  \begin{prop}
    \label{prop:chain}
    Then $I \in ref_k(R)$ but $I \not\in ref_{k+1}(R)$ if and only if $R$ is $k + 1$-reflexive but not $k$-reflexive, moreover the
      following holds:
    \[
      \cal{A} = ref_1(R) = ref_2(R) = \dots = ref_k(R) \supset ref_{k + 1}(R) = ref_{k + 2}(R) = \dots = w_0(R).
    \]
  \end{prop}
  This proposition follows from the next one which we prove first.
  It can be seen as a generalization of the proposition of Azoff and Shehada in their paper (TODO cite).
  \begin{prop}
    $w(R)$ is 1-reflexive.
  \end{prop}
  \begin{proof}
    Let $s_n = \sum_{k=1}^n f^*_k \otimes f_k$ and $p_n = \sum_{k=1}^n e_k \otimes e_k$.
    \begin{lemma}
      We have 
      \begin{align*}
        &s_m p_n = p_n s_m = s_m, \qquad n > m + D,\\
        &s_m p_n = p_n s_m = p_n, \qquad n < m - D,
      \end{align*}
      where $D$ is the width of $\fsys$.
    \end{lemma}
    \begin{proof}
      TODO
    \end{proof}
    Suppose we have an operator $t \in w(R)_\perp$.
    It implies that the trace of $t$ is zero.
    We define a finite rank approximations of $t$:
    $t_n = p_n t p_n - \sum_{l = n - D}^{n + D} \alpha_{nl} e_l \otimes e_l,$
    where $\alpha_{nl} = tr(t p_l) - tr(t p_{l - 1})$.
    \begin{lemma}
      The sequence $t_n$ tends to $t$ in SOT.
    \end{lemma}
    \begin{proof}
      The first member $p_n t p_n$ tends to $t$ in SOT, and the sum of $\alpha_{nl}$ tends to zero,
        since $tr(t) = 0$.
    \end{proof}
    \begin{lemma}
      Each $t_n$ belongs to $w(R)_\perp$.
    \end{lemma}
    \begin{lemma}
     $t_n$ belongs to a span of rank one operators from $w(R)_\perp$.
    \end{lemma}
    \begin{proof}
      Notice that $t_n = t_n p_{n + D + 1} = t_n p_{n + D + 1} s_n = t_n s_n.$
      Since the $w(R)_\perp$ is closed under multiplication on the elements of the algebra $\cal{A}$,
        each rank one summand of $t_n s_n$ belongs to the preannihilator.
    \end{proof}
    The last lemma finishes the proof of the whole proposition.
  \end{proof}

  \begin{proof}[Proof of~\ref{prop:chain}]
  \end{proof}

\section{B-class: reordered basis view}
  It is conceivable that $\fsys$ is equal to $e_n$ for any even $n$.
  The same can be said about $\fstarsys$ (only $n$ must be odd)
  The idea is to regroup this two parts of $\fsys$ and see what happens.
  

  So lets look at the linear map $\Phi$ that performs the transition between ${e_n}$ and ${f_n}$.
  It is easy to see that it is a two-by-two block matrix:
  \[
    \Phi= \left( {
      \begin{array}{cc}
        I & A\\
        0 & I\\
      \end{array}
    }
  \right).
  \]


  %Now consider an operator $T$ which has a finite rank.
  %In this case we write $T$ as a finite sum $T = \sum_{s=1}^k y^s \otimes x^s$,
    %where $x^s, y^s \in \cal{H}$ and $y \otimes x$ denotes the rank one operator sending a vector $v \in \cal{H}$
    %to $\inner{v}{y}x$.

  %Let us define vectors $v_n$ and $u_n$ in $\mathbb{R}^k$ as follows:
  %\begin{equation*}
    %v_n = (x^1_n, x^2_n, \dots, x^k_n)\qquad
    %u_n = (y^1_n, y^2_n, \dots, y^k_n),
  %\end{equation*}
  %where $x^s_n = \inner{x^s}{e_n}$ and $y^s_n = \inner{y^s}{e_n}$.
  %Since $\inner{Te_m}{e_l} = \inner{u_m}{v_l}$ for any $m$ and $l$, we can rewrite $\Xi_n$ in terms of
    %the scalar products of $\{u_n\}_{n=0}^\infty$ and $\{v_n\}_{n=0}^\infty$.
  %In turn it means that~\eqref{eq:prop:ref} can be rewritten in terms of the scalar products
    %of $\{u_n\}_{n=0}^\infty$ and $\{v_n\}_{n=0}^\infty$.

  %Hence, the existence of $T$ would be reduced to the existence of
    %the vectors $u_n$, $v_n$ in $\mathbb{R}^k$ such that the sequences $\{\lvert u_n\rvert\}_{n=0}^\infty$,
    %$\{\lvert v_n\rvert\}_{n=0}^\infty$ are both square summable and~\eqref{eq:prop:ref} is satisfied.
  %Thus, instead of looking for $k$ vectors $x^s$ and $y^s$ in $\cal{H}$, we would look for an infinite sequence
    %of $k$-dimensional vectors $v_n$ and $u_n$ such that $\{v_n \}_{n=0}^\infty$, $\{u_n\}_{n=0}^\infty$ belong to $\ell^2(\mathbb{R}^k)$.
  %This is one of the key ideas in our method of analysing $k$ point density property for $\fsys$.

  %Thus, we have just found the following reformulation for $k$ point density property.
  %\begin{prop}
    %\label{prop:kreformulation}
    %The following two statements are equivalent:
    %\begin{enumerate}
      %\item there exists a $k$-dimensional operator $T$ which annihilates $R_1(\cal{A})$
        %such that $Tr\, T \neq 0$,
      %\item there exist vectors $\{u_n\}_{n=0}^\infty$, $\{v_n\}_{n=0}^\infty$ in $\ell^2(\mathbb{R}^k)$ such that
        %for the operator $\widehat{T} = \displaystyle\sum_{t,l=0}^\infty \inner{u_t}{v_l} e_t \otimes e_l$ we have
        %\begin{equation}
          %\label{eq:thm}
          %\Xi_n + \sum_{m=0}^n \inner{\widehat{T}e_m}{e_m} = 0
        %\end{equation}
          %for any $n \geq 0$.
    %\end{enumerate}
  %\end{prop}
  %As we already mentioned, the equation~\eqref{eq:thm} can be expressed via $u_n$ and $v_n$.
  %Moreover, we can also write the trace of $T$ in terms of $u_n$, $v_n$:
  %%\begin{prop*}
    %\begin{equation}
      %\label{eq:propstar}
      %%\begin{align*}
        %Tr\,T = \sum_{s=1}^k \langle y^s, x^s \rangle = \sum_{s=1}^k \sum_{n=0}^\infty y^s_n x^s_n
              %= \sum_{n=0}^\infty \sum_{s=1}^k y^s_n x^s_n = \sum_{n=0}^\infty \langle u_n, v_n \rangle.
      %%\end{align*}
    %\end{equation}
  %%\end{prop*}
  %Essentially, the $k$ point density property can be viewed as a
    %possibility of placing the sequence of vectors in $\mathbb{R}^k$ which are constrained with
    %a series of relations~\eqref{eq:thm} and~\eqref{eq:propstar}.
%\section{Classification for the Larson--Wogen $M$-basis}
  \label{section:lw-sys}
  In this section we study Larson--Wogen vector system $\fsys_{LW}$ (Example~\ref{lw-sys}).
  Namely, we prove a theorem similar to Theorem 2.2 of~\cite{katavolos}.
  Up until now there existed two different techniques in studying $k$ point density, one for $k=1$ (strong $M$-bases)
    and a different one for $k\geq2$.
  Here we demonstrate a universal method for the analysis of $k$ point density property.
  \begin{theorem}[\cite{katavolos}, Theorem 2.2]
    \label{thm:katavolos}
    The sequences $\fsys_{LW}$ and $\fsys_{LW}^*$ are biorthogonal and both are complete in $\cal{H}$.
    Moreover, the following is true.
    \begin{enumerate}
      \item  the system $\fsys_{LW}$ is one point dense (a strong $M$-basis) if and only if the sequence
        \begin{equation}
          \mu_n = \frac{a_{n-1} a_{n-3} \dots}{a_{n} a_{n-2} \dots }
        \end{equation}
        does not belong to $\ell^2$.
      \item the system $\fsys_{LW}$ is $k$ point dense ($k > 1$) if and only if the sequence $\{1/a_n\}_{n=1}^\infty$ does not belong to $\ell^1$.
    \end{enumerate}
  \end{theorem}
  \begin{proof}
    Due to Proposition~\ref{prop:kreformulation} we know that $k$ point density of the system $\fsys$ is equivalent to the existence of $k$-dimensional
      vectors $u_n$, $v_n$ such that~\eqref{eq:thm} holds for the corresponding operator $T$.
    %We reproduce the logic from the previous section and define $\Xi_n$, $u_n$, $v_n$ exactly in the way it is described there.
    For the given $M$-basis $\fsys = \fsys_{LW}$ we can calculate $\Xi_n$ precisely:
    \begin{align*}
      \Xi_{2n-1} &= a_{2n}T_{2n - 1, 2n},\\
      \Xi_{2n} &= a_{2n + 1}T_{2n + 1, 2n},
    \end{align*}
      where $T_{ij} = \inner{Te_j}{e_i}$.

    Since $T_{ij} = \inner{u_j}{v_i}$, we have
    \begin{align*}
      \Xi_{2n-1} &= a_{2n}   \inner{u_{2n}}{v_{2n-1}},\\
      \Xi_{2n}   &= a_{2n+1} \inner{u_{2n}}{v_{2n+1}},
    \end{align*}
      where $\langle\cdot, \cdot\rangle$ denotes the scalar product in $\mathbb{R}^k$.

    For the convenience of the reader we will introduce the sequences of vectors $w_n$ and $w^*_n$.
    \begin{align*}
      w_{2n} &= u_{2n}, \quad w^*_{2n} = v_{2n},\\
      w_{2n+1} &= v_{2n+1}, \quad w^*_{2n+1} = u_{2n+1}.
    \end{align*}
    In view of this notation $\Xi_n = a_{n+1} \inner{w_n}{w_{n+1}}$ and due to Equation~\eqref{eq:propstar}
      $Tr\, T = \sum_{m=0}^\infty \inner{w_m}{w^*_m}$.

    Thus, we get that $\fsys$ is not $k$ point dense if and only if there exist
      $k$-dimensional vectors $\{w_n\}_{n=0}^\infty$, $\{w^*_n\}_{n=0}^\infty$ lying in $\ell^2(\mathbb{R}^k)$ such that
    \begin{equation}
      \label{eq:vector2}
      a_{n+1} \inner{w_n}{w_{n+1}} = -\sum_{m=0}^n \inner{w_m}{w^*_m},
    \end{equation}
      for any $n \geq 0$, and $\sum_{m=0}^\infty \inner{w_m}{w^*_m} \neq 0$.

    \medskip
    In what follows we show that the latter can be simplified even more.
    \begin{prop}
      \label{prop:reformulation-lw}
      The system $\fsys$ is not $k$ point dense if and only if there exists a sequence of vectors
        $\{r_n\}_{n=0}^\infty$ in $\ell^2(\mathbb{R}^k)$ such that
      \begin{equation}
        \label{eq:vector3}
        a_{n+1} \inner{r_n}{r_{n+1}} = 1,
      \end{equation}
        for any $n \geq 0$.
    \end{prop}
    \begin{proof}
      Suppose we found such $r_n$.
      Then we solve~\eqref{eq:vector2} by putting $w^*_n$ to zero, $w_n$ to $r_n$ for any $n > 0$ and
        choosing the vector $w^*_0$ so that $\inner{w_0}{w^*_0} = -1$.

      Now we prove the converse.
      Suppose we found such $w_n$ that~\eqref{eq:vector2} holds.
      Given that the vectors $w_n$ lie in $\mathbb{R}^k$, we rewrite the scalar product as
        the product of the vector lengths and the cosine of the angle between the vectors.
      Namely, we define $W_n = \lvert w_n\rvert$ and real $\theta_n$ that
        $\inner{w_{n}}{w_{n+1}} = W_n W_{n+1} \cos{\theta_n}.$

      The sequence $\Xi_n = -\sum_0^n \inner{w_m}{w^*_m}$ has a non-zero limit, so let us
        find the largest $N > 0$ such that $\Xi_N = 0$.
      Then we can modify the original sequence by setting $w_n$, $w^*_n$ to zero for any $0 \leq n \leq N$ so that~\eqref{eq:vector2}
        still holds.
      Therefore, without loss of generality we can assume that $\Xi_n \neq 0$ for any $n \geq 0$.
      %\begin{prop}
        %For a given system $k$ point density is equivalent to the existence such $0 < a'_n \leq a_n$ that
        %\[
          %\nu_n = \frac{a'_{n-1} a'_{n-3} \dots}{a'_{n} a'_{n-2} \dots }
        %\]
        %belongs to $\ell^2$.
      %\end{prop}
      Setting $a'_n = a_n \cos{\theta_n}$ we see that the sequence
      \[
        W_n = \frac{\Xi_{n-1}/a'_n}{\Xi_{n-2}/a'_{n-1}} \cdot \frac{\Xi_{n-3}/a'_{n-2}}{\Xi_{n-4}/a'_{n-3}} \cdots
      \]
        belongs to $\ell^2$.
      Now since $\Xi_n = -\sum_0^n \inner{w_m}{w^*_m}$, we discover that
      \[
        \frac{\Xi_n}{\Xi_{n-1}} = 1 + \eta_n,
      \]
        where $\{\eta_n\}_{n=1}^\infty \in \ell^1$.
      Thus the product of such $(1 + \eta_m)$ fractions is bounded by some constant above.
      It follows that the sequence
      \[
        W^\#_n = \frac{1/a'_n}{1/a'_{n-1}} \cdot \frac{1/a'_{n-2}}{1/a'_{n-3}} \cdots
      \]
        belongs to $\ell^2$.
      Now we set $r_n$ to $\frac{W^\#_n}{W_n}w_n$, and then~\eqref{eq:vector2} holds since
      \[
        a_{n+1} \inner{r_n}{r_{n+1}} = a_{n+1} \frac{1/a'_{n+1}}{\Xi_n/a'_{n+1}} \inner{w_n}{w_{n+1}} = 1.
      \]
      Since $\lvert r_n \rvert = \lvert W^\#_n \rvert$ and the sequence $\left\{\lvert W^\#_n \rvert\right\}_{n=1}^\infty$ belongs to $\ell^2$,
        the sequence $\{r_n\}_{n=1}^\infty$ belongs to $\ell^2(\mathbb{R}^k)$ as well.
    \end{proof}

    Now we are ready to prove the theorem for the case $k=1$.
    \begin{prop}
      The system $\fsys_{LW}$ is one point dense if and only if $\seqone{\mu}$ does not belong to $\ell^2$.
    \end{prop}
    \begin{proof}
      It follows from Proposition~\ref{prop:reformulation-lw}.

      The case $k=1$ has all the vectors $r_n$, $r^*_n$ lying on the same line ($\mathbb{R}^1$).
      Since all $r_n$ are collinear, the lengths of the vectors $r_n$ are precisely $\mu_n$.
      Hence, Equation~\eqref{eq:vector3} can be satisfied if and only if $\seqone{\mu}$ is square summable.
    \end{proof}

    After this we consider the case $k > 1$.
    \begin{prop}
      The system $\fsys_{LW}$ is $k$ point dense \textup($k > 1$\textup) if and only if the sequence $\{1/a_n\}_{n=1}^\infty$
        does not belong to $\ell^1$.
    \end{prop}
    \begin{proof}
      According to Proposition~\ref{prop:reformulation-lw}, the system $\fsys_{LW}$ is $k$ point dense
        if and only if there is no such sequence $\seq{r}$ in $\ell^2(\mathbb{R}^k)$ which satisfy $a_n \inner{r_n}{r_{n-1}} = 1$.
      Obviously, if there are such vectors $r_n$, then $\{1/a_n\}_{n=1}^\infty$ belongs to $\ell^1$.

      Conversely, suppose $\{1/a_n\}_{n=1}^\infty$ belongs to $\ell^1$.
      Then the sequence $R_n = \max(\lvert a_n \rvert^{-\frac{1}{2}}, \lvert a_{n+1} \rvert^{-\frac{1}{2}})$ is square summable.
      Observe that $R_nR_{n-1} \geq 1/\lvert a_n\rvert$, and so it is always possible to choose the angle $\theta_n$ so that
      \[
        a_n \langle r_n, r_{n-1} \rangle = a_n R_n R_{n-1}\cos{\theta_n} = 1.
      \]
      Now we have defined the lengths for $r_n$ and the angles between each two consecutive vectors $r_{n-1}$, $r_n$.
      Obviously, for any $k \geq 2$ we are able to place the vectors $r_n$ in $\mathbb{R}^k$.
    \end{proof}
    The last two propositions prove Theorem~\ref{thm:katavolos}.
  \end{proof}
  %\begin{remark}
    %The difference between two cases $k=1$ and $k\geq 2$ is in $\mathbb{R}^1$ any two vectors
      %have $0$ or $\pi$ angle between them.
  %\end{remark}
  % \begin{remark}
  %     Right here we can deduce the necessary and sufficient condition for the $k$-completeness property (for $k>1$) having the condition
  %     for the strong approximation property discovered earlier.
  %     As we remember $f_n$ approximates strongly iff the $a_n^{-1}$ is summable. Also we know as a fact that
  %     a system approximates strongly iff for any $k$ it is $k$-complete. Taking the last proposition into account we
  %     understand that the $k$-completeness condition is the same for all $k > 1$ and is identical to the strong approximation condition: $a_n^{-1}$ must be summable.
  %o  \end{remark}

%\section{Pentadiagonal example}
  \label{section:pentadiagonal}
  In this section we explore another vector system $\fsys$ and its biorthogonal system $\fstarsys$ defined as follows:
  \begin{equation*}
    \label{eq:5system}
    \begin{aligned}
      &\mathbf{f_{4j}} = e_{4j}, \quad
      \mathbf{f^*_{4j}} = e_{4j} + d_{2j - 1} e_{4j-2} - b_{2j-1} e_{4j-1} + a_{2j} e_{4j+1} + c_{2j} e_{4j+2}\\
      &\mathbf{f_{4j+1}} = -a_{2j} e_{4j} + e_{4j+1}, \quad
      \mathbf{f^*_{4j+1}} = e_{4j+1} + b_{2j} e_{4j+2},\\
      &\mathbf{f_{4j+2}} = e_{4j+2} + d_{2j} e_{4j} - b_{2j} e_{4j+1} + a_{2j+1} e_{4j+3} + c_{2j+1} e_{4j+4},\quad
      \mathbf{f^*_{4j+2}} = e_{4j+2},\\
      &\mathbf{f_{4j+3}} = e_{4j+3} + b_{2j+1} e_{4j+4},\quad
      \mathbf{f^*_{4j+3}} = -a_{2j+1} e_{4j+2} + e_{4j+3},
    \end{aligned}
  \end{equation*}
    where the real coefficients $a_n$, $b_n$, $c_n$, $d_n$ are equal to zero whenever $n < 0$, and satisfy the equality
      $c_n + d_n = a_n b_n$ for any $n \geq 0$.
  \begin{prop}
    The given system is an $M$-basis.
  \end{prop}
  \begin{proof}
    The equality $c_n + d_n = a_n b_n$ guarantees the bi\-orthogonality,
      while the completeness of $\fsys$ and $\fstarsys$ is easy to check.
  \end{proof}

  We prove a theorem similar to Theorem~\ref{thm:katavolos}, though we do not investigate the case $k = 1$ in this section.
  %\section{Main result}
    \begin{theorem}
    \label{thm:5diag}
      The following statements are equivalent:
      \begin{enumerate}
        \item the given system is rank one dense,
        \item the given system is $k$ point dense for some (equivalently any) $k > 1$,
        \item the sequence
          \[
            \mu_n = \min\left(\frac{1}{\lvert a_n \rvert} + \frac{1}{\lvert b_n \rvert}, \frac{1 + \lvert b_n\rvert}{\lvert d_n\rvert},
                    \frac{1 + \lvert a_n\rvert}{\lvert c_n\rvert}\right)
          \]
        does not belong to $\ell^1$.
      \end{enumerate}
    \end{theorem}
    \begin{proof}
      In order to investigate the density properties we repeat the reasoning from Section~\ref{sec:preliminaries}.
      Presume that $\Xi_n$ are defined by~\eqref{eq:xi}.

      Thus, for any $j \geq 0$ we have
      \begin{equation}
        \label{eq:xi5}
        \begin{aligned}
          \Xi_{4j} &= a_{2j} T_{4j+1, 4j} + c_{2j} T_{4j+2, 4j},\\
          \Xi_{4j + 1} &= -d_{2j} T_{4j+2, 4j} + b_{2j} T_{4j+2, 4j+1},\\
          \Xi_{4j + 2} &= a_{2j+1} T_{4j+2, 4j+3} + c_{2j+1} T_{4j+2, 4j+4},\\
          \Xi_{4j + 3} &= -d_{2j+1} T_{4j+2, 4j+4} + b_{2j+1} T_{4j+3, 4j+4},
        \end{aligned}
      \end{equation}
        where $T_{ij}$ stands for $\langle Te_j, e_i \rangle$.

      First of all we investigate the conditions of rank one density property for $\fsys$.
      \begin{prop}
        \label{prop:inf-dim}
        The following statements are equivalent:
        \begin{enumerate}
          \item the system $\fsys$ is not rank one dense,
          %\item there exists an operator $T$ with the trace equal to $-1$ such that $\inner{Tf_n}{f_n^*} = 0$ for any $n \geq 0$,
          \item there exists an operator $T$ such that $Tr\,T = -1$ and for any $n \geq 0$ one has \[\Xi_n + \sum_{m=0}^n \inner{Te_m}{e_m} = 0,\]
          \item the sequence $\left\{\mu_n\right\}_{n=1}^\infty$ belongs to $\ell^1$.
        \end{enumerate}
      \end{prop}
      \begin{proof}
        The equivalence of the first two statements is due to Proposition~\ref{prop:reformulation}.
        We are going to prove the equivalence between the last two statements.

        Assume that $\left\{\mu_n\right\}_{n=1}^\infty \in \ell^1$; our purpose is to construct the required operator $T$.
        Let $T_{00}$ be equal to $-1$, and $T_{jj}$ be equal to zero for any $j > 0$.
        Next we consider three cases for each $n \geq 0$.

        \noindent\textbf{Case 1.}
        Suppose $\mu_n = 1/\lvert a_n\rvert + 1/\lvert b_n \rvert$.
        For $n=2j$ we set
        \[
          T_{4j+1,4j}=1/a_n, \quad T_{4j+2,4j} = 0, \quad T_{4j+2,4j+1}=1/b_n.
        \]
        That guarantees the equality $\Xi_{2n} = \Xi_{2n+1} = 1$.
        For $n=2j+1$ we set
        \[
          T_{4j+2,4j+3}=1/a_n, \quad T_{4j+2,4j+4} = 0, \quad T_{4j+3,4j+4}=1/b_n,
        \]
        which provides the equality $\Xi_{2n} = \Xi_{2n+1} = 1$.

        \medskip
        \noindent\textbf{Case 2.}
        Assume $\mu_n = (1 + \lvert b_n\rvert)/\lvert d_n\rvert$.
        For $n=2j$ we set
        \[
          T_{4j+1,4j} = b_{2j}/d_{2j}, \quad T_{4j+2,4j} = -1/d_{2j}, \quad T_{4j+2,4j+1} = 0.
        \]
        Again, we have $\Xi_{2n} = \Xi_{2n+1} = 1$.
        For $n = 2j + 1$ we set
        \[
          T_{4j+2,4j+3}=b_{2j+1}/d_{2j+1},  \quad T_{4j+2,4j+4} = -1/d_{2j+1}, \quad T_{4j+3,4j+4}=0,
        \]

        The third case $\mu_n = (1 + \lvert a_n\rvert)/\lvert c_n\rvert$ is left to the reader.
        \medskip

        All the other entries $T_{ij}$ we set to zero.
        These equalities ensure that $\Xi_n = -\sum_{s=0}^n T_{ss} = 1$ for any $n \geq 0$.

        The constructed operator $T$ belongs to the trace class since the non-zero operator matrix entries are summable
          due to the assumption that $\left\{\mu_n\right\}_{n=1}^\infty \in \ell^1$.
        Since the trace of $T$ is equal to $-1$, the sufficiency is proved.

        \medskip
        Conversely, assume that there exists a trace class operator $T$ in the annihilator of $R_1(\cal{A})$ with the trace equal to $-1$.

        First of all, we prove that $\{\mu_{2j}\}_{j=1}^\infty$ is a summable sequence.
        Since $T$ is in the trace class, the sequence of vectors $\nu_n = \lvert T_{nn} \rvert + \lvert T_{n, n + 1} \rvert + \lvert T_{n, n+2} \rvert$
          belongs to $\ell^1$.
        Obviously, $\{\Xi_n\}_{n=1}^\infty$ belongs $\ell^1$ as well.
        As a consequence, we have $\lvert \Xi_n\rvert \geq 0.5$ for all $n$ large enough; we will assume that it holds for any $n > 0$.

        It can be easily checked that if for some $n$ one of the numbers $a_n,b_n,c_n,d_n$ is equal to zero, then $\nu_n \geq \mu_n/2$.
        From this point we will suppose that the coefficients are nonzero for any $n > 0$.

        For any even $n = 2j$ consider the linear function
        \begin{align*}
          g_{n}(x) = \Big\lvert \frac{\Xi_{2n} - c_{n} x}{a_{n}} \Big\rvert +
                       \lvert x \rvert +
                     \Big\lvert \frac{\Xi_{2n+1} + d_{n} x}{b_{n}} \Big\rvert.
        \end{align*}
        Obviously, we have $\nu_{n} = g_{n}(T_{2n+2, 2n})$.
        %\begin{align*}
          %S_{2j} = \Big\lvert \frac{\Xi_{4j} - c_{2j} T_{4j+2, 4j}}{a_{2j}} \Big\rvert +
                   %\Big\lvert T_{4j+2, 4j} \Big\rvert +
                   %\Big\lvert \frac{\Xi_{4j+1} + d_{2j} T_{4j+2, 4j}}{b_{2j}} \Big\rvert.
        %\end{align*}
        The function $g_n$ is piecewise linear, so its minimum is attained in the breakpoints.
        The breakpoints are zero, $y_n = \Xi_{2n}/c_n$ and $z_n = -\Xi_{2n+1}/d_n$.
        We have $g_n(0) \geq \mu_n/2$.
        Consider the set $N_1 \subseteq \mathbb{N}_{{even}}$, such that for any $n \in N_1$ the function $g_n$ attains its minimum in
          the point $\Xi_{2n}/c_n$.
        Thus, for any $n\in N_1$ we have $\nu_n \geq g_n(y_n)$.
        We have
        \[
          g_n(y_n) = \Big\lvert \Xi_{2n}/c_n \Big\rvert +
                   \Big\lvert \frac{\Xi_{2n+1} + d_{n} \Xi_{2n}/c_n}{b_{n}} \Big\rvert.
        \]
        Since $\nu_n$ is summable and $\nu_n \geq g_n(y_n) \geq 0.5/|c_n|$ for any $n \in N_1$
          we deduce that $\sum_{n \in N_1} |c_n|^{-1} < \infty$.

        Clearly, $G_n = g_n(y_n) - \Big\lvert \Xi_{2n}/c_n \Big\rvert$ is summable.
        Let $\Delta_n$ stand for the difference $\left(\Xi_{2n+1} - \Xi_{2n}\right)$.
        Then
        \[
          G_n = \Big\lvert \frac{c_n \Xi_{2n+1} + d_{n} \Xi_{2n}}{c_n b_n} \Big\rvert
              = \Big\lvert \frac{c_n \Delta_n + (c_n + d_n) \Xi_{2n}}{c_n b_n} \Big\rvert
              = \Big\lvert \frac{c_n \Delta_n + a_n b_n\Xi_{2n}}{c_n b_n} \Big\rvert.
        \]
        Hence, $\big\lvert G_n - \Xi_{2n}\lvert a_n/c_n\rvert \big\rvert \leq \lvert \Delta_n/b_n \rvert$.

        Consider the sets $N_2 = \{n\in N_1 \mid 0.5 \leq \lvert b_n \rvert\}$ and $N_3 = N_1 \setminus N_2$.
        Since $\Xi_n$ has a finite limit, we have $\sum_{n \in N_2} \lvert a_n/c_n \rvert < \infty$.
        Hence, $\{\mu_n\}_{n \in N_2} \in \ell^1$.

        Assume that $\sum_{n \in N_3} \lvert a_n/c_n\rvert = \infty$.

        We have $\big\lvert |b_n| G_n - \Xi_{2n}\lvert (c_n + d_n)/c_n \rvert\big\rvert \leq \lvert \Delta_n \rvert$.
        Since the sequences $\{b_n G_n\}_{n\in N_3}$ and $\{\Delta_n\}_{n \in N_3}$ are absolutely summable, the sequence
          $\{(c_n + d_n) / c_n\}_{n\in N_3} $ is absolutely summable as well.
        Consequently, $\lvert d_n/c_n \rvert \geq 0.5$ when $n$ is large enough.
        We get $\{1/c_n\}_{n \in N_3} \in \ell^1$, and thus $\{1/d_n\}_{n \in N_3} \in \ell^1$.
        Since for $n \in N_3$ one has $\lvert b_n \rvert \leq 0.5$, we have
        \[
          \sum_{n\in N_3} \mu_n \leq \sum_{n \in N_3} \frac{1 + \lvert b_n\rvert}{\lvert d_n \rvert} < \infty.
        \]

        Repeating the reasoning for odd $n$, we get that $\left\{\mu_n\right\}_{n=1}^\infty$ is a summable sequence.
      \end{proof}
      \bigskip
      Now consider the case of the $k$-dimensional operator $T = \sum_{s=1}^k y^s \otimes x^s$, where $x^s, y^s \in \cal{H}$.
      This time we define the vectors $\seq{v}$ and $\seq{u}$ in $\mathbb{R}^k$ as follows:
      \begin{align*}
        v_{2j} &= (y^1_{4j}, y^2_{4j}, \dots ,y^k_{4j}) \quad
        &v^*_{2j} = (x^1_{4j}, x^2_{4j}, \dots ,x^k_{4j}) \\
        v_{2j+1} &= (x^1_{4j+2}, x^2_{4j+2}, \dots ,x^k_{4j+2}) \quad
        &v^*_{2j+1} = (y^1_{4j+2}, y^2_{4j+2}, \dots ,y^k_{4j+2}) \\
        u_{2j} &= (x^1_{4j+1}, x^2_{4j+1}, \dots ,x^k_{4j+1}) \quad
        &u^*_{2j} = (y^1_{4j+1}, y^2_{4j+1}, \dots ,y^k_{4j+1}) \\
        u_{2j+1} &= (y^1_{4j+3}, y^2_{4j+3}, \dots ,y^k_{4j+3}) \quad
        &u^*_{2j+1} = (x^1_{4j+3}, x^2_{4j+3}, \dots ,x^k_{4j+3})
      \end{align*}
      Note that the sequences $\seq{v}$, $\seq{u}$ belong to $\ell^2(\mathbb{R}^k)$.

      Now we can rewrite the equations~\eqref{eq:xi5} using the introduced vectors:
      \begin{align*}
        \Xi_{4j} &= a_{2j} \langle u_{2j}, v_{2j}\rangle + c_{2j} \langle v_{2j+1}, v_{2j}\rangle,\\
        \Xi_{4j + 1} &= -d_{2j} \langle v_{2j+1}, v_{2j}\rangle + b_{2j} \langle v_{2j+1}, u_{2j}\rangle,\\
        \Xi_{4j + 2} &= a_{2j+1} \langle v_{2j+1}, u_{2j+1} \rangle + c_{2j+1} \langle v_{2j+1}, v_{2j+2} \rangle,\\
        \Xi_{4j + 3} &= -d_{2j+1} \langle v_{2j+1}, v_{2j+2}\rangle + b_{2j+1} \langle u_{2j+1}, v_{2j+2} \rangle,
      \end{align*}
        where $\langle\cdot, \cdot\rangle$ denotes the scalar product in $\mathbb{R}^k$.
      These equations simplify to
      \begin{equation}
        \label{eq:vector-eqs}
        \begin{aligned}
          \Xi_{2j} &= a_{j} \langle u_{j}, v_{j} \rangle  + c_{j} \langle v_{j+1}, v_{j} \rangle,\\
          \Xi_{2j + 1} &= -d_{j} \langle v_{j+1}, v_{j} \rangle + b_{j} \langle v_{j+1}, u_{j}\rangle.
        \end{aligned}
      \end{equation}

      Next we are going to analyze the necessary condition of $k$ point density property for $\fsys$.
      \begin{prop}
        \label{prop:2pd}
        If $\left\{\mu_n\right\}_{n=1}^\infty$ belongs to $\ell^1$ then it is possible to construct
          the vector sequences $\seq{u}, \seq{v} \in \ell^2(\mathbb{R}^2)$ such that for any $n \geq 0$ we have $\Xi_n = 1$.
      \end{prop}
      \begin{corol}
        \label{corol:2density}
        If $\fsys$ is $k$ point dense for any $k \geq 2$ then $\sum_{n=1}^\infty \lvert\mu_n\rvert = \infty$.
      \end{corol}
      \begin{proof}[Proof of the corollary]
        Assume the converse: $\{\mu_n\}_{n=1}^\infty \in \ell^1$.

        We apply the proposition and get the vectors $u_n$ and $v_n$.
        Now without loss of generality we can assume that $u_0 \neq 0$.
        Then consider $u^*_0$ so that $\inner{u_0}{u^*_0} = -1$ and set all $u^*_n$ ($n > 0$), $v^*_n$ to zero.
        Since the trace of the resulting operator $T$ is equal to $\sum_{n=0}^\infty \left(\inner{u_n}{u^*_n} + \inner{v_n}{v^*_n}\right) \neq 0$,
          Proposition~\ref{prop:kreformulation} implies that $\fsys$ is not two point dense.
        Trivially, when $\fsys$ is not two point dense, it is also not $k$ point dense for any $k \geq 2$.
      \end{proof}
      \begin{proof}[Proof of Proposition~\ref{prop:2pd}]
        %Here again we are going to look at three possible values of the $\mu_n$.
        First we are going to present the vector lengths $V_n = \lvert v_n \rvert$ for each $n \geq 0$.
        %and three angles between $v_n$, $v_{n+1}$ and $u_n$:
        %\[
          %\alpha_n = \angle (v_n, v_{n+1}) \quad
          %\beta_n = \angle (v_n, v_{n+1}) \quad
          %\gamma_n = \angle (v_n, v_{n+1}).
        %\]
        %We will prove that the corresponding vectors could be settled in $\mathbb{R}^2$.

        %Within the construction we are going to set all $V_n$ step by step.
        For this purpose we are going to define an auxiliary sequence $\{M_n\}_{n=1}^\infty \in \ell^2$ such that $V_n \geq M_n$ for any $n$.
        On each step $n$ we will define $V_n$ and $M_{n+1}$.

        We start by setting $V_0 = M_1 = 1$.

        For any $n > 0$ we have three choices for $\mu_n$:
        \begin{enumerate}
          \item if $\mu_n = 1/\lvert a_n \rvert+ 1/\lvert b_n \rvert$, we set 
              \[
                M_{n+1} = \frac{1}{\sqrt{\smash[b]{\lvert b_n \rvert}}} \quad
                V_n = \max\left(M_n, \frac{1}{\sqrt{\smash[b]{\lvert a_n \rvert}}}\right),
              \]
          \item whenever $\mu_n = (1 + \lvert a_n \rvert)/\lvert c_n \rvert$, we set 
              \[
                M_{n+1} = \max\biggl(\smash[b]{\frac{\sqrt{|a_n|}}{\sqrt{\lvert c_n \rvert}}}, \frac{1}{\sqrt{\smash[b]{\lvert c_n \rvert}}}\biggr)\quad
                V_n = \max\left(M_n, \frac{2}{\sqrt{\smash[b]{|c_n|}}}\right),
              \]
          \item if $\mu_n = (1 + \lvert b_n \rvert)/\lvert d_n \rvert$, we set 
              \[
                M_{n+1} = \max\biggl(\smash[t]{\frac{\sqrt{\lvert b_n \rvert}}{\sqrt{\lvert d_n \rvert}}}, \frac{1}{\sqrt{\smash[b]{\lvert d_n \rvert}}}\biggr)\quad
                V_n = \max\left(M_n, \frac{2 + \sqrt{\lvert b_n\rvert}}{\sqrt{\smash[b]{\lvert d_n \rvert}}}\right).
              \]
        \end{enumerate}
        Now all $V_n$, $M_n$ are set, and obviously $V_n \geq M_n$ for any $n > 0$.
        
        Next we are going to present the vector lengths $U_n = \lvert u_n \rvert$, for each $n \geq 0$.
        Set $U_0$ to zero and for any $n > 0$ we have three cases again:
        \begin{enumerate}
          \item when $\mu_n = 1/\lvert a_n \rvert+ 1/\lvert b_n \rvert$, we set 
              $
                U_n = \displaystyle\sqrt{\frac{1}{a_n^2 V_n^2} + \frac{1}{b_n^2 V_{n+1}^2}},
              $ 
          \item whenever $\mu_n = (1 + \lvert a_n \rvert)/\lvert c_n \rvert$, we set 
              $
                U_n =  \dfrac{a_n V_n}{\displaystyle\sqrt{c_n^2 V_{n+1}^2 V_n^2 - 1}},
              $
          \item if $\mu_n = (1 + \lvert b_n \rvert)/\lvert d_n \rvert$, we set 
              $
                U_n = \dfrac{b_n V_{n+1}}{\displaystyle\sqrt{ d_n^2 V_{n+1}^2 V_n^2 - 1}}.
              $
        \end{enumerate}

        Due to our choice of $V_n$, $M_n$ the values $U_n$ are well-defined for each $n > 0$.

        %In the following we prove a simple lemma.
        \begin{lemma}
          \label{lemma}
          If for some nonzero real $A$, $B$, $X$, $Y$, $Z$ we have $(AX)^{-2} + (BY)^{-2} = Z^2$, there exist such vectors $x$, $y$, $z$ with lengths
            $X$, $Y$, $Z$ correspondingly such that
          \begin{equation}
            \label{eqn:system}
            \begin{aligned}
              \langle x, y \rangle &= 0,\\
              \langle x, z \rangle &= 1/A,\\
              \langle y, z \rangle &= 1/B.
            \end{aligned}
          \end{equation}
        \end{lemma}
        \begin{proof}
          Take $\alpha$ such that $\cos \alpha  = 1/(AXZ)$ and $\sin \alpha  = 1/(BYZ)$.
          Consider three vectors $x$, $y$, $z$ in $\mathbb{R}^2$ with lengths $X$, $Y$, $Z$ such that
            $\angle(y, z) = \pi/2 - \alpha$ and $\angle(x, z) = \alpha$.
          Clearly, the vectors $x$ and $y$ must be orthogonal now.
          The equations~\eqref{eqn:system} are trivial to check.
        \end{proof}
        \begin{prop}
          For any $n\geq 0$ there are vectors $u_n$, $v_n$ with lengths $U_n$, $V_n$ in $\mathbb{R}^2$ such that~\eqref{eq:vector-eqs} are satisfied.
        \end{prop}
        \begin{proof}
          %As a result there always be such $\beta_n$ and $\gamma_n$ that
          %\begin{align*}
            %\lvert U_n \cos{\beta_n} \rvert &= \frac{1}{\lvert a_n \rvert V_n},\\
            %\lvert U_n \cos{\gamma_n}\rvert &= \lvert U_n \sin{\beta_n}\rvert = \frac{1}{\lvert b_n \rvert V_{n+1}}.
          %\end{align*}
          We argue by induction.
          We start with $v_0 = (0, 1)$ and $u_0 = 0$.
          Suppose that we have constructed a sequence of vectors $v_m, u_m$ for all $m < n$ and $v_n$.
          We are going to build $u_n$ and $v_{n+1}$.
          We consider three cases for $\mu_n$.

          In the first case the chosen $U_n$, $1/\lvert a_n V_n \rvert$ and $1/\lvert b_n  V_{n+1}\rvert$
              form a right triangle with hypotenuse $U_n$, and so here Lemma~\ref{lemma} can be applied.
          It follows that there are vectors $u'_n$, $v'_n$, $v'_{n+1}$ in $\mathbb{R}^2$ with lengths $U_n$, $V_n$, $V_{n+1}$ correspondingly such that
          \begin{equation}
            \label{eq:system1}
            \begin{aligned}
              \langle u'_n, v'_n \rangle &= 1/a_n,\\
              \langle u'_n, v'_{n+1} \rangle &= 1/b_n,\\
              \langle v'_n, v'_{n+1} \rangle &= 0,
            \end{aligned}
          \end{equation}
          which in turn yields the equations~\eqref{eq:vector-eqs}.
          Now we can simply rotate the triple $(u'_n, v'_n, v'_{n+1})$ so that $v'_n$ coincides with $v_n$.
          We will set $u_n$ and $v_{n+1}$ to the rotated $u'_n$ and $v'_{n+1}$ accordingly.
          Since the rotation preserves the scalar product inside the triple, the equations~\eqref{eq:system1} hold for $u_n$, $v_n$, $v_{n+1}$ as well.

          In the second case the chosen $V_{n+1}$, $a_n/(c_n U_n)$ and $1/(c_n V_n)$ also form a right triangle and Lemma~\ref{lemma} applies here
            as well.
          It implies that there are vectors $u'_n$, $v'_n$, $v'_{n+1}$ in $\mathbb{R}^2$ with lengths $U_n$, $V_n$, $V_{n+1}$ correspondingly such that
          \begin{equation}
            \begin{aligned}
              \langle u'_n, v'_n \rangle &= 0,\\
              \langle u'_n, v'_{n+1} \rangle &= a_n/c_n,\\
              \langle v'_n, v'_{n+1} \rangle &= 1/c_n,
            \end{aligned}
          \end{equation}
          and the equations~\eqref{eq:vector-eqs} follow from that.
          Using rotation again, we receive $u_n$, $v_n$, $v_{n+1}$.

          In the third case the chosen $V_n$, $b_n/(d_n U_n)$ and $1/(d_n V_{n+1})$ also form a right triangle and Lemma~\ref{lemma} applies here
            as well.
          It implies that there are vectors $u'_n$, $v'_n$, $v'_{n+1}$ in $\mathbb{R}^2$ with lengths $U_n$, $V_n$, $V_{n+1}$ correspondingly such that
          \begin{equation}
            \begin{aligned}
              \langle u'_n, v'_n \rangle &= b_n/d_n,\\
              \langle u'_n, v'_{n+1} \rangle &= 0,\\
              \langle v'_n, v'_{n+1} \rangle &= -1/d_n,
            \end{aligned}
          \end{equation}
          and the equations~\eqref{eq:vector-eqs} are also true.
          One more time we do the rotation, and we get $u_n$, $v_n$, $v_{n+1}$.
        \end{proof}
        \begin{prop}
          The following inequalities are true.
          \begin{align*}
            M_{n+1} &\leq \sqrt{\mu_n},\\
            V_n &\leq \max(2\sqrt{\mu_n}, M_n),\\
            U_n &\leq 2\sqrt{\mu_n}.
          \end{align*}
        \end{prop}
        \begin{proof}
          First two statements are trivial.
          For the last statement we consider the same three cases.

          In the first case we have $\lvert a_n V_n\rvert \geq \sqrt{\lvert a_n \rvert}$ and
            $\lvert b_n V_{n+1} \rvert \geq  \lvert b_n M_{n+1}\rvert = \sqrt{\lvert b_n \rvert}$.
          It follows that $U_n \leq \sqrt{\mu_n}$.

          In the second case we have $c_n V_{n+1} V_n \geq c_n M_{n+1} V_n \geq 2$ and so
          \[
            U_n \leq \dfrac{\lvert a_n\rvert V_n}{\sqrt{\frac{3}{4} c_n^2 V_{n+1}^2 V_n^2}} \leq 2\dfrac{\lvert a_n\rvert}{\lvert c_n\rvert V_{n+1}}
                \leq 2\dfrac{\lvert a_n\rvert}{\lvert c_n\rvert M_{n+1}} \leq 2\sqrt{\dfrac{\lvert a_n\rvert}{\lvert c_n\rvert}} < 2\sqrt{\mu_n}.
          \]

          In the third case we have $d_n V_{n+1} V_n \geq 2$ and hence
          \[
            U_n \leq \dfrac{\lvert b_n\rvert V_{n+1}}{\sqrt{\frac{3}{4} d_n^2 V_{n+1}^2 V_n^2}} \leq 2\dfrac{\lvert b_n\rvert}{\lvert d_n\rvert V_{n}}
                \leq 2\sqrt{\dfrac{\lvert b_n\rvert}{\lvert d_n\rvert}} < 2\sqrt{\mu_n}.
          \]
        \end{proof}

        %In each of three cases we guaranteed that $M_{n+1} \leq \sqrt{\mu_n}$ and that $V_n \leq \max(M_n, 2\sqrt{\mu_n})$.
        The last proposition implies that $V_n$ is bounded up to some constant by $\max(\sqrt{\mu_{n-1}}, \sqrt{\mu_n})$, and
          $U_n \leq \sqrt{\mu_n}$ for any $n > 0$.
        Hence, the constructed sequences $V_n$ and $U_n$ belong to $\ell^2$.
        That finishes the proof of Proposition~\ref{prop:2pd}.
      \end{proof}
      Now due to Corollary~\ref{corol:2density} we get that two point density of $\fsys$ implies the divergence of $\sum_{n=1}^\infty \lvert \mu_n\rvert$,
        which in turn is equivalent to rank one density property of $\fsys$ (see Proposition~\ref{prop:inf-dim}).
      Since rank one density implies $k$ point density for any $k$, Theorem~\ref{thm:5diag} is proved.
    \end{proof}

%\bigskip

%\section{Acknowledgements}
  %This is part of author's Ph.D. thesis, written under the supervision of Anton Baranov at the PDMI RAS.
  %The author gratefully acknowledges the many helpful suggestions of Anton Baranov during the preparation of the paper.

%\begin {thebibliography}{20}
  \bibitem{argyroslambrou}
    S.~\!Argyros, M.~\!Lambrou and W.E~\!Longstaff,
    \emph{Atomic Boolean Subspace Lattices and Applications to the Theory of Bases},
    Memoirs. Amer. Math. Soc., No. 445 (1991).

  \bibitem{azoff}
    E.~\!Azoff, H.~\!Shehada,
    \emph{Algebras generated by mutually orthogonal idempotent operators},
    J. Oper. Theory, 29 (1993), 2, 249--267.

  \bibitem{larson}
    D.~\!Larson, W.~\!Wogen,
    \emph{Reflexivity properties of $T\bigoplus0$},
    J. Funct. Anal., 92 (1990), 448--467.

  \bibitem{review}
    J.A~\!Erdos,
    \emph{Basis theory and operator algebras},
    In: A. Katavolos (ed.), Operator Algebras and Application, Kluwer Academic Publishers, 1997, pp. 209--223.

  \bibitem{katavolos}
    A.~\!Katavolos, M.~\!Lambrou and M.~\!Papadakis,
    \emph{On some algebras diagonalized by $M$-bases of $\ell^2$},
    Integr. Equat. Oper. Theory, 17 (1993), 1, 68--94.

  \bibitem{erdos}
    J.A~\!Erdos,
    \emph{Operators of finite rank in nest algebras},
    J. London Math. Soc., 43 (1968), 391--397.

  \bibitem{laurielongstaff}
    C.~\!Laurie, W.~\!Longstaff,
    \emph{A note on rank one operators in reflexive algebras},
    Proc. Amer. Math. Soc., 89 (1983), 293 - 297.

  \bibitem{longstaff}
    W.E~\!Longstaff,
    \emph{Operators of rank one in reflexive algebras},
    Canadian J. Math., 27 (1976), 19--23.

  \bibitem{bbb}
    A. Baranov, Yu. Belov and A. Borichev,
    \emph{Hereditary completeness for systems of exponentials and reproducing kernels},
    Adv. Math., 235 (2013), 1, 525--554.

  \bibitem{bbb1}
    A. Baranov, Yu. Belov and A. Borichev,
    \emph{Spectral synthesis in de Branges spaces},
    Geom. Funct. Anal. (GAFA), 25 (2015), 2, 417--452.

  \bibitem{ad_preprint}
    A.D.~\!Baranov, D.V.~\!Yakubovich,
    \emph{Completeness and spectral synthesis of nonselfadjoint one-dimensional perturbations of selfadjoint operators},
    arXiv:1212.5965 [math.FA].

% \bibitem{nikolski}
    %N.K.~\!Nikol'skii,
    %\emph{Complete extensions of Volterra operators},
    %Izv. Akad. Nauk SSSR Ser. Mat 33(1969), 1349--1355. (Russian)
\end{thebibliography}

\end{document}
