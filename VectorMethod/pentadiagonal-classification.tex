\section{Pentadiagonal example}
  \label{section:pentadiagonal}
  In this section we explore the vector system $\fsys$ and its biorthogonal system $\fstarsys$ defined as follows:
  \begin{equation}
    \label{main-system}
    \begin{aligned}
      &\mathbf{f_{4j}} = e_{4j} \quad
      \mathbf{f^*_{4j}} = e_{4j} + d_{2j - 1} e_{4j-2} - b_{2j-1} e_{4j-1} + a_{2j} e_{4j+1} + c_{2j} e_{4j+2}\\
      &\mathbf{f_{4j+1}} = -a_{2j} e_{4j} + e_{4j+1} \quad
      \mathbf{f^*_{4j+1}} = e_{4j+1} + b_{2j} e_{4j+2},\\
      &\mathbf{f_{4j+2}} = e_{4j+2} + d_{2j} e_{4j} - b_{2j} e_{4j+1} + a_{2j+1} e_{4j+3} + c_{2j+1} e_{4j+4}\quad
      \mathbf{f^*_{4j+2}} = e_{4j+2},\\
      &\mathbf{f_{4j+3}} = e_{4j+3} + b_{2j+1} e_{4j+4}\quad
      \mathbf{f^*_{4j+3}} = -a_{2j+1} e_{4j+2} + e_{4j+3},
    \end{aligned}
  \end{equation}
    where the real coefficients $a_n$, $b_n$, $c_n$, $d_n$ are equal to zero whenever $n < 0$, and satisfy the equality
      $c_n + d_n = a_n b_n$ for any $n \geq 0$.
  \begin{prop}
    The given system is an $M$-basis.
  \end{prop}
  \begin{proof}
    The equality $c_n + d_n = a_n b_n$ guarantees the bi\-orthogonality,
      while the completeness of $\fsys$ and $\fstarsys$ is easy to check.
  \end{proof}

  \section{Main result}
    \begin{theorem}
      The given system is NOT $k$ point dense for some (equivalently any) $k > 1$ if and only if the sequence
      \[
        \mu_n = \min\left(\frac{1}{|a_n|} + \frac{1}{|b_n|}, \frac{1 + |b_n|}{|d_n|}, \frac{1 + |a_n|}{|c_n|}\right)
      \]
        belongs to $\ell^1$.
    \end{theorem}
    \begin{proof}
      %Let us consider a $k$-dimensional operator $T$ such that 
        %$Tr(TR) = 0$ for each $R \in \cal{R}_1(\cal{A})$ which essentially means that
        %$\langle Tf_n, f_n^* \rangle = 0$ for any $n > 0$. 
      Notice that the partial sums of the Fourier series for the given system are somehow close to the
        partial sums of the canonical Fourier series (using the orthonormal basis $e_n$).
      Consider the following differences
      \[
        \Xi_n = \sum_0^n \langle Tf_s, f_s^* \rangle - \sum_0^n \langle Te_s, e_s \rangle,
      \]
        where $\langle \cdot, \cdot\rangle$ denotes the inner product in $\cal{H}$.
      For any $j \geq 0$ we have
      \begin{align*}
        \Xi_{4j} = a_{2j} T_{4j+1, 4j} + c_{2j} T_{4j+2, 4j},\\
        \Xi_{4j + 1} = -d_{2j} T_{4j+2, 4j} + b_{2j} T_{4j+2, 4j+1},\\
        \Xi_{4j + 2} = a_{2j+1} T_{4j+2, 4j+3} + c_{2j+1} T_{4j+2, 4j+4},\\
        \Xi_{4j + 3} = -d_{2j+1} T_{4j+2, 4j+4} + b_{2j+1} T_{4j+3, 4j+4},
      \end{align*}
      where $T_{ij}$ stands for $\langle Te_j, e_i \rangle$.
      \begin{prop}
        \label{inf-dim-statement}
        There exists an operator $T$ with the trace equal to $-1$ such that $\langle Tf_n, f_n^*\rangle = 0$ for any $n \geq 0$
          if and only if the sequence $\left\{\mu_n\right\}$ belongs to $\ell^1$.
      \end{prop}
      \begin{proof}
        Assume that $\mu_n \in \ell^1$.
        We will then construct a required operator $T$.
        Let $T_{00}$ be equal to $-1$, and $T_{jj}$ be equal to zero for any $j > 0$.

        Consider three cases for each $n \geq 0$.

        \noindent\textbf{Case 1.}
        Suppose $\mu_n = 1/|a_n| + 1/|b_n|$.
        For $n=2j$ we set:
        \begin{align*}
          T_{4j+1,4j}&=1/a_n & \quad T_{4j+2,4j} = 0,\\
          T_{4j+2,4j+1}&=1/b_n.
        \end{align*}
        That guarantees an equality $\Xi_{2n} = \Xi_{2n+1} = 1$.
        For $n=2j+1$ we set:
        \begin{align*}
          T_{4j+2,4j+3}&=1/a_n & \quad T_{4j+2,4j+4} = 0,\\
          T_{4j+3,4j+4}&=1/b_n,
        \end{align*}
        which provides an equality $\Xi_{2n} = \Xi_{2n+1} = 1$.

        \noindent\textbf{Case 2.}
        Assume $\mu_n = (1 + |b_n|)/|d_n|$. 
        For $n=2j$ we set
        \begin{align*}
          T_{4j+1,4j} &= b_{2j}/d_{2j} & \quad T_{4j+2,4j} = -1/d_{2j},\\
          T_{4j+2,4j+1} &= 0.
        \end{align*}
        Again, we have $\Xi_{2n} = \Xi_{2n+1} = 1$.
        The case $n = 2j + 1$ is left to the reader.

        \noindent\textbf{Case 3.}
        Suppose $\mu_n = (1 + |a_n|)/|c_n|$. 
        For $n = 2j + 1$ we assign:
        \begin{align*}
          T_{4j+2,4j+3} &= 0 & \quad T_{4j+2,4j+4} = 1/c_{2j+1},\\
          T_{4j+3,4j+4} &= a_{2j+1}/c_{2j+1}.
        \end{align*}
        The case $n = 2j$ is analogous.
        \medskip
        All the other entries $T_{ij}$ we set to zero.
        These assignments ensure that $\Xi_n = 1$ for any $n \geq 0$.

        The constructed operator $T$ belongs to the trace class since all the non-zero elements are summable 
          due to the assumption that $\left\{\mu_n\right\} \in \ell^1$.
        Notice that $T$ annihilates all the rank one operators $f^*_n \otimes f_n$ due to the equality $\Xi_n = 1$.
        Since the trace of the operator $T$ is equal to $-1$, the sufficiency is proved.

        \medskip
        Now assume that there exists a trace class operator $T$ with the trace equal to $-1$, which annihilates all the rank one operatos $f^*_n \otimes f_n$.
        Then all the operator matrix elements $T_{nn}, T_{n, n+1}, T_{n, n+2}$ belong to $\ell^1$ (simple property of a trace class operator).
        As we know, $\Xi_n$ tends to $1$, since the trace of the operator is $-1$.
        Look at the sum $S_{2j} = |T_{4j+1, 4j}| + |T_{4j+2,4j}| + |T_{4j+2,4j+1}|$.
        Observe that $S_{2j}$ is a linear function if considered as a function of $T_{4j+2, 4j}$.
        Its minimum is attained on the boundary of the domain, thus it is greater than $\mu_{2j}/2$ whenever $j$ is sufficiently large.
        The second pair of equalities gives out the minimum value greater than $\mu_{2j+1}/2$ when $j$ is large,
          which shows that the stated condition is necessary for the existence of the operator $T$.
        %\alex{TODO elaborate}
      \end{proof}
      Now consider the case of the $k$-dimensional operator $T$.
      Let us view the operator $T$ as a sum of $k$ rank one operators:
      \[
        T = \sum_1^k y^s \otimes x^s,
      \]
        where $x^s, y^s \in \cal{H}$.
      Let us define vectors $v_n$ and $u_n$ for $n \geq 0 $ which lie in $\mathbb{R}^k$ as follows:
      \begin{align*}
        v_{2j} &= (y^1_{4j}, y^2_{4j}, \dots ,y^k_{4j}) \quad
        &v^*_{2j} = (x^1_{4j}, x^2_{4j}, \dots ,x^k_{4j}) \\
        v_{2j+1} &= (x^1_{4j+2}, x^2_{4j+2}, \dots ,x^k_{4j+2}) \quad
        &v^*_{2j+1} = (y^1_{4j+2}, y^2_{4j+2}, \dots ,y^k_{4j+2}) \\
        u_{2j} &= (x^1_{4j+1}, x^2_{4j+1}, \dots ,x^k_{4j+1}) \quad
        &u^*_{2j} = (y^1_{4j+1}, y^2_{4j+1}, \dots ,y^k_{4j+1}) \\
        u_{2j+1} &= (y^1_{4j+3}, y^2_{4j+3}, \dots ,y^k_{4j+3}) \quad
        &u^*_{2j+1} = (x^1_{4j+3}, x^2_{4j+3}, \dots ,x^k_{4j+3}) 
      \end{align*}
      Note that the sequences $|v_n|$, $|u_n|$ belong to $\ell^2$.
      We can rewrite the previous equations using the introduced vectors.
      \begin{align*}
        \Xi_{4j} &= a_{2j} \langle u_{2j}, v_{2j}\rangle + c_{2j} \langle v_{2j+1}, v_{2j}\rangle,\\
        \Xi_{4j + 1} &= -d_{2j} \langle v_{2j+1}, v_{2j}\rangle + b_{2j} \langle v_{2j+1}, u_{2j}\rangle,\\
        \Xi_{4j + 2} &= a_{2j+1} \langle v_{2j+1}, u_{2j+1} \rangle + c_{2j+1} \langle v_{2j+1}, v_{2j+2} \rangle,\\
        \Xi_{4j + 3} &= -d_{2j+1} \langle v_{2j+1}, v_{2j+2}\rangle + b_{2j+1} \langle u_{2j+1}, v_{2j+2} \rangle.
      \end{align*}
      Here $\langle\cdot, \cdot\rangle$ denotes the scalar product in $\mathbb{R}^k$.
      %\begin{note}
        %Observe that in such setup we always have $\Xi_n = 0$ for any $n \leq 1$.
      %\end{note}
      Now we will study the simplified system instead.
      \begin{align*}
        \Xi_{2j} &= a_{j} \langle u_{j}, v_{j} \rangle  + c_{j} \langle v_{j+1}, v_{j} \rangle,\\
        \Xi_{2j + 1} &= -d_{j} \langle v_{j+1}, v_{j} \rangle + b_{j} \langle v_{j+1}, u_{j}\rangle.
      \end{align*}
      To summarize, we obtained that the existence of an operator $T$ could be reduced to the existence of vectors $u_n$, $v_n$ in the $\mathbb{R}^k$
        which satisfy the conditions given above.
      Given that the sequences of vectors $u_n$, $v_n$ lie in the $\mathbb{R}^k$, we might think of the scalar product as
        the product of the vector lengths and the cosinus of the angle between the vectors.
      \begin{prop}
        \label{k-dim-statement}
        If $\left\{\mu_n\right\}$ belongs to $\ell^1$ then it is possible to construct the vectors $u_n, v_n \in \mathbb{R}^2$ such that
          the equations above are true and for any $n \geq 0$ we have $\Xi_n = 1$.
      \end{prop}
      \begin{corol*}
        For any $k > 1$ there exists a $k$-dimensional operator $T$ with the trace equal to $-1$ such that $\langle Tf_n, f_n^*\rangle = 0$ for any $n \geq 0$
          if and only if the sequence $\left\{\mu_n\right\}$ belongs to $\ell^1$.
      \end{corol*}
      \begin{proof}
        %The operator matrix will look very similar to one we built in the previous proposition.
        Here again we are going to look at three possible values of the $\mu_n$.
        For each $n \geq 0$ we are going to find the vector lengths $V_n = |v_n|$, $U_n = |u_n|$ and three angles:
          $\alpha_n$ which stands for the angle between the vectors $v_n$ and $v_{n + 1}$,
          $\beta_n$ which denotes the angle between $v_n$ and $u_n$,
          and $\gamma_n$ standing for the angle between $v_{n + 1}$ and $u_n$.
        We will write out the vector lengths and define the angles and we will prove that the corresponding vectors could be settled in $\mathbb{R}^2$.
        %We choose the vector $v^*_0$ such that $\langle v_0, v^*_0 \rangle = -1$ and set all
          %the other vectors $u^*_n$, $v^*_n$ to $\vec{0}$.
        %That guarantees that the trace of the constructed operator (if it belongs to the trace class) is equal to $-1$.

        Within the construction we are going to set all $V_n$ step by step.
        In order to do that we are going to define an auxiliary sequence $\{M_n\}_{n=0}^\infty \in \ell^2$ such that $V_n \geq M_n$ for any $n \geq 0$.
        On each step $n$ we are going to define $V_n$ and $M_{n+1}$.
        We start by setting $M_0 = V_0 = 1$.

        \noindent\textbf{Case 1.} Suppose $\mu_n = 1/|a_n| + 1/|b_n|$.

          Here we want to have $v_n$ orthogonal to $v_{n+1}$.
          We set:
          \begin{align*}
              V_n &= \max\left(M_n, \frac{1}{\sqrt{\smash[b]{|a_n|}}}\right),\\
              U_n &= \sqrt{\frac{1}{a_n^2 V_n^2} + \frac{1}{b_n^2 V_{n+1}^2}},\\
              M_{n+1} &= \frac{1}{\sqrt{\smash[b]{|b_n|}}}.
          \end{align*}
          \begin{prop}
            The following inequalities are true.
            \begin{align*}
              M_{n+1} &\leq \sqrt{\mu_n},\\
              U_n &\leq \sqrt{\mu_n},\\
              V_n &\leq \max(\sqrt{\mu_n}, M_n).
            \end{align*}
            There exist such angles $\beta_n$, $\gamma_n$ that with the values $U_n$, $V_n$ defined like this the following is true:
            \begin{align*}
              \langle u_n, v_n \rangle &= 1/a_n,\\
              \langle u_n, v_{n+1} \rangle &= 1/b_n,\\
              \langle v_n, v_{n+1} \rangle &= 0.
            \end{align*}
          \end{prop}
          \begin{proof}
            First part of the proposition is trivial.
            Now we want to understand why there exist such $\beta_n$ and $\gamma_n$ that all the scalar products of
            $v_n$, $v_{n+1}$, $u_n$ satisfy the conditions above.
            Now observe that the chosen $U_n$, $1/(|a_n| V_n)$ and $1/(|b_n| V_{n+1})$ 
              comprise a right triangle with $U_n$ as a hypotenuse.
            As a result there always be such $\beta_n$ and $\gamma_n$ that
            \begin{align*}
              |U_n \cos{\beta_n}| &= \frac{1}{|a_n|V_n},\\
              \left|U_n \cos{\gamma_n}\right| &= \left|U_n \sin{\beta_n}\right| = \frac{1}{|b_n|V_{n+1}}.
            \end{align*}
          \end{proof}
        \noindent\textbf{Case 2.} Assume $\mu_n = (1 + |a_n|)/|c_n|$.

          Here we will have $u_n$ orthogonal to $v_n$.
          Assign:
          \begin{align*}
            M_{n+1} &= \max\biggl(\smash[b]{\frac{\sqrt{|a_n|}}{\sqrt{|c_n|}}}, \frac{1}{\sqrt{\smash[b]{|c_n|}}}\biggr),\\
            V_n &= \max\left(M_n, \frac{2}{\sqrt{\smash[b]{|c_n|}}}\right),\\
            \alpha_n &= \arccos{\frac{1}{c_n V_n V_{n+1}}},\\
            \gamma_n &= \frac{\pi}{2} \pm \alpha_n,\\
            U_n &= \frac{a_n}{c_n \cos{\gamma_n} V_{n+1}}.
          \end{align*}
          \begin{remark*}
            We choose plus or minus in the expression of $\gamma_n$ in order to make the $a_n/(c_n \cos{\gamma_n})$ always positive.
          \end{remark*}
          \begin{prop}
            The angle $\alpha_n$ is defined correctly and the following inequalities are true:
            \begin{align*}
                M_{n+1} &\leq \sqrt{\mu_n},\\
                V_n &\leq \max(2\sqrt{\mu_n}, M_n),\\
                U_n &\leq \sqrt{\mu_n}.
            \end{align*}
            With the values $U_n$, $V_n$, $\gamma_n$, $\alpha_n$ defined like we have
            \begin{align*}
              \langle u_n, v_n \rangle &= 0,\\
              \langle u_n, v_{n+1} \rangle &= a_n/c_n,\\
              \langle v_n, v_{n+1} \rangle &= 1/c_n.
            \end{align*}
          \end{prop}
          \begin{proof}
            Firstly, $\alpha_n$ is defined correctly since
            \[
              V_n V_{n+1} \geq V_n M_{n+1} \geq \frac{2}{\sqrt{\smash[b]{|c_n|}}} \frac{1}{\sqrt{\smash[b]{|c_n|}}} = \frac{2}{|c_n|}
            \]
            which means that the absolute value of the arccosinus argument is always less than $1/2$.
            Now taking into account that $|\cos{\alpha_n}| \leq 1/2$, notice that $|\cos{\gamma_n}| = |\sin{\alpha_n}|$ is always greater than $1/2$.
            It is the fact which we need later.
            
            Trivially, $M_{n+1} \leq \sqrt{\mu_n}$ and $V_n \leq \max(2\sqrt{\mu_n}, M_n)$.
            Next we set
            \[
              U_n = |U_n| \leq 2 \frac{|a_n|}{|c_n|} \frac{1}{V_{n+1}} \leq 2 \frac{\sqrt{|a_n|}}{\sqrt{|c_n|}} \leq \sqrt{\mu_n}.
            \]
            It is easy to notice that the angles and $U_n$ are chosen in such way that the scalar product conditions are 
              satisfied.
            Finally, we are able to lay out the vectors $v_n$, $u_n$ and $v_{n+1}$ with the prescribed angles 
              since we have $\beta_n = \pi/2 = (\pi/2 \pm \alpha_n) \mp \alpha_n = \gamma_n \mp \alpha_n$,
              where we choose the '$\mp$' sign accordingly to our choice in the expression of the $\gamma_n$ angle.
          \end{proof}
        \noindent\textbf{Case 3.} Suppose $\mu_n = (1 + |b_n|)/|d_n|$.

          This case is almost identical to the previous one.
          Here we will have $u_n$ orthogonal to $v_{n+1}$.
          Assign:
          \begin{align*}
            M_{n+1} &= \max\biggl(\smash[t]{\frac{\sqrt{|b_n|}}{\sqrt{|d_n|}}}, \frac{1}{\sqrt{\smash[b]{|d_n|}}}\biggr),\\
            V_n &= \max\left(M_n, \frac{2}{\sqrt{\smash[b]{|d_n|}}}\right),\\
            \alpha_n &= \arccos{\frac{1}{-d_n V_n V_{n+1}}},\\
            \beta_n &= \frac{\pi}{2} \pm \alpha_n,\\
            U_n &= \frac{b_n}{d_n \cos{\beta_n} V_n}.
          \end{align*}
          \begin{remark*}
            We choose plus or minus in the expression of $\beta_n$ in order to make the $b_n/(d_n \cos{\beta_n})$ always positive.
          \end{remark*}
          \begin{prop}
              The angle $\alpha_n$ is defined correctly and the following inequalities are true:
              \begin{align*}
                M_{n+1} &\leq \sqrt{\mu_n},\\
                V_n &\leq \max(2\sqrt{\mu_n}, M_n),\\
                U_n &\leq \sqrt{\mu_n}.
              \end{align*}
              With the values $U_n$, $V_n$, $\beta_n$, $\alpha_n$ defined like this the following is true:
              \begin{align*}
                \langle u_n, v_n \rangle &= b_n/d_n,\\
                \langle u_n, v_{n+1} \rangle &= 0,\\
                \langle v_n, v_{n+1} \rangle &= -1/d_n.
              \end{align*}
          \end{prop}
          \begin{proof}
            Firstly, $\alpha_n$ is defined correctly since
            \[
              V_n V_{n+1} \geq V_n M_{n+1} \geq \frac{2}{\sqrt{\smash[b]{|d_n|}}} \frac{1}{\sqrt{\smash[b]{|d_n|}}} = \frac{2}{|d_n|},
            \]
              which means that the absolute value of the arccosinus argument is always less than $1/2$. Now taking into
              account that $|\cos{\alpha_n}|$ is less than or equal to $1/2$, we always have $|\cos{\beta_n}| > 1/2$.
            
            Trivially, we have $M_{n+1} \leq \sqrt{\mu_n}$ and $V_n \leq \max(2\sqrt{\mu_n}, M_n)$.
            Next
            \[
              U_n = |U_n| \leq 2 \frac{|b_n|}{|d_n|} \frac{1}{V_n} \leq 2 \frac{\sqrt{|b_n|}}{\sqrt{|d_n|}} \leq \sqrt{\mu_n}.
            \]
            It is easy to notice that the angles and $U_n$ are chosen in such way that the scalar product conditions are satisfied.
            Finally, we are able to lay out the vectors $v_n$, $u_n$ and $v_{n+1}$ with the prescribed angles 
              since we have $\gamma_n = \pi/2 = (\pi/2 \pm \alpha_n) \mp \alpha_n = \beta_n \mp \alpha_n$,
              where we choose the '$\mp$' sign accordingly to our choice in the expression of the $\beta_n$ angle.
          \end{proof}
        In each of three cases we guaranteed that $M_{n+1} \leq \sqrt{\mu_n}$ and
          that $V_n \leq \max(M_n, 2\sqrt{\mu_n})$.
        Hence, we get that $V_n$ is bounded up to some constant by $\max(\sqrt{\mu_{n-1}}, \sqrt{\mu_n})$.
        Due to the three propositions above, for any $n \geq 0$ we have $U_n \leq \sqrt{\mu_n}$.
        Thus the constructed sequences $V_n$ and $U_n$ belong to $\ell^2$.

        In the end we choose the vector $v_0^*$ such that $\langle v_0, v_0^*\rangle$ is equal to $-1$.
        It is possible since the vector $v_0$ is not trivial.
        Furthermore, we set all the other $v^*_n$ and $u^*_n$ to the zero vector in order to guarantee that for any $n \geq 0$ we have $\Xi_n = 1$.
        The trace of the constructed operator is obviously equal to $-1$.
      \end{proof}
      The proof of the theorem is finished with the proof of the statements~\ref{inf-dim-statement} and~\ref{k-dim-statement}.
    \end{proof}
