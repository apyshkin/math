\documentclass[12pt]{article}
\usepackage{cmap} % тоже для кодировки
%\usepackage[cp866]{inputenc}
\usepackage[T2A]{fontenc}
\usepackage[utf8]{inputenc} % любая желаемая кодировка
\usepackage[english]{babel}
\usepackage[pdftex,unicode]{hyperref}
\usepackage{amsmath}
\usepackage{amssymb}
\usepackage{amsthm}
\usepackage{amsfonts}
\usepackage{graphicx}
\usepackage[normalem]{ulem}
\usepackage{extsizes}
\usepackage{float}
\usepackage{bbold}
\usepackage{dsfont}
\usepackage{calc}
%\usepackage [a4paper,% other options: a3paper, a5paper, etc
%  left=3cm,
%  right=1.5cm,
%  top=2cm,
%  bottom=2cm,
%]{geometry}

\usepackage{tocloft}
%\renewcommand{\cfttoctitlefont}{\hspace{0.38\textwidth} \bfseries}
%\renewcommand{\cftbeforetoctitleskip}{-1em}
%\renewcommand{\cftaftertoctitle}{\mbox{}\hfill \\ \mbox{}\hfill{\footnotesize Стр.}\vspace{-.5em}}
%\providecommand{\cftchapfont}{\normalsize\bfseries \sectionname}
%\renewcommand{\cftsecfont}{\hspace{31pt}}
%\renewcommand{\cftsubsecfont}{\hspace{11pt}}
%\providecommand{\cftbeforechapskip}{1em}
%\renewcommand{\cftparskip}{-1mm}
%\renewcommand{\cftdotsep}{1}
%\setcounter{tocdepth}{2} % задать глубину оглавления - до subsection включительно

\usepackage{titlesec}
\sloppy
%\titleformat{\section}
%{\normalsize\bfseries}
%{\thesection}
%{1em}{}

%\titleformat{\subsection}
%{\normalsize\bfseries}
%{\thesubsection}
%{1em}{}

% Настройка вертикальных и горизонтальных отступов
%\titlespacing*{\chapter}{0pt}{-30pt}{8pt}
%\titlespacing*{\section}{\parindent}{*4}{*4}
%
%\linespread{1.3}

\newlength{\widecommentlength}
\setlength{\widecommentlength}{5in}
% \newcommand{\widecommentbox}[2]{\def#1##1{\strut\newline\noindent\colorbox{#2}{\linespread{1}\parbox{.95\textwidth}{\small ##1}}\newline}}
% \usepackage{pgfplots}
\newcommand{\widecommentbox}[3]{\def#1##1{\strut\newline\noindent\colorbox{#3}{\linespread{1}\parbox{.95\textwidth}{\small {\bf [#2]} ##1}}\newline}}
\def\commentsep{\noindent\dotfill}

% To temporarily omit all comments, enable these two lines:
% \renewcommand{\widecommentbox}[3]{\def#1##1{}}
% \let\commentsep\relax

\widecommentbox{\alex}{Alex}{green!20!white}
\widecommentbox{\ad}{AD}{red!20!white}

\usepackage{enumitem}
\usepackage{setspace}

\makeatletter
%\@addtoreset{theorem}{section}
%\@addtoreset{lemma}{section}
\@addtoreset{prop}{section}
\makeatother

%\newcommand{\sectionbreak}{\clearpage}
\usepackage[square,numbers,sort&compress]{natbib}
\usepackage{mathtools}
\renewcommand{\bibnumfmt}[1]{#1.\hfill} % нумерация источников в самом списке - через точку
% \renewcommand{\bibsection}{\section*{Список литературы}} % заголовок специального раздела
\setlength{\bibsep}{0pt}
\newcommand*{\Scale}[2][4]{\scalebox{#1}{\ensuremath{#2}}}%

%\titleformat{\section}[block]{\Large\bfseries\centering}{}{1em}{}
%\titleformat{\subsection}[block]{\large\bfseries\centering}{}{1em}{}
\renewcommand{\cal}[1]{\mathcal{#1}}
\renewcommand{\leq}{\leqslant}
\renewcommand{\geq}{\geqslant}
\renewcommand{\phi}{\varphi}
\newtheorem{theorem}{Theorem}
\newtheorem{prop}{Proposition}
\newtheorem{prop_under_lemma}{Утверждение}
\newtheorem{lemma}{Lemma}
\theoremstyle{definition}
\newtheorem{corol}{Corollary}
\newtheorem{remark}{Remark}
\newtheorem*{note}{Примечание}
\newtheorem{definition}{Definition}
\newtheorem{example}{Пример}
\newcommand\bigmatrixzero{\raisebox{-0.25\height}{\textnormal{\Huge 0}}}
\newcommand\bigzero{\makebox(10, 10){\text{\Huge 0}}}
\newcommand{\system}[1]{\{{#1}_k\}_{k=1}^\infty}

\numberwithin{remark}{section}
\numberwithin{theorem}{section}
\numberwithin{prop}{section}
\numberwithin{equation}{section}
\numberwithin{lemma}{section}
\numberwithin{prop_under_lemma}{lemma}

\begin{document}
% \title{Методы суммирования ряда Фурье \\
% относительно системы Азоффа--Шехада}

% \author{Алексей Пышкин\thanks{Работа поддержана грантом Президента РФ для государственной 
% поддержки молодых российских учёных -- докторов наук МД-5758.2015.1.}}
% \date{}
  \section*{Classification theorem for the AS system}
    We want to consider systems like the one regarded by Azoff and Shehada~\cite{azoff} and to determine the exact conditions for the $k$-completeness property
    of such vector systems (a system is called $k$-complete whenever the corresponding partial fourier sums are able to approximate an identity operator in at least $k$ points). Up until now we have seen two very different techniques for the 
    $1$-completeness analysis (strong $M$-basisness or hereditary completeness) and for the $k$-completeness analysis with $k>2$. In this section we are going to elaborate a single method for the analysis of the $k$-completeness property.
    
    Here we use the standard setup. Given the system $f_n$ we define the $L$ as the lattice generated by the one-dimensional spaces $[f_n]$. Also we denote
    the algebra of operators which leave all of the $[f_n]$ invariant with $\cal{A}$:
    $$\cal{A} = \operatorname{Alg}(L).$$
    Also let $\cal{R}_1(\cal{A})$ be the rank one operators in the algebra $\cal{A}$.
    Here we assume that the hilbert space $H$ is a real hilbert space.
    
    Let us start with the canonical example of the Azoff--Shehada~\cite{azoff} vector system. The system is given as follows:
    \begin{equation}
        \label{as-system}
        \begin{aligned}
          &f_1 = e_1 + a_2 e_2, \qquad &f_{2j}=e_{2j}, \quad&j \geq 1,&\\
          &f_{2j-1}=-a_{2j-1}e_{2j-2} + e_{2j-1} + a_{2j}e_{2j}, \qquad &\makebox[5em]{} \quad&j \geq 2,&\\
          &f^*_{2j}=-a_{2j}e_{2j-1}+e_{2j}+a_{2j+1}e_{2j+1}, \qquad &f^*_{2j-1}=e_{2j-1}, \quad&j \geq 1&
        \end{aligned}
    \end{equation}
    for some real $a_n > 0$, where $n > 1$.
    
    \begin{theorem}
        \label{thm_as}
        Firstly, the sequences $\{f_j\}$, $\{f^*_j\}$ are biorthogonal and both are complete in $\cal{H}$ (thus it is an $M$-basis).
        Secondly, the following statements are true:
        \begin{itemize}
            \item  The system~\eqref{as-system} is NOT $1$-complete (a strong $M$-basis) iff the following sequence
                \begin{equation}
                    \mu_n = \frac{a_{n-1} a_{n-3} \dots}{a_{n} a_{n-2} \dots }\\
                \end{equation}
                belongs to $l^2$.
            \item For any $k>1$: the system is NOT $k$-complete iff the sequence $1/a_n$ belongs to $l^1$.
        \end{itemize}
    \end{theorem}
    \begin{remark}
       Note that any such system carries the strong approximation property (which is equivalent to the
       ultraweak density of rank one elements of the corresponding operator algebra for the sequence) 
       if and only if the system holds the $k$-completeness property for any $k>0$.
    \end{remark}
    \begin{proof}[Proof of the theorem]
        First we establish the basic properties of the system:
        \begin{prop}
            The system is $k$-complete whenever each $k$-dimensional operator $T$ 
            such that $\langle Tf_n, f_n^* \rangle = 0$ for any $n$ has a zero trace.
        \end{prop}
        \begin{proof}
            In the paper~\cite{katavolos} authors prove the proposition for $k = 2$. For greater $k$s the same reasoning will work.
        \end{proof}
        Let us consider a $k$-dimensional operator $T$ such that 
        $Tr(TR) = 0$ for each $R \in \cal{R}_1(\cal{A})$ which essentially means that
        $\langle Tf_n, f_n^* \rangle = 0$ for any $n$. 
        We would like to find the necessary and sufficient conditions of the existence of the operator $T$ with a non-zero trace.
        Let us for now discover some common properties for an arbitrary $k$-dimensional operator $T$ which belongs to the annihilator of the rank one subalgebra of $\cal{A}$.\\
        Notice that the partial sums of the fourier series for the given system are somehow close to the
        partial sums of the canonical fourier series (using the orthonormal basis $e_k$). Define
        $$
          \Xi_n := \sum_1^n \langle Tf_s, f_s^* \rangle - \sum_1^n \langle Te_s, e_s \rangle = -\sum_1^n \langle Te_s, e_s \rangle.
        $$
        where the $\langle \cdot, \cdot\rangle$ denotes a standard scalar product in $\cal{H}$. 
        These residuals have also a concise form:
        \begin{align*}
            \Xi_{2n-1} &= a_{2n}T_{2n - 1, 2n},\\
            \Xi_{2n} &= a_{2n + 1}T_{2n + 1, 2n},
        \end{align*}
        having $T_{ij}$ equal to the $\langle Te_j, e_i \rangle$.
        \begin{remark}
            If such operator $T$ exists and has a non-zero trace, then the sequence $1/a_k$ is summable.
        \end{remark}
        \begin{proof}
            Suppose trace is equal to $-1$ without loss of generality.
            Then the residuals $\Xi_k$ tend to $1$ with $k$ going to infinity. Since $T_{k-1, k}$ as well as $T_{k, k-1}$
            is a summable sequence for any trace operator $T$, we get the required condition.
        \end{proof}
        Let us put the operator $T$ as a finite sum:
        $$
            T = \sum_1^k y^s \otimes x^s,
        $$
        where $x^s, y^s \in \cal{H}$.
        Therefore $T_{ij} = \sum_{s=1}^k {y^s_j x^s_i}$. We can rewrite the previous equations:
        \begin{align*}
            \Xi_{2n-1} = a_{2n} \sum_1^k y^s_{2n} x^s_{2n - 1},\\
            \Xi_{2n} = a_{2n + 1} \sum_1^k y^s_{2n} x^s_{2n + 1},
        \end{align*}
        Now we are going to do a simple trick which reveals the essence of the problem.
        Let us define vectors $v_n$ and $u_n$ which lie within $\mathbb{R}^k$ as follows:
        \begin{align*}
            v_n = (x^1_n, x^2_n, \dots ,x^k_n),\\
            u_n = (y^1_n, y^2_n, \dots ,y^k_n). 
        \end{align*}
        Using these definitions we are able to observe that:
        \begin{align}
            \label{vector-eq}
            \Xi_{2n-1} = a_{2n} \langle u_{2n}, v_{2n - 1}\rangle,\\
            \Xi_{2n} = a_{2n + 1} \langle u_{2n}, v_{2n + 1}\rangle.
        \end{align}
        Here the $\langle\cdot, \cdot\rangle$ denotes a standard scalar product in the $\mathbb{R}^k$.
        We want to point out that the existence of such operator $T$
        might be reduced to the existence of vectors $u_n$, $v_n$ in the $\mathbb{R}^k$ (such that $|u_n|$, $|v_n|$ are both square summable) which respects the condition given above.
        Also the condition of trace of $T$ being non-zero reduces to the simple restriction on the vectors $u_n$ and $v_n$:
        \begin{prop}
            There exists a $k$-dimensional operator $T$ such that $\langle Tf_n, f_n^*\rangle  = 0$ for any $n$
            which has a trace equal to $-1$ if and only if there exist such vectors $u_n$, $v_n$ belonging to the
            $\mathbb{R}^k$ that $|u_n|$, $|v_n|$ are square summable, the equations~\eqref{vector-eq} are satisfied and
            $\sum \langle u_n,v_n \rangle = -1$. 
        \end{prop}
        \begin{proof}
            Only the trace condition is left to transform.
            Trace of the operator $T$ is equal to
            \begin{multline*}
                Tr(T) = \sum_{s=1}^k \langle y^s, x^s \rangle = \sum_s \sum_{n=1}^\infty y^s_n x^s_n =\\
                      = \sum_n \sum_s y^s_n x^s_n = \sum_n \langle u_n, v_n \rangle.
            \end{multline*}
        \end{proof}
        Just for convenience we might take a new sequence of vectors $w_n$ which incorporates both $u_n$ and $v_n$
        \begin{align*}
            w_{2n} = u_{2n}\\
            w_{2n + 1} = v_{2n + 1}.
        \end{align*}
        Now we are going to examine the vectors $w_n$ that $|w_n|$ is in $l^2$ and
        \begin{align*}
            a_{n} \langle w_{n}, w_{n - 1}\rangle = \Xi_{n - 1}.
        \end{align*}
        Given that the vectors $w_n$ lie in the $\mathbb{R}^k$ we might think of the scalar product as
        the usual product of the vector lengths and the cosinus of the angle between the vectors.
        Namely the existence of $w_n$ is the existence such $W_n := |w_n|$ and real $\theta_n$ that
        $\langle w_{n}, w_{n+1}\rangle = W_n W_{n+1} \cos{\theta_n}.$
        In other words we need to find such $a'_n = a_n \cos{\theta_n}$ that
        the sequence
        $$
          W_n = \frac{\Xi_{n-1}/a'_n}{\Xi_{n-2}/a'_{n-1}} \cdot \frac{\Xi_{n-3}/a'_{n-2}}{\Xi_{n-4}/a'_{n-3}} \cdots
        $$
        belongs to $l^2$. For our conveniency we assume that neither $\Xi_k$ nor $a_k$ are equal to zero (it requires some kind of truncating though it does not alter the proof significantly).\\
        Now since $\Xi_n = -\sum_1^n \langle Te_s, e_s\rangle$ we can see that
        $$
            \frac{\Xi_n}{\Xi_{n-1}} = 1 + \eta_n,
        $$
        where $\eta_n$ is a summable sequence. Thus the product of such $(1 + \eta_s)$ fractions is bounded by 
        some constant above. That fact allows us to drop all the $\Xi_n$ from the expression above which is important in the
        case of $k = 1$.
        
        Now we are ready to finish the proof.
        First we prove the theorem for the case $k=1$. Lets start with the proposition:
        \begin{prop}
            There exists a sequence of real numbers $w_n$ which satisfies the equation $a_n w_n w_{n-1} = 1$ if and
            only if the sequence for the statement of the theorem $\mu_n$ belongs to $l^2$.
        \end{prop}
        \begin{proof}
            The proof of the proposition is trivial since $w_n$ would be equal to $\mu_n$ up to a constant.
        \end{proof}
        The formula for the multidimensional case reduces to the expression of $\mu_n$ whenever we
        set all the angles to zero (thus $a'_n$ is equal to $a_n$). Due to the idea of setting all the $\Xi_k$ to one,
        the existence of the one-dimensional operator $T$ is equivalent to the existence of the sequence $w_n$.
        Therefore, when we have such one-dimensional operator $T$, we have proven that $\mu_n$ must belong to $l^2$.\\
        
        Now, lets prove the sufficiency of the first statement of the theorem.
        The trace condition on $T$ in the one-dimensional case is 
        equivalent to having $\sum \langle u_n, v_n\rangle $ to be equal to $-1$ as we have already noted.
        Whenever $\mu_n$ belongs to $l^2$, we might take $u_1$ equal to $-1/w_1$ and $w_n$ equal to $\mu_n$.
        All the $u_{2n+1}$ and $v_{2n}$ for all $n>0$ are to be set to zero.
        Such $w_n$ defines an operator with trace equal to $-1$, exactly the one we were looking for.
        \begin{remark}
            Here the necessity is discussed so shortly due to the arguments discussed previously: most importantly we
            were able to replace the residuals $\Xi_k$ with $1$.
        \end{remark}
        \noindent And now we consider the higher dimensions. Lets start our proof of the $k > 1$ case with the 
        \begin{prop}
            There exists a sequence of vectors $w_n$ in the $\mathbb{R}^k$ for any $k > 1$ which satisfy $a_n \langle w_n, w_{n-1} \rangle = 1$ if and only if $1/a_n$ is a summable sequence.
        \end{prop}
        \begin{proof}
            Though we made a remark on the necessity of the summability
            in the beginning of our proof of the theorem it is still
            easy to see that the existence of a sequence ensures that the $1/a'_n$ is summable and hence
            the $1/a_n$ is summable as well.
            
            Now we may look at the sufficiency of the condition.
            We have two possible choices to argue here. Both are elementary and lead to the same construction but we would
            like to present several ways of looking at the problem.\\
            \textbf{Reasoning 1.}\\
            Suppose $\sum 1/a_n$ is a summable series. Let us define $W_n \in l^2$ as:
            $$
            W_n := \max(a_n^{-\frac{1}{2}}, a^{-\frac{1}{2}}_{n+1}).
            $$
            Then the product $W_nW_{n-1}$ will always be bigger than $1/\sqrt{a_n}$ thus giving us a possibility to
            find the angles $\theta_n$ so that 
            $$
            a_n \langle w_n, w_{n-1} \rangle = a_n W_n W_{n-1}\cos{\theta_n} = 1.
            $$
            The construction of the vectors $w_n$ is then done by induction using the lengths $W_n$ and the chosen
            angles $\theta_n$. Obviously any $k$-dimensional eucledian space suits us for $k > 1$.\\
            \textbf{Reasoning 2.}\\
            Suppose $1/a_n$ is a summable sequence. We need to find such $a'_n$ that $|a'_n| \leq |a_n|$ and
            $\mu'_n = \frac{a'_{n-1} a'_{n-3} \dots}{a'_{n} a'_{n-2} \dots }$ belongs to $l^2$. Note that essentially we need to reduce $a_n$ in such a way that the fractions of the $\mu_n$ type belong to $l^2$ (sic!). Suprisingly there is an appropriate sequence $a'_n$ which has a very explicit and compact form:
            \begin{equation*}
                a'_n := \begin{cases}
                    \sqrt{a_n a_{n-1}} & \quad a_{n-1} \leq a_n \leq a_{n+1},\\
                    \sqrt{a_n a_{n+1}} & \quad a_{n-1} \geq a_n \geq a_{n+1},\\
                    a_n & \quad a_{n-1} \geq a_n \leq a_{n+1},\\
                    \sqrt{a_{n-1} a_{n+1}} &\quad a_{n-1} \leq a_n \geq a_{n+1}.\\
                \end{cases}
            \end{equation*}
            It is easy to check that for such a sequence $a'_n$ the generated sequence $\mu'_n$ is equal either to $1/\sqrt{a_n}$ or to the $1/\sqrt{a_{n+1}}$. Then obviously the constructed $\mu'_n$ belongs to $l^2$.
            From this construction we are able to calculate the precise lengths $W_n$ and angles $\theta_n$.
            Given lengths and angles the whole sequence of vectors $w_n$ is constructed using an induction.
            The dimension $k$ of the enclosing euclidean space is not important here -- the algorithm works 
            for any $k$ greater than one.
        \end{proof}
        \begin{remark}
            Note that the trick with the $\Xi_k$ removal is not needed if we are proving the second statement of the theorem.
            It is important only for the one-dimensional case (the necessity implication).
            For the multidimensional case the necessity of
            the sequence to be summable is rather easy to prove, and the construction of the operator $T$ is
            performed along with all $\Xi_k$ being equal to $1$.
        \end{remark}
        Essentially the last proposition proves the second statement of the theorem.
        Observe that during the multidimensional construction of the operator $T$ again
        we are able to choose the first vector $u_1$ so that $\langle u_1,v_1 \rangle = -1$,
        and all the other vectors $u_{2n+1}$ and $v_{2n}$ for all $n>0$ are equal to zero (exactly how we did it in the 
        one-dimensional case).
    \end{proof}
    \begin{remark}
        Note that the cases $k=1$ and $k > 1$ now could be seen as of a great difference since we do not have angles
        on a real line.
    \end{remark}
    % \begin{remark}
    %     Right here we can deduce the necessary and sufficient condition for the $k$-completeness property (for $k>1$) having the condition
    %     for the strong approximation property discovered earlier.
    %     As we remember $f_n$ approximates strongly iff the $a_n^{-1}$ is summable. Also we know as a fact that
    %     a system approximates strongly iff for any $k$ it is $k$-complete. Taking the last proposition into account we
    %     understand that the $k$-completeness condition is the same for all $k > 1$ and is identical to the strong approximation condition: $a_n^{-1}$ must be summable.
    % \end{remark}

\medskip
% E-mail: aapyshkin@gmail.com
\bigskip
\begin {thebibliography}{20}
    \bibitem{azoff}
    E.~\!Azoff, H.~\!Shehada,
    \emph{Algebras generated by mutually orthogonal idempotent operators}.
    J. Oper. Theory, 29 (1993), 2, 249--267.
    \bibitem{bbb} 
    A. Baranov, Yu. Belov, A. Borichev,                                       
    \emph{Hereditary completeness for systems of exponentials and reproducing kernels},
    Adv. Math., 235 (2013), 1, 525--554.
    \bibitem{bbb1}
    A. Baranov, Yu. Belov, A. Borichev, 
    \emph{Spectral synthesis in de Branges spaces},
    Geom. Funct. Anal. (GAFA), 25 (2015), 2, 417--452.
    \bibitem{ad_preprint}
    A.D.~\!Baranov, D.V.~\!Yakubovich,
    \emph{Completeness and spectral synthesis of nonselfadjoint one-dimensional
    perturbations of selfadjoint operators}.
    arXiv:1212.5965 [math.FA]
    \bibitem{katavolos}
    A.~\!Katavolos, M.~\!Lambrou, M.~\!Papadakis,
    \emph{On some algebras diagonalized by $M$-bases of $\ell^2$}.
    Integr. Equat. Oper. Theory, 17 (1993), 1, 68--94.
    %\bibitem{wermer}
    %J.~\!Wermer,
    %\emph{On invariant subspaces of normal operators}.
    %Proc. Amer. Math. Soc., 3(1952), 2, 270--277.
    \bibitem{larson}
    D.~\!Larson, W.~\!Wogen,
    \emph{Reflexivity properties of $T\bigoplus0$}.
    J. Funct. Anal., 92 (1990), 448--467.
    %\bibitem{rotfeld}
    %В.В.~\!Пеллер,
    %\emph{Операторы Ганкеля и их приложения}.
    %Издательство РХД, Ижевск(2005).
    %N.K.~\!Nikol'skii,
    %\emph{Complete extensions of Volterra operators},
    %Izv. Akad. Nauk SSSR Ser. Mat 33(1969), 1349--1355. (Russian)

\end{thebibliography}
\vspace{1em}
\noindent{\bf Keywords:} complete minimal system, biorthogonal system, hereditary completeness, strong M-basis, summation method.

\end{document}