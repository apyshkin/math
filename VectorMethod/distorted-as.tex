\documentclass[12pt]{amsart}
\usepackage{cmap}
%\usepackage[cp866]{inputenc}
\usepackage[T2A]{fontenc}
\usepackage[utf8]{inputenc}
\usepackage[ngerman, english]{babel}
\usepackage[pdftex,unicode]{hyperref}
\usepackage{amsmath}
\usepackage{amssymb}
\usepackage{amsthm}
\usepackage{verbatim}
\usepackage{relsize}
\usepackage{amsfonts}
\usepackage{graphicx}
\usepackage[normalem]{ulem}
\usepackage{extsizes}
\usepackage{float}
\usepackage{bbold}
\usepackage{dsfont}
\usepackage{calc}
\usepackage{bm}
%\usepackage{tocloft}
%\renewcommand{\cfttoctitlefont}{\hspace{0.38\textwidth} \bfseries}
%\renewcommand{\cftbeforetoctitleskip}{-1em}
%\renewcommand{\cftaftertoctitle}{\mbox{}\hfill \\ \mbox{}\hfill{\footnotesize Стр.}\vspace{-.5em}}
%\providecommand{\cftchapfont}{\normalsize\bfseries \sectionname}
%\renewcommand{\cftsecfont}{\hspace{31pt}}
%\renewcommand{\cftsubsecfont}{\hspace{11pt}}
%\providecommand{\cftbeforechapskip}{1em}
%\renewcommand{\cftparskip}{-1mm}
%\renewcommand{\cftdotsep}{1}
%\setcounter{tocdepth}{2} % задать глубину оглавления - до subsection включительно

%\usepackage{titlesec}
%\sloppy
%\titleformat{\section}
%{\normalsize\bfseries}
%{\thesection}
%{1em}{}

%\titleformat{\subsection}
%{\normalsize\bfseries}
%{\thesubsection}
%{1em}{}

% Настройка вертикальных и горизонтальных отступов
%\titlespacing*{\chapter}{0pt}{-30pt}{8pt}
%\titlespacing*{\section}{\parindent}{*4}{*4}
%
%\linespread{1.3}

\newlength{\widecommentlength}
\setlength{\widecommentlength}{5in}
% \newcommand{\widecommentbox}[2]{\def#1##1{\strut\newline\noindent\colorbox{#2}{\linespread{1}\parbox{.95\textwidth}{\small ##1}}\newline}}
\usepackage{pgfplots}
\newcommand{\widecommentbox}[3]{\def#1##1{\strut\newline\noindent\colorbox{#3}{\linespread{1}\parbox{.95\textwidth}{\small {\bf [#2]} ##1}}\newline}}
\def\commentsep{\noindent\dotfill}

% To temporarily omit all comments, enable these two lines:
% \renewcommand{\widecommentbox}[3]{\def#1##1{}}
% \let\commentsep\relax

\widecommentbox{\alex}{AP}{green!20!white}
\widecommentbox{\ad}{AD}{red!20!white}


%\makeatletter
%\@addtoreset{theorem}{section}
%\@addtoreset{lemma}{section}
%\@addtoreset{prop}{section}
%\makeatother

\usepackage{enumitem}
%\usepackage{setspace}
%\newcommand{\sectionbreak}{\clearpage}
\usepackage[square,numbers,sort&compress]{natbib}
\usepackage{mathtools}
\renewcommand{\bibnumfmt}[1]{#1.\hfill} % нумерация источников в самом списке - через точку
% \renewcommand{\bibsection}{\section*{Список литературы}} % заголовок специального раздела
\setlength{\bibsep}{0pt}
\newcommand*{\Scale}[2][4]{\scalebox{#1}{\ensuremath{#2}}}%

%\titleformat{\section}[block]{\Large\bfseries\centering}{}{1em}{}
%\titleformat{\subsection}[block]{\large\bfseries\centering}{}{1em}{}
\newcommand{\cal}[1]{\mathcal{#1}}
\renewcommand{\leq}{\leqslant}
\renewcommand{\geq}{\geqslant}
\renewcommand{\phi}{\varphi}
\newtheorem{theorem}{Theorem}
\newtheorem*{theorem*}{Theorem}
\newtheorem{prop}{Proposition}
\newtheorem{lemma}{Lemma}
\newtheorem{corol}{Corollary}
\theoremstyle{definition}
\newtheorem{definition}{Definition}
\newtheorem*{definition*}{Definition}
\newtheorem{example}{Example}
\theoremstyle{remark}
\newtheorem{remark}{Remark}
\newtheorem*{remark*}{Remark}
\newtheorem*{note}{Note}
\newcommand\inner[2]{\langle #1, #2 \rangle}
\newcommand\bigmatrixzero{\raisebox{-0.25\height}{\textnormal{\Huge 0}}}
\newcommand\bigzero{\makebox(10, 10){\text{\Huge 0}}}
\newcommand{\seq}[1]{\{{#1}_n\}_{n=1}^\infty}
\newcommand{\fsys}{\mathfrak{F}}
\newcommand{\fstarsys}{\mathfrak{F^{*}}}
\newcommand{\wt}{\mathrm{\hat{w}}}
\newcommand{\wtp}{\mathrm{w}}
\newcommand{\len}{\mathfrak{L}}
\newcommand{\depth}{\operatorname{depth}}
\newcommand{\flow}{\mathcal{\hat{F}}}
\newcommand{\flowpos}{\mathcal{F}}
\newcommand{\preflow}{\mathcal{F^{*}}}
\newcommand{\flowposn}[1]{\mathcal{F}_{#1}}
\newcommand{\flown}{\cal{\hat{F}}_{n}}
\newcommand{\flowsgn}{\cal{\hat{F}}}
\newcommand{\source}{\mathbf{s}}
\newcommand{\sink}{\mathbf{t}}
\newcommand{\init}{init}
\newcommand{\ter}{ter}
\newcommand{\ein}{in}
\newcommand{\eout}{out}
\newcommand{\eback}{\mathbf{back}}
\newcommand{\efor}{\mathbf{forward}}
\renewcommand{\root}{\mathbf{r}}
\newcommand{\scal}[2]{\langle {#1}, {#2} \rangle}
\newcommand{\net}{\Delta}
\newcommand{\onet}{\vec{\Delta}}
\newcommand{\gpaths}{\cal{P}_{G}}
\newcommand{\gnpaths}{\cal{P}_n}
\newcommand{\gstar}{G^{*}}
\newcommand{\gfi}{\varphi_{G}}
\newcommand{\vspan}[1]{span\left(#1\right)}
\newcommand{\cspan}[1]{\overline{span}\left(#1\right)}
\newcommand{\phistar}{\phi_{*}}
\numberwithin{remark}{section}
\numberwithin{theorem}{section}
\numberwithin{prop}{section}
\numberwithin{equation}{section}
\numberwithin{lemma}{section}

\newtheoremstyle{case}{}{}{}{}{}{:}{ }{}
\theoremstyle{case}
\newtheorem{case}{Case}

\begin{document}
\title{On $k$-point density problem for finite diagonal $M$-bases}
\author{Alexey Pyshkin}
\maketitle

\section{Introduction}
TODO
\section{Preliminaries}
  We will discuss some properties of a particular system of vectors in a separable Hilbert space $\cal{H}$.
  We assume that $\cal{H}$ is a real Hilbert space. 
  Suppose that $\cal{H}$ has an orthonormal basis $\{e_j\}_{j=0}^\infty$.

\section{Classification theorem for the AS system}
  We want to consider systems like the one regarded by Azoff and Shehada~\cite{azoff} and to determine the exact conditions 
    for the $k$-completeness property of such vector systems
    (a system is called $k$-complete whenever the corresponding partial fourier sums are able to approximate an identity operator in at least $k$ points).
  Up until now we have seen two very different techniques for the $1$-completeness analysis (strong $M$-basisness or hereditary completeness) and
    for the $k$-completeness analysis with $k>2$. In this section we are going to elaborate a single method for the analysis of the $k$-completeness property.
  Here we use the standard setup. Given the system $f_n$ we define the $L$ as the lattice generated by the one-dimensional spaces $[f_n]$.
  Also we denote the algebra of operators which leave all of the $[f_n]$ invariant with $\cal{A}$: $$\cal{A} = \operatorname{Alg}(L).$$
  Also let $\cal{R}_1(\cal{A})$ be the rank one operators in the algebra $\cal{A}$. 
  Here we assume that the hilbert space $H$ is a real hilbert space.
  Let us start with the canonical example of the Azoff--Shehada~\cite{azoff} vector system.
  The system is given as follows: 
  \begin{equation} 
    \label{as-system}
      \begin{aligned}
        &f_1 = e_1 + a_2 e_2, \qquad &f_{2j}=e_{2j}, \quad&j \geq 1,&\\
        &f_{2j-1}=-a_{2j-1}e_{2j-2} + e_{2j-1} + a_{2j}e_{2j}, \qquad &\makebox[5em]{} \quad&j \geq 2,&\\
        &f^*_{2j}=-a_{2j}e_{2j-1}+e_{2j}+a_{2j+1}e_{2j+1}, \qquad &f^*_{2j-1}=e_{2j-1}, \quad&j \geq 1&
      \end{aligned}
  \end{equation}
  for some real $a_n > 0$, where $n > 1$.
  
  \begin{theorem}
    \label{thm_as}
    Firstly, the sequences $\{f_j\}$, $\{f^*_j\}$ are biorthogonal and both are complete in $\cal{H}$ (thus it is an $M$-basis).
    Secondly, the following statements are true:
    \begin{itemize}
      \item  The system~\eqref{as-system} is NOT $1$-complete (a strong $M$-basis) iff the following sequence
        \begin{equation}
          \mu_n = \frac{a_{n-1} a_{n-3} \dots}{a_{n} a_{n-2} \dots }\\
        \end{equation}
        belongs to $\ell^2$.
      \item For any $k>1$: the system is NOT $k$-complete iff the sequence $1/a_n$ belongs to $l^1$.
    \end{itemize}
  \end{theorem}
  \begin{remark}
    Note that any such system carries the strong approximation property (which is equivalent to the
    ultraweak density of rank one elements of the corresponding operator algebra for the sequence) 
    if and only if the system holds the $k$-completeness property for any $k>0$.
  \end{remark}
  \begin{proof}[Proof of the theorem]
    First we establish the basic properties of the system:
    \begin{prop}
      The system is $k$-complete whenever each $k$-dimensional operator $T$ 
      such that $\langle Tf_n, f_n^* \rangle = 0$ for any $n$ has a zero trace.
    \end{prop}
    \begin{proof}
      In the paper~\cite{katavolos} authors prove the proposition for $k = 2$. For greater $k$s the same reasoning will work.
    \end{proof}
    Let us consider a $k$-dimensional operator $T$ such that $Tr(TR) = 0$ for each $R \in \cal{R}_1(\cal{A})$ which essentially means that
      $\langle Tf_n, f_n^* \rangle = 0$ for any $n$. 
    We would like to find the necessary and sufficient conditions of the existence of the operator $T$ with a non-zero trace.
    Let us for now discover some common properties for an arbitrary $k$-dimensional operator $T$ which belongs to the annihilator of the rank one subalgebra of $\cal{A}$.\\
    Notice that the partial sums of the fourier series for the given system are somehow close to the
      partial sums of the canonical fourier series (using the orthonormal basis $e_k$). Define
    \[
      \Xi_n = \sum_1^n \langle Tf_s, f_s^* \rangle - \sum_1^n \langle Te_s, e_s \rangle = -\sum_1^n \langle Te_s, e_s \rangle,
    \]
      where the $\langle \cdot, \cdot\rangle$ denotes a standard scalar product in $\cal{H}$. 
    These residuals have also a concise form:
    \begin{align*}
      \Xi_{2n-1} &= a_{2n}T_{2n - 1, 2n},\\
      \Xi_{2n} &= a_{2n + 1}T_{2n + 1, 2n},
    \end{align*}
      having $T_{ij}$ equal to the $\langle Te_j, e_i \rangle$.
    \begin{remark}
      If such operator $T$ exists and has a non-zero trace, then the sequence $1/a_k$ is summable.
    \end{remark}
    \begin{proof}
      Suppose trace is equal to $-1$ without loss of generality.
      Then the residuals $\Xi_k$ tend to $1$ with $k$ going to infinity. Since $T_{k-1, k}$ as well as $T_{k, k-1}$
        is a summable sequence for any trace operator $T$, we get the required condition.
    \end{proof}
    Let us put the operator $T$ as a finite sum $T = \sum_1^k y^s \otimes x^s$,
      where $x^s, y^s \in \cal{H}$.
    Therefore $T_{ij} = \sum_{s=1}^k {y^s_j x^s_i}$.
    We can rewrite the previous equations:
    \begin{align*}
      \Xi_{2n-1} = a_{2n} \sum_1^k y^s_{2n} x^s_{2n - 1},\\
      \Xi_{2n} = a_{2n + 1} \sum_1^k y^s_{2n} x^s_{2n + 1},
    \end{align*}
    Now we are going to do a simple trick which reveals the essence of the problem.
    Let us define vectors $v_n$ and $u_n$ which lie within $\mathbb{R}^k$ as follows:
    \begin{align*}
      v_n = (x^1_n, x^2_n, \dots ,x^k_n),\\
      u_n = (y^1_n, y^2_n, \dots ,y^k_n). 
    \end{align*}
    Using these definitions we are able to observe that:
    \begin{align}
      \label{vector-eq}
      \Xi_{2n-1} = a_{2n} \langle u_{2n}, v_{2n - 1}\rangle,\\
      \Xi_{2n} = a_{2n + 1} \langle u_{2n}, v_{2n + 1}\rangle.
    \end{align}
    Here the $\langle\cdot, \cdot\rangle$ denotes a standard scalar product in the $\mathbb{R}^k$.
    We want to point out that the existence of such operator $T$
      might be reduced to the existence of vectors $u_n$, $v_n$ in the $\mathbb{R}^k$ (such that $|u_n|$, $|v_n|$ are both square summable) which respects the condition given above.
    Also the condition of trace of $T$ being non-zero reduces to the simple restriction on the vectors $u_n$ and $v_n$:
    \begin{prop}
      There exists a $k$-dimensional operator $T$ such that $\langle Tf_n, f_n^*\rangle  = 0$ for any $n$
      which has a trace equal to $-1$ if and only if there exist such vectors $u_n$, $v_n$ belonging to the
      $\mathbb{R}^k$ that $|u_n|$, $|v_n|$ are square summable, the equations~\eqref{vector-eq} are satisfied and
      $\sum \langle u_n,v_n \rangle = -1$. 
    \end{prop}
    \begin{proof}
      Only the trace condition is left to transform.
      Trace of the operator $T$ is equal to
      \begin{multline*}
        Tr(T) = \sum_{s=1}^k \langle y^s, x^s \rangle = \sum_s \sum_{n=1}^\infty y^s_n x^s_n =\\
              = \sum_n \sum_s y^s_n x^s_n = \sum_n \langle u_n, v_n \rangle.
      \end{multline*}
    \end{proof}
    Just for convenience we might take a new sequence of vectors $w_n$ which incorporates both $u_n$ and $v_n$
    \begin{align*}
      w_{2n} = u_{2n}\\
      w_{2n + 1} = v_{2n + 1}.
    \end{align*}
    Now we are going to examine the vectors $w_n$ that $|w_n|$ is in $l^2$ and
    \begin{align*}
      a_{n} \langle w_{n}, w_{n - 1}\rangle = \Xi_{n - 1}.
    \end{align*}
    Given that the vectors $w_n$ lie in the $\mathbb{R}^k$ we might think of the scalar product as
    the usual product of the vector lengths and the cosinus of the angle between the vectors.
    Namely the existence of $w_n$ is the existence such $W_n := |w_n|$ and real $\theta_n$ that
    $\langle w_{n}, w_{n+1}\rangle = W_n W_{n+1} \cos{\theta_n}.$
    In other words we need to find such $a'_n = a_n \cos{\theta_n}$ that
    the sequence
    \[
      W_n = \frac{\Xi_{n-1}/a'_n}{\Xi_{n-2}/a'_{n-1}} \cdot \frac{\Xi_{n-3}/a'_{n-2}}{\Xi_{n-4}/a'_{n-3}} \cdots
    \]
    belongs to $l^2$. For our conveniency we assume that neither $\Xi_k$ nor $a_k$ are equal to zero (it requires some kind of truncating though it does not alter the proof significantly).\\
    Now since $\Xi_n = -\sum_1^n \langle Te_s, e_s\rangle$ we can see that
    \[
      \frac{\Xi_n}{\Xi_{n-1}} = 1 + \eta_n,
    \]
      where $\eta_n$ is a summable sequence.
    Thus the product of such $(1 + \eta_s)$ fractions is bounded by some constant above.
    That fact allows us to drop all the $\Xi_n$ from the expression above which is important in the case of $k = 1$.
    
    Now we are ready to finish the proof.
    First we prove the theorem for the case $k=1$. Lets start with the proposition:
    \begin{prop}
      There exists a sequence of real numbers $w_n$ which satisfies the equation $a_n w_n w_{n-1} = 1$ if and
      only if the sequence for the statement of the theorem $\mu_n$ belongs to $l^2$.
    \end{prop}
    \begin{proof}
      The proof of the proposition is trivial since $w_n$ would be equal to $\mu_n$ up to a constant.
    \end{proof}
    The formula for the multidimensional case reduces to the expression of $\mu_n$ whenever we
      set all the angles to zero (thus $a'_n$ is equal to $a_n$).
    Due to the idea of setting all the $\Xi_k$ to one, the existence of the one-dimensional operator $T$ is equivalent to the existence of the sequence $w_n$.
    Therefore, when we have such one-dimensional operator $T$, we have proven that $\mu_n$ must belong to $l^2$.\\
    
    Now lets prove the sufficiency of the first statement of the theorem.
    The trace condition on $T$ in the one-dimensional case is 
      equivalent to having $\sum \langle u_n, v_n\rangle $ to be equal to $-1$ as we have already noted.
    Whenever $\mu_n$ belongs to $l^2$, we might take $u_1$ equal to $-1/w_1$ and $w_n$ equal to $\mu_n$.
    All the $u_{2n+1}$ and $v_{2n}$ for all $n>0$ are to be set to zero.
    Such $w_n$ defines an operator with trace equal to $-1$, exactly the one we were looking for.
    \begin{remark}
      Here the necessity is discussed so shortly due to the arguments discussed previously: most importantly we
      were able to replace the residuals $\Xi_k$ with $1$.
    \end{remark}
    \noindent And now we consider the higher dimensions. Lets start our proof of the $k > 1$ case with the 
    \begin{prop}
      There exists a sequence of vectors $w_n$ in the $\mathbb{R}^k$ for any $k > 1$ which satisfy $a_n \langle w_n, w_{n-1} \rangle = 1$ if and only if $1/a_n$ is a summable sequence.
    \end{prop}
    \begin{proof}
      Though we made a remark on the necessity of the summability in the beginning of our proof of the theorem it is still easy to see that
        the existence of a sequence ensures that the $1/a'_n$ is summable and hence the $1/a_n$ is summable as well.
      
      Now we may look at the sufficiency of the condition.
      We have two possible choices to argue here.
      Both are elementary and lead to the same construction but we would like to present several ways of looking at the problem.

      \textbf{Reasoning 1.}\\
      Suppose $\sum 1/a_n$ is a summable series. Let us define $W_n \in l^2$ as:
      \[
        W_n = \max(a_n^{-\frac{1}{2}}, a^{-\frac{1}{2}}_{n+1}).
      \]
      Then the product $W_nW_{n-1}$ will always be bigger than $1/\sqrt{a_n}$ thus giving us a possibility to
      find the angles $\theta_n$ so that 
      \[
        a_n \langle w_n, w_{n-1} \rangle = a_n W_n W_{n-1}\cos{\theta_n} = 1.
      \]
      The construction of the vectors $w_n$ is then done by induction using the lengths $W_n$ and the chosen angles $\theta_n$.
      Obviously any $k$-dimensional eucledian space suits us for $k > 1$.

      \textbf{Reasoning 2.}\\
      Suppose $1/a_n$ is a summable sequence.
      We need to find such $a'_n$ that $|a'_n| \leq |a_n|$ and $\mu'_n = \frac{a'_{n-1} a'_{n-3} \dots}{a'_{n} a'_{n-2} \dots }$ belongs to $\ell^2$.
      Note that essentially we need to reduce $a_n$ in such a way that the fractions of the $\mu_n$ type belong to $\ell^2$ (sic!).
      Suprisingly there is an appropriate sequence $a'_n$ which has a very explicit and compact form:
      \begin{equation*}
        a'_n := \begin{cases}
          \sqrt{a_n a_{n-1}} & \quad a_{n-1} \leq a_n \leq a_{n+1},\\
          \sqrt{a_n a_{n+1}} & \quad a_{n-1} \geq a_n \geq a_{n+1},\\
          a_n & \quad a_{n-1} \geq a_n \leq a_{n+1},\\
          \sqrt{a_{n-1} a_{n+1}} &\quad a_{n-1} \leq a_n \geq a_{n+1}.\\
        \end{cases}
      \end{equation*}
      It is easy to check that for such a sequence $a'_n$ the generated sequence $\mu'_n$ is equal either to $1/\sqrt{a_n}$ or to the $1/\sqrt{a_{n+1}}$.
      Then obviously the constructed $\mu'_n$ belongs to $\ell^2$.
      From this construction we are able to calculate the precise lengths $W_n$ and angles $\theta_n$.
      Given lengths and angles of the whole sequence of vectors $w_n$ is constructed using an induction.
      The dimension $k$ of the enclosing euclidean space is not important here -- the procedure works for any $k$ greater than one.
    \end{proof}
    \begin{remark}
      Note that the trick with the $\Xi_k$ removal is not needed if we are proving the second statement of the theorem.
      It is important only for the one-dimensional case (the necessity implication).
      For the multidimensional case the necessity of
      the sequence to be summable is rather easy to prove, and the construction of the operator $T$ is
      performed along with all $\Xi_k$ being equal to $1$.
    \end{remark}
    Essentially the last proposition proves the second statement of the theorem.
    Observe that during the multidimensional construction of the operator $T$ again
      we are able to choose the first vector $u_1$ so that $\langle u_1,v_1 \rangle = -1$,
      and all the other vectors $u_{2n+1}$ and $v_{2n}$ for all $n>0$ are equal to zero (exactly how we did it in the one-dimensional case).
  \end{proof}
  \begin{remark}
    Note that the cases $k=1$ and $k > 1$ now could be seen as of a great difference since we do not have angles other than zero on a real line.
  \end{remark}
  % \begin{remark}
  %     Right here we can deduce the necessary and sufficient condition for the $k$-completeness property (for $k>1$) having the condition
  %     for the strong approximation property discovered earlier.
  %     As we remember $f_n$ approximates strongly iff the $a_n^{-1}$ is summable. Also we know as a fact that
  %     a system approximates strongly iff for any $k$ it is $k$-complete. Taking the last proposition into account we
  %     understand that the $k$-completeness condition is the same for all $k > 1$ and is identical to the strong approximation condition: $a_n^{-1}$ must be summable.
  %o  \end{remark}


\section{Pentadiagonal example}
  Throughout the paper we explore the vector system $\{f_n\}$ and its biorthogonal system $\{f^*_n\}$ defined as follows:
  \begin{equation}
    \label{main-system}
    \begin{aligned}
      &\mathbf{f_{4j}} = e_{4j} \quad 
      \mathbf{f^*_{4j}} = e_{4j} + d_{2j - 1} e_{4j-2} - b_{2j-1} e_{4j-1} + a_{2j} e_{4j+1} + c_{2j} e_{4j+2}\\
      &\mathbf{f_{4j+1}} = -a_{2j} e_{4j} + e_{4j+1} \quad 
      \mathbf{f^*_{4j+1}} = e_{4j+1} + b_{2j} e_{4j+2},\\
      &\mathbf{f_{4j+2}} = e_{4j+2} + d_{2j} e_{4j} - b_{2j} e_{4j+1} + a_{2j+1} e_{4j+3} + c_{2j+1} e_{4j+4}\quad
      \mathbf{f^*_{4j+2}} = e_{4j+2},\\
      &\mathbf{f_{4j+3}} = e_{4j+3} + b_{2j+1} e_{4j+4}\quad
      \mathbf{f^*_{4j+3}} = -a_{2j+1} e_{4j+2} + e_{4j+3},
    \end{aligned}
  \end{equation}
    where the coefficients $a_n$, $b_n$, $c_n$, $d_n$ are equal to zero whenever $n < 0$, and are bound with the equality
      $c_n + d_n = a_n b_n$ for any $n \geq 0$. 
  \begin{prop}
    The given system is an $M$-basis.
  \end{prop}
  \begin{proof}
    The equality $c_n + d_n = a_n b_n$ guarantees the bi\-orthogonality,
      while the completeness of the $\{f_n\}$ and $\{f_n^*\}$ is easy to check.
  \end{proof}

  \section{Main result}
    \begin{theorem}
      The given system is NOT $k$-complete for some (equivalently any) $k > 1$ if and only if the sequence
      \[
        \mu_n = \min\left(\frac{1}{|a_n|} + \frac{1}{|b_n|}, \frac{1 + |b_n|}{|d_n|}, \frac{1 + |a_n|}{|c_n|}\right)
      \]
        belongs to $\ell^1$.
    \end{theorem}
    \begin{proof}
      %Let us consider a $k$-dimensional operator $T$ such that 
        %$Tr(TR) = 0$ for each $R \in \cal{R}_1(\cal{A})$ which essentially means that
        %$\langle Tf_n, f_n^* \rangle = 0$ for any $n > 0$. 
      Notice that the partial sums of the Fourier series for the given system are somehow close to the
        partial sums of the canonical Fourier series (using the orthonormal basis $e_n$).
      Consider the following differences
      \[
        \Xi_n = \sum_0^n \langle Tf_s, f_s^* \rangle - \sum_0^n \langle Te_s, e_s \rangle,
      \]
        where $\langle \cdot, \cdot\rangle$ denotes the inner product in $\cal{H}$.
      For any $j \geq 0$ we have
      \begin{align*}
        \Xi_{4j} = a_{2j} T_{4j+1, 4j} + c_{2j} T_{4j+2, 4j},\\
        \Xi_{4j + 1} = -d_{2j} T_{4j+2, 4j} + b_{2j} T_{4j+2, 4j+1},\\
        \Xi_{4j + 2} = a_{2j+1} T_{4j+2, 4j+3} + c_{2j+1} T_{4j+2, 4j+4},\\
        \Xi_{4j + 3} = -d_{2j+1} T_{4j+2, 4j+4} + b_{2j+1} T_{4j+3, 4j+4},
      \end{align*}
      where $T_{ij}$ stands for $\langle Te_j, e_i \rangle$.
      \begin{prop}
        \label{inf-dim-statement}
        There exists an operator $T$ with the trace equal to $-1$ such that $\langle Tf_n, f_n^*\rangle = 0$ for any $n \geq 0$
          if and only if the sequence $\left\{\mu_n\right\}$ belongs to $\ell^1$.
      \end{prop}
      \begin{proof}
        Assume that $\mu_n \in \ell^1$.
        We will then construct a required operator $T$.
        Let $T_{00}$ be equal to $-1$, and $T_{jj}$ be equal to zero for any $j > 0$.

        Consider three cases for each $n \geq 0$.

        \noindent\textbf{Case 1.}
        Suppose $\mu_n = 1/|a_n| + 1/|b_n|$.
        For $n=2j$ we set:
        \begin{align*}
          T_{4j+1,4j}&=1/a_n & \quad T_{4j+2,4j} = 0,\\
          T_{4j+2,4j+1}&=1/b_n.
        \end{align*}
        That guarantees an equality $\Xi_{2n} = \Xi_{2n+1} = 1$.
        For $n=2j+1$ we set:
        \begin{align*}
          T_{4j+2,4j+3}&=1/a_n & \quad T_{4j+2,4j+4} = 0,\\
          T_{4j+3,4j+4}&=1/b_n,
        \end{align*}
        which provides an equality $\Xi_{2n} = \Xi_{2n+1} = 1$.

        \noindent\textbf{Case 2.}
        Assume $\mu_n = (1 + |b_n|)/|d_n|$. 
        For $n=2j$ we set
        \begin{align*}
          T_{4j+1,4j} &= b_{2j}/d_{2j} & \quad T_{4j+2,4j} = -1/d_{2j},\\
          T_{4j+2,4j+1} &= 0.
        \end{align*}
        Again, we have $\Xi_{2n} = \Xi_{2n+1} = 1$.
        The case $n = 2j + 1$ is left to the reader.

        \noindent\textbf{Case 3.}
        Suppose $\mu_n = (1 + |a_n|)/|c_n|$. 
        For $n = 2j + 1$ we assign:
        \begin{align*}
          T_{4j+2,4j+3} &= 0 & \quad T_{4j+2,4j+4} = 1/c_{2j+1},\\
          T_{4j+3,4j+4} &= a_{2j+1}/c_{2j+1}.
        \end{align*}
        The case $n = 2j$ is analogous.
        \medskip
        All the other entries $T_{ij}$ we set to zero.
        These assignments ensure that $\Xi_n = 1$ for any $n \geq 0$.

        The constructed operator $T$ belongs to the trace class since all the non-zero elements are summable 
          due to the assumption that $\left\{\mu_n\right\} \in \ell^1$.
        Notice that $T$ annihilates all the rank one operators $f^*_n \otimes f_n$ due to the equality $\Xi_n = 1$.
        Since the trace of the operator $T$ is equal to $-1$, the sufficiency is proved.

        \medskip
        Now assume that there exists a trace class operator $T$ with the trace equal to $-1$, which annihilates all the rank one operatos $f^*_n \otimes f_n$.
        Then all the operator matrix elements $T_{nn}, T_{n, n+1}, T_{n, n+2}$ belong to $\ell^1$ (simple property of a trace class operator).
        As we know, $\Xi_n$ tends to $1$, since the trace of the operator is $-1$.
        Look at the sum $S_{2j} = |T_{4j+1, 4j}| + |T_{4j+2,4j}| + |T_{4j+2,4j+1}|$.
        Observe that $S_{2j}$ is a linear function if considered as a function of $T_{4j+2, 4j}$.
        Its minimum is attained on the boundary of the domain, thus it is greater than $\mu_{2j}/2$ whenever $j$ is sufficiently large.
        The second pair of equalities gives out the minimum value greater than $\mu_{2j+1}/2$ when $j$ is large,
          which shows that the stated condition is necessary for the existence of the operator $T$.
        %\alex{TODO elaborate}
      \end{proof}
      Now consider the case of the $k$-dimensional operator $T$.
      Let us view the operator $T$ as a sum of $k$ rank one operators:
      \[
        T = \sum_1^k y^s \otimes x^s,
      \]
        where $x^s, y^s \in \cal{H}$.
      Let us define vectors $v_n$ and $u_n$ for $n \geq 0 $ which lie in $\mathbb{R}^k$ as follows:
      \begin{align*}
        v_{2j} &= (y^1_{4j}, y^2_{4j}, \dots ,y^k_{4j}) \quad
        &v^*_{2j} = (x^1_{4j}, x^2_{4j}, \dots ,x^k_{4j}) \\
        v_{2j+1} &= (x^1_{4j+2}, x^2_{4j+2}, \dots ,x^k_{4j+2}) \quad
        &v^*_{2j+1} = (y^1_{4j+2}, y^2_{4j+2}, \dots ,y^k_{4j+2}) \\
        u_{2j} &= (x^1_{4j+1}, x^2_{4j+1}, \dots ,x^k_{4j+1}) \quad
        &u^*_{2j} = (y^1_{4j+1}, y^2_{4j+1}, \dots ,y^k_{4j+1}) \\
        u_{2j+1} &= (y^1_{4j+3}, y^2_{4j+3}, \dots ,y^k_{4j+3}) \quad
        &u^*_{2j+1} = (x^1_{4j+3}, x^2_{4j+3}, \dots ,x^k_{4j+3}) 
      \end{align*}
      Note that the sequences $|v_n|$, $|u_n|$ belong to $\ell^2$.
      We can rewrite the previous equations using the introduced vectors.
      \begin{align*}
        \Xi_{4j} &= a_{2j} \langle u_{2j}, v_{2j}\rangle + c_{2j} \langle v_{2j+1}, v_{2j}\rangle,\\
        \Xi_{4j + 1} &= -d_{2j} \langle v_{2j+1}, v_{2j}\rangle + b_{2j} \langle v_{2j+1}, u_{2j}\rangle,\\
        \Xi_{4j + 2} &= a_{2j+1} \langle v_{2j+1}, u_{2j+1} \rangle + c_{2j+1} \langle v_{2j+1}, v_{2j+2} \rangle,\\
        \Xi_{4j + 3} &= -d_{2j+1} \langle v_{2j+1}, v_{2j+2}\rangle + b_{2j+1} \langle u_{2j+1}, v_{2j+2} \rangle.
      \end{align*}
      Here $\langle\cdot, \cdot\rangle$ denotes the scalar product in $\mathbb{R}^k$.
      %\begin{note}
        %Observe that in such setup we always have $\Xi_n = 0$ for any $n \leq 1$.
      %\end{note}
      Now we will study the simplified system instead.
      \begin{align*}
        \Xi_{2j} &= a_{j} \langle u_{j}, v_{j} \rangle  + c_{j} \langle v_{j+1}, v_{j} \rangle,\\
        \Xi_{2j + 1} &= -d_{j} \langle v_{j+1}, v_{j} \rangle + b_{j} \langle v_{j+1}, u_{j}\rangle.
      \end{align*}
      To summarize, we obtained that the existence of an operator $T$ could be reduced to the existence of vectors $u_n$, $v_n$ in the $\mathbb{R}^k$
        which satisfy the conditions given above.
      Given that the sequences of vectors $u_n$, $v_n$ lie in the $\mathbb{R}^k$, we might think of the scalar product as
        the product of the vector lengths and the cosinus of the angle between the vectors.
      \begin{prop}
        \label{k-dim-statement}
        If $\left\{\mu_n\right\}$ belongs to $\ell^1$ then it is possible to construct the vectors $u_n, v_n \in \mathbb{R}^2$ such that
          the equations above are true and for any $n \geq 0$ we have $\Xi_n = 1$.
      \end{prop}
      \begin{corol*}
        For any $k > 1$ there exists a $k$-dimensional operator $T$ with the trace equal to $-1$ such that $\langle Tf_n, f_n^*\rangle = 0$ for any $n \geq 0$
          if and only if the sequence $\left\{\mu_n\right\}$ belongs to $\ell^1$.
      \end{corol*}
      \begin{proof}
        %The operator matrix will look very similar to one we built in the previous proposition.
        Here again we are going to look at three possible values of the $\mu_n$.
        For each $n \geq 0$ we are going to find the vector lengths $V_n = |v_n|$, $U_n = |u_n|$ and three angles:
          $\alpha_n$ which stands for the angle between the vectors $v_n$ and $v_{n + 1}$,
          $\beta_n$ which denotes the angle between $v_n$ and $u_n$,
          and $\gamma_n$ standing for the angle between $v_{n + 1}$ and $u_n$.
        We will write out the vector lengths and define the angles and we will prove that the corresponding vectors could be settled in $\mathbb{R}^2$.
        %We choose the vector $v^*_0$ such that $\langle v_0, v^*_0 \rangle = -1$ and set all
          %the other vectors $u^*_n$, $v^*_n$ to $\vec{0}$.
        %That guarantees that the trace of the constructed operator (if it belongs to the trace class) is equal to $-1$.

        Within the construction we are going to set all $V_n$ step by step.
        In order to do that we are going to define an auxiliary sequence $\{M_n\}_{n=0}^\infty \in \ell^2$ such that $V_n \geq M_n$ for any $n \geq 0$.
        On each step $n$ we are going to define $V_n$ and $M_{n+1}$.
        We start by setting $M_0 = V_0 = 1$.

        \noindent\textbf{Case 1.} Suppose $\mu_n = 1/|a_n| + 1/|b_n|$.

          Here we want to have $v_n$ orthogonal to $v_{n+1}$.
          We set:
          \begin{align*}
              V_n &= \max\left(M_n, \frac{1}{\sqrt{\smash[b]{|a_n|}}}\right),\\
              U_n &= \sqrt{\frac{1}{a_n^2 V_n^2} + \frac{1}{b_n^2 V_{n+1}^2}},\\
              M_{n+1} &= \frac{1}{\sqrt{\smash[b]{|b_n|}}}.
          \end{align*}
          \begin{prop}
            The following inequalities are true.
            \begin{align*}
              M_{n+1} &\leq \sqrt{\mu_n},\\
              U_n &\leq \sqrt{\mu_n},\\
              V_n &\leq \max(\sqrt{\mu_n}, M_n).
            \end{align*}
            There exist such angles $\beta_n$, $\gamma_n$ that with the values $U_n$, $V_n$ defined like this the following is true:
            \begin{align*}
              \langle u_n, v_n \rangle &= 1/a_n,\\
              \langle u_n, v_{n+1} \rangle &= 1/b_n,\\
              \langle v_n, v_{n+1} \rangle &= 0.
            \end{align*}
          \end{prop}
          \begin{proof}
            First part of the proposition is trivial.
            Now we want to understand why there exist such $\beta_n$ and $\gamma_n$ that all the scalar products of
            $v_n$, $v_{n+1}$, $u_n$ satisfy the conditions above.
            Now observe that the chosen $U_n$, $1/(|a_n| V_n)$ and $1/(|b_n| V_{n+1})$ 
              comprise a right triangle with $U_n$ as a hypotenuse.
            As a result there always be such $\beta_n$ and $\gamma_n$ that
            \begin{align*}
              |U_n \cos{\beta_n}| &= \frac{1}{|a_n|V_n},\\
              \left|U_n \cos{\gamma_n}\right| &= \left|U_n \sin{\beta_n}\right| = \frac{1}{|b_n|V_{n+1}}.
            \end{align*}
          \end{proof}
        \noindent\textbf{Case 2.} Assume $\mu_n = (1 + |a_n|)/|c_n|$.

          Here we will have $u_n$ orthogonal to $v_n$.
          Assign:
          \begin{align*}
            M_{n+1} &= \max\biggl(\smash[b]{\frac{\sqrt{|a_n|}}{\sqrt{|c_n|}}}, \frac{1}{\sqrt{\smash[b]{|c_n|}}}\biggr),\\
            V_n &= \max\left(M_n, \frac{2}{\sqrt{\smash[b]{|c_n|}}}\right),\\
            \alpha_n &= \arccos{\frac{1}{c_n V_n V_{n+1}}},\\
            \gamma_n &= \frac{\pi}{2} \pm \alpha_n,\\
            U_n &= \frac{a_n}{c_n \cos{\gamma_n} V_{n+1}}.
          \end{align*}
          \begin{remark*}
            We choose plus or minus in the expression of $\gamma_n$ in order to make the $a_n/(c_n \cos{\gamma_n})$ always positive.
          \end{remark*}
          \begin{prop}
            The angle $\alpha_n$ is defined correctly and the following inequalities are true:
            \begin{align*}
                M_{n+1} &\leq \sqrt{\mu_n},\\
                V_n &\leq \max(2\sqrt{\mu_n}, M_n),\\
                U_n &\leq \sqrt{\mu_n}.
            \end{align*}
            With the values $U_n$, $V_n$, $\gamma_n$, $\alpha_n$ defined like we have
            \begin{align*}
              \langle u_n, v_n \rangle &= 0,\\
              \langle u_n, v_{n+1} \rangle &= a_n/c_n,\\
              \langle v_n, v_{n+1} \rangle &= 1/c_n.
            \end{align*}
          \end{prop}
          \begin{proof}
            Firstly, $\alpha_n$ is defined correctly since
            \[
              V_n V_{n+1} \geq V_n M_{n+1} \geq \frac{2}{\sqrt{\smash[b]{|c_n|}}} \frac{1}{\sqrt{\smash[b]{|c_n|}}} = \frac{2}{|c_n|}
            \]
            which means that the absolute value of the arccosinus argument is always less than $1/2$.
            Now taking into account that $|\cos{\alpha_n}| \leq 1/2$, notice that $|\cos{\gamma_n}| = |\sin{\alpha_n}|$ is always greater than $1/2$.
            It is the fact which we need later.
            
            Trivially, $M_{n+1} \leq \sqrt{\mu_n}$ and $V_n \leq \max(2\sqrt{\mu_n}, M_n)$.
            Next we set
            \[
              U_n = |U_n| \leq 2 \frac{|a_n|}{|c_n|} \frac{1}{V_{n+1}} \leq 2 \frac{\sqrt{|a_n|}}{\sqrt{|c_n|}} \leq \sqrt{\mu_n}.
            \]
            It is easy to notice that the angles and $U_n$ are chosen in such way that the scalar product conditions are 
              satisfied.
            Finally, we are able to lay out the vectors $v_n$, $u_n$ and $v_{n+1}$ with the prescribed angles 
              since we have $\beta_n = \pi/2 = (\pi/2 \pm \alpha_n) \mp \alpha_n = \gamma_n \mp \alpha_n$,
              where we choose the '$\mp$' sign accordingly to our choice in the expression of the $\gamma_n$ angle.
          \end{proof}
        \noindent\textbf{Case 3.} Suppose $\mu_n = (1 + |b_n|)/|d_n|$.

          This case is almost identical to the previous one.
          Here we will have $u_n$ orthogonal to $v_{n+1}$.
          Assign:
          \begin{align*}
            M_{n+1} &= \max\biggl(\smash[t]{\frac{\sqrt{|b_n|}}{\sqrt{|d_n|}}}, \frac{1}{\sqrt{\smash[b]{|d_n|}}}\biggr),\\
            V_n &= \max\left(M_n, \frac{2}{\sqrt{\smash[b]{|d_n|}}}\right),\\
            \alpha_n &= \arccos{\frac{1}{-d_n V_n V_{n+1}}},\\
            \beta_n &= \frac{\pi}{2} \pm \alpha_n,\\
            U_n &= \frac{b_n}{d_n \cos{\beta_n} V_n}.
          \end{align*}
          \begin{remark*}
            We choose plus or minus in the expression of $\beta_n$ in order to make the $b_n/(d_n \cos{\beta_n})$ always positive.
          \end{remark*}
          \begin{prop}
              The angle $\alpha_n$ is defined correctly and the following inequalities are true:
              \begin{align*}
                M_{n+1} &\leq \sqrt{\mu_n},\\
                V_n &\leq \max(2\sqrt{\mu_n}, M_n),\\
                U_n &\leq \sqrt{\mu_n}.
              \end{align*}
              With the values $U_n$, $V_n$, $\beta_n$, $\alpha_n$ defined like this the following is true:
              \begin{align*}
                \langle u_n, v_n \rangle &= b_n/d_n,\\
                \langle u_n, v_{n+1} \rangle &= 0,\\
                \langle v_n, v_{n+1} \rangle &= -1/d_n.
              \end{align*}
          \end{prop}
          \begin{proof}
            Firstly, $\alpha_n$ is defined correctly since
            \[
              V_n V_{n+1} \geq V_n M_{n+1} \geq \frac{2}{\sqrt{\smash[b]{|d_n|}}} \frac{1}{\sqrt{\smash[b]{|d_n|}}} = \frac{2}{|d_n|},
            \]
              which means that the absolute value of the arccosinus argument is always less than $1/2$. Now taking into
              account that $|\cos{\alpha_n}|$ is less than or equal to $1/2$, we always have $|\cos{\beta_n}| > 1/2$.
            
            Trivially, we have $M_{n+1} \leq \sqrt{\mu_n}$ and $V_n \leq \max(2\sqrt{\mu_n}, M_n)$.
            Next
            \[
              U_n = |U_n| \leq 2 \frac{|b_n|}{|d_n|} \frac{1}{V_n} \leq 2 \frac{\sqrt{|b_n|}}{\sqrt{|d_n|}} \leq \sqrt{\mu_n}.
            \]
            It is easy to notice that the angles and $U_n$ are chosen in such way that the scalar product conditions are satisfied.
            Finally, we are able to lay out the vectors $v_n$, $u_n$ and $v_{n+1}$ with the prescribed angles 
              since we have $\gamma_n = \pi/2 = (\pi/2 \pm \alpha_n) \mp \alpha_n = \beta_n \mp \alpha_n$,
              where we choose the '$\mp$' sign accordingly to our choice in the expression of the $\beta_n$ angle.
          \end{proof}
        In each of three cases we guaranteed that $M_{n+1} \leq \sqrt{\mu_n}$ and
          that $V_n \leq \max(M_n, 2\sqrt{\mu_n})$.
        Hence, we get that $V_n$ is bounded up to some constant by $\max(\sqrt{\mu_{n-1}}, \sqrt{\mu_n})$.
        Due to the three propositions above, for any $n \geq 0$ we have $U_n \leq \sqrt{\mu_n}$.
        Thus the constructed sequences $V_n$ and $U_n$ belong to $\ell^2$.

        In the end we choose the vector $v_0^*$ such that $\langle v_0, v_0^*\rangle$ is equal to $-1$.
        It is possible since the vector $v_0$ is not trivial.
        Furthermore, we set all the other $v^*_n$ and $u^*_n$ to the zero vector in order to guarantee that for any $n \geq 0$ we have $\Xi_n = 1$.
        The trace of the constructed operator is obviously equal to $-1$.
      \end{proof}
      The proof of the theorem is finished with the proof of the statements~\ref{inf-dim-statement} and~\ref{k-dim-statement}.
    \end{proof}
  \section{Acknowledgements}
    TODO
    
\medskip
% E-mail: aapyshkin@gmail.com
\bigskip
\noindent{\bf Keywords:} complete minimal system, biorthogonal system, hereditary completeness, strong M-basis, linear summation method.

\end{document}
