\documentclass[12pt]{article}
\usepackage{cmap} % тоже для кодировки
%\usepackage[cp866]{inputenc}
\usepackage[T2A]{fontenc}
\usepackage[utf8]{inputenc} % любая желаемая кодировка
\usepackage[english]{babel}
\usepackage[pdftex,unicode]{hyperref}
\usepackage{amsmath}
\usepackage{amssymb}
\usepackage{amsthm}
\usepackage{amsfonts}
\usepackage{graphicx}
\usepackage[normalem]{ulem}
\usepackage{extsizes}
\usepackage{float}
\usepackage{bbold}
\usepackage{dsfont}
\usepackage{calc}
\usepackage{bm}
%\usepackage [a4paper,% other options: a3paper, a5paper, etc
%  left=3cm,
%  right=1.5cm,
%  top=2cm,
%  bottom=2cm,
%]{geometry}

\usepackage{tocloft}
%\renewcommand{\cfttoctitlefont}{\hspace{0.38\textwidth} \bfseries}
%\renewcommand{\cftbeforetoctitleskip}{-1em}
%\renewcommand{\cftaftertoctitle}{\mbox{}\hfill \\ \mbox{}\hfill{\footnotesize Стр.}\vspace{-.5em}}
%\providecommand{\cftchapfont}{\normalsize\bfseries \sectionname}
%\renewcommand{\cftsecfont}{\hspace{31pt}}
%\renewcommand{\cftsubsecfont}{\hspace{11pt}}
%\providecommand{\cftbeforechapskip}{1em}
%\renewcommand{\cftparskip}{-1mm}
%\renewcommand{\cftdotsep}{1}
%\setcounter{tocdepth}{2} % задать глубину оглавления - до subsection включительно

\usepackage{titlesec}
\sloppy
%\titleformat{\section}
%{\normalsize\bfseries}
%{\thesection}
%{1em}{}

%\titleformat{\subsection}
%{\normalsize\bfseries}
%{\thesubsection}
%{1em}{}

% Настройка вертикальных и горизонтальных отступов
%\titlespacing*{\chapter}{0pt}{-30pt}{8pt}
%\titlespacing*{\section}{\parindent}{*4}{*4}
%
%\linespread{1.3}

\newlength{\widecommentlength}
\setlength{\widecommentlength}{5in}
% \newcommand{\widecommentbox}[2]{\def#1##1{\strut\newline\noindent\colorbox{#2}{\linespread{1}\parbox{.95\textwidth}{\small ##1}}\newline}}
% \usepackage{pgfplots}
\newcommand{\widecommentbox}[3]{\def#1##1{\strut\newline\noindent\colorbox{#3}{\linespread{1}\parbox{.95\textwidth}{\small {\bf [#2]} ##1}}\newline}}
\def\commentsep{\noindent\dotfill}
\newcommand\inner[2]{\langle #1, #2 \rangle}

% To temporarily omit all comments, enable these two lines:
% \renewcommand{\widecommentbox}[3]{\def#1##1{}}
% \let\commentsep\relax

\widecommentbox{\alex}{Alex}{green!20!white}
\widecommentbox{\ad}{AD}{red!20!white}

\usepackage{enumitem}
\usepackage{setspace}

\makeatletter
%\@addtoreset{theorem}{section}
%\@addtoreset{lemma}{section}
\@addtoreset{prop}{section}
\makeatother

%\newcommand{\sectionbreak}{\clearpage}
\usepackage[square,numbers,sort&compress]{natbib}
\usepackage{mathtools}
\renewcommand{\bibnumfmt}[1]{#1.\hfill} % нумерация источников в самом списке - через точку
% \renewcommand{\bibsection}{\section*{Список литературы}} % заголовок специального раздела
\setlength{\bibsep}{0pt}
\newcommand*{\Scale}[2][4]{\scalebox{#1}{\ensuremath{#2}}}%

%\titleformat{\section}[block]{\Large\bfseries\centering}{}{1em}{}
%\titleformat{\subsection}[block]{\large\bfseries\centering}{}{1em}{}
\renewcommand{\cal}[1]{\mathcal{#1}}
\renewcommand{\leq}{\leqslant}
\renewcommand{\geq}{\geqslant}
\renewcommand{\phi}{\varphi}
\newtheorem{theorem}{Theorem}
\newtheorem{prop}{Proposition}
\newtheorem{prop_under_lemma}{Утверждение}
\newtheorem{lemma}{Lemma}
\theoremstyle{definition}
\newtheorem{corol}{Corollary}
\newtheorem{remark}{Remark}
\newtheorem*{remark*}{Remark}
\newtheorem*{note}{Note}
\newtheorem{definition}{Definition}
\newtheorem{example}{Пример}
\newcommand\bigmatrixzero{\raisebox{-0.25\height}{\textnormal{\Huge 0}}}
\newcommand\bigzero{\makebox(10, 10){\text{\Huge 0}}}
\newcommand{\seq}[1]{\{{#1}_n\}_{n=1}^\infty}
\newcommand{\fsys}{\mathbb{F}}

\numberwithin{remark}{section}
\numberwithin{theorem}{section}
\numberwithin{prop}{section}
\numberwithin{equation}{section}
\numberwithin{lemma}{section}
\numberwithin{prop_under_lemma}{lemma}

\begin{document}
% \title{Методы суммирования ряда Фурье \\
% относительно системы Азоффа--Шехада}

% \author{Алексей Пышкин\thanks{Работа поддержана грантом Президента РФ для государственной 
% поддержки молодых российских учёных -- докторов наук МД-5758.2015.1.}}
% \date{}

\section{B-class of vector system examples}
    We are interested in the completeness properties of the minimal vector system $\fsys = \{f_n\}$ in a separable real [TODO complex] Hilbert space $\cal{H}$.
    Suppose that $\cal{H}$ has an orthonormal basis $\{e_j\}_{j=1}^\infty$.
    Let $\{f^*_n\}$ be a biorthogonal vector system to the original system $\{f_n\}$.
    \begin{definition}
        The pair of vector systems $\{f_n\}, \{f^*_n\}$ belongs to the \textit{B-class} whenever the following conditions are satisfied:
        \begin{itemize}
            \item either $f_n = e_n$ or $f^*_n = e_n$ for any $n > 0$,
            \item $\langle f_n, e_n\rangle = \langle f^*_n, e_n \rangle = 1$ for any $n > 0$,
            \item $\langle f_n, e_k \rangle = -\langle f^*_k, e_n \rangle$ for any $n, k > 0$,
            \item the matrices $\{\langle f_n, e_k\rangle\}$ and $\inner{f^*_n}{e_k}$ are both finite-band.
        \end{itemize}
    \end{definition}
    This class of vector systems is the subject of this text.
    \begin{prop}
        The definition given above guarantess the biorthogonality of the $f_n$ and $f^*_n$.
    \end{prop}
    \begin{proof}
        Straightforward check.
    \end{proof}
    \begin{remark}
        Note that the Azoff-Shehada example~\cite{azoff} lies in the B-class.
    \end{remark}
    B-class has an interesting property we are going to exploit later.
    \begin{prop}
        A vector system which belongs to the B-class could be associated with a 
        weighted oriented graph $B(\{f_k\}) = (V, E, w)$ with the following properties:
        \begin{itemize}
            \item The graph $B(\{f_k\})$ having the orientation removed is a bipartite graph.
            \item $w$ is a positive weight function defined on the edge set $E$.
        \end{itemize}
        Also the inverse is true: for each such graph $(V, E, w)$ which satisifies the conditions above there is a unique
        vector system from the B-class.
    \end{prop}
    \begin{proof}
        First we build up an unoriented bipartite graph and later provide the orientation on its edges.
        For each index $l > 0$ such that $f^*_l = e_l$ we put a vertex $v_l$ in the first part of the bipartite graph.
        We will call this part from now on the \textit{left} part of the graph $B(\{f_k\})$.
        For any other index $r > 0$ we construct a vertex in the other part of the graph.
        Evidently, for such indices $r > 0$ the condition $f_r = e_r$ holds due to the definition of the B-class vector systems.
        The second part of the graph will be referred as the \textit{right} part of the graph $B(\{f_k\})$.
        We put an edge between two vertices $v_l$ and $v_r$ from the left and right parts respectively,
        whenever the scalar product $\langle f_l, e_r \rangle$ is not zero.
        
        For these two vertices we have $\langle f_l, e_r \rangle$ = $-\langle f_r, e_l \rangle$.
        Denote by $w_{lr}$ the expression $w_{lr} = \left|\langle f_l, e_r \rangle\right|^{-1}$.
        Obviously $w_{lr} = -w_{rl}$, so if $w_{lr} > 0$ we set the orientation of the edge $(v_l v_r)$ from $v_l$
        to $v_r$, and if $w_{rl} > 0$ we set the orientation of the edge from $v_r$ to $v_l$.
        //todo backwards
    \end{proof}
    We are going to find the necessary and sufficient condition for the $k$-completeness property for the B-class vector system. In order to pursue that
    we need to investigate the conditions, under which the B-class system admits a linear summation method.
    First of all we intend to demonstrate an elegant reformulation of the linear summation method existence problem for
    the B-class vector systems.
    Recall that there exists a linear summation method for the system $\{f_n\}, \{f^*_n\}$
    iff there is no operator $T$ with the trace equal to $1$, such that $\langle Tf_n, f_n^*\rangle = 0$ for any $n$.
    Suppose that there is a bounded operator $T$ such that $\langle Tf_n, f_n^*\rangle = 0$ for any $n$.
    There are two cases: either $f^*_n = e_n$ or $f_n = e_n$.
    In the first case the condition $\langle Tf_n, f_n^*\rangle = 0$ turns into
    \begin{equation}
        \label{left-eqn}
        \sum_j T_{nj} \langle f_n, e_j \rangle = 0,
    \end{equation}
    and in the second case it is equivalent to 
    \begin{equation}
        \label{right-eqn}
        \sum_j T_{jn} \langle f^*_n, e_j \rangle = 0.
    \end{equation}
    Now we are ready to define a function $\cal{F(T)}: E \to \mahtbb{R}$ on the edges of the graph $B(\fsys)$.
    \begin{equation*}
        \cal{F}(v_l, v_r) := T_{lr} \langle f_l, e_r \rangle,\\
        \cal{F}(v_r, v_l) := T_{lr} \langle f^*_r, e_l \rangle.
    \end{equation*}
    Observe that $\cal{F(T)}$ is a skew-symmetric function.
    Moreover, the equalities~\eqref{left-eqn} and~\eqref{right-eqn} are reduced to the simple
    \begin{equation}
        \sum_{u \in V} \cal{F}(v, u) + T_{vv} = 0
    \end{equation}
    for each vertex $v$ from the graph $B(\fsys)$.
    Notice that the flow function defined on the graph $B(\fsys)$ could be easily transformed
    into a \textit{conserving} flow function (in other words, such function that $\sum_{u \in V} \cal{F}(v,u)$ is equal to zero)
    on a slightly modified graph $BG(\fsys)$.
    Namely, we add two auxiliary vertices to the original graph $B(\fsys)$:
    the vertex $\mathit{source}$ and the vertex $\mathit{sink}$.
    Suppose we have enumerated the vertices from the left part of the graph, for instance $\{l_k\}_{k=1}^\infty$.
    Then for each $k > 0$ we connect the $\mathit{source}$ vertex and the $l_k$ vertex with
    the edge $e_k = (\mathit{source} l_k)$,
    and set the flow $\cal{F}(l_k source)$ equal to $T_{l_k l_k}$.
    As we added the edges ${e_k}$, the flow becomes preserved at each vertex of the left part of the graph $BG(\fsys)$.
    In the similar manner we will transform the right part of the graph, as follows.
    Assuming we have the vertices from the right part of the graph $B(\{f_n\})$ in the form of the sequence $\{r_k\}_{k=1}^\infty$.
    So for each $k > 0$ we connect the $r_k$ vertex to the $\mathit{sink}$ with the edge $e'_k$
    and set the flow $\cal{F}(\mathit{sink} r_k)$ equal to $-T_{r_k r_k}$.
    What can we say about the total flow in each of the vertex in the constructed graph $BG(\fsys)$?
    Due to the little trick we used, the total flow became zero in each of the vertices from the left and the right parts.
    The total flow in the $\mathit{source}$ vertex is equal to $\sum T_{l_k l_k}$, and the total flow in the
    $\mathit{sink}$ vertex is now equal to $-\sum T_{r_k r_k}$.
    In order to formalize this construction we are going to need several definintions.
    \begin{definition}
        Consider a graph $G = (V, E)$ with two vertices $\mathit{source}$ and
        $\mathit{sink}$, and a real-valued function $\cal{F}: V \times V \to R$ such that
        for any pair of vertices $u$, $v$ of the graph $\cal{F}(uv) = -\cal{F}(vu)$.
        The graph $G$ we will call a \textit{network}.
        The vertices in the \textit{network} might be called \textit{nodes}, however we will use these terms interchangeably.
        The function $\cal{F}$ will be called a \textit{pseudo-flow}.
    \end{definition}
    \begin{definition}
        Let $\cal{N} = (V, E)$ be a \textit{network}. For each vertex $v \in V$
        denote by $d_{+}(v)$ the sum of the flows \textit{leaving} the node $v$ and by
        $d_{-}(v)$ the sum of the flows \textit{entering} $v$:
        \begin{equation*}
            d_{+}(v) = \sum_{u | \cal{F}(vu) > 0} \cal{F}(vu),
            d_{-}(v) = \sum_{u | \cal{F}(uv) > 0} \cal{F}(uv).
        \end{equation*}
        Also let $d(v)$ be equal to $d_{+}(v) - d_{-}(v)$. Sometimes we will refer to this value as
        a \textit{total flow} of the vertex $v$.
        The vertex is $deficient$ iff $d(v)$ is less than zero,
        $active$ iff $d(v)$ is greater than zero and
        $preserving$ iff $d(v)$ is precisely zero, meaning that the total incoming flow
        is equal to the total outcoming flow of the vertex $v$.
    \end{definition}
    \begin{definition}
        Given a network $\cal{N} = (V, E)$ and a pseudo-flow function $\cal{F}$ we will name $\cal{F}$ a flow function iff 
        any vertex $v\in V\setminus (\mathit{source}\cup\mathit{sink})$ \textit{preserves} the flow, namely
        $d(v) = 0$.
    \end{definition}
    \begin{definition}
        For an \textit{network} $\mathcal{N} =  (V, E)$ and a flow $\cal{F}$ we will say,
        that the flow $\cal{F}$ is \textit{preserving} if $d(\mathit{sink}) = -d(\mathit{source})$ vertex, which
        in simple words means that the total flow coming out of the \textit{source} is equal to the total flow
        coming into the \textit{sink}.
    \end{definition}
    
    For this graph we also define a positive weight function $w: V\times V \to R^{+}$ as follows:
    $$
        w_{rl} := w(v_r, v_l) = \|\langle f^*_r, e_l \rangle^{-1}\|,
    $$
    for an edge $v_r v_l$, which belongs the graph $B(\fsys)$, and
    $$
        w_{uv} := 1
    $$
    for any other edge (namely the edge which has either $\mathit{source}$ or $\mathit{sink}$ as an endpoint).
    \begin{definition}
        
    \end{definition}
    
    \begin{prop}
        A B-class system $\seq{f}$ admits a linear summation method if and only if for a graph $BG(\fsys)$
    \end{prop}
    
    \begin{theorem}
        Such flow is defined correctly 
    \end{theorem}
    \begin{proof}
        
    \end{proof}
    \begin{prop}
        The given system is an $M$-basis if the coefficients satisfy the restrictions above.
    \end{prop}
    \begin{proof}
        The equality $c_n + d_n = a_n b_n$ guarantees the bi\-orthogonality,
        while the completeness of the $\{f_n\}$ and $\{f_n^*\}$ is
        easy to check.
    \end{proof}
    
    \begin{theorem}
        The given system is NOT $k$-complete for any $k > 1$ iff the sequence
        $$
            \mu_n = \min(1/|a_n| + 1/|b_n|, (1 + |b_n|)/|d_n|, (1 + |a_n|)/|c_n|)
        $$ belongs to $l^1$.
    \end{theorem}
    \begin{proof}
        Let us consider a $k$-dimensional operator $T$ such that 
        $Tr(TR) = 0$ for each $R \in \cal{R}_1(\cal{A})$ which essentially means that
        $\langle Tf_n, f_n^* \rangle = 0$ for any $n$. 
        Notice that the partial sums of the fourier series for the given system are somehow close to the
        partial sums of the canonical fourier series (using the orthonormal basis $e_n$). Define
        $$
          \Xi_n := \sum_1^n \langle Tf_s, f_s^* \rangle - \sum_1^n \langle Te_s, e_s \rangle = -\sum_1^n \langle Te_s, e_s \rangle.
        $$
        where the $\langle \cdot, \cdot\rangle$ denotes a standard scalar product in $\cal{H}$.
        These residuals has also a concise form:
        \begin{align*}
            \Xi_{4j} = a_{2j} T_{4j+1, 4j} + c_{2j} T_{4j+2, 4j},\\
            \Xi_{4j + 1} = -d_{2j} T_{4j+2, 4j} + b_{2j} T_{4j+2, 4j+1},\\
            \Xi_{4j + 2} = a_{2j+1} T_{4j+2, 4j+3} + c_{2j+1} T_{4j+2, 4j+4},\\
            \Xi_{4j + 3} = -d_{2j+1} T_{4j+2, 4j+4} + b_{2j+1} T_{4j+3, 4j+4},
        \end{align*}
        having $T_{ij}$ equal to the $\langle Te_j, e_i\rangle$.
        \begin{prop}
            \label{inf-dim-statement}
            There exists an operator $T$ with trace equal to $1$ such that
            $\langle Tf_n, f_n^*\rangle = 0$ for any $n$ if and only if the
            sequence from the statement of the theorem $\mu_n$ belong to $l^1$.
        \end{prop}
        \begin{proof}
            Assume $\mu_n$ is a $l^1$ sequence. We will construct a required operator $T$.
            Let $T_{11}$ be equal to $-1$, and $T_{jj}$ be equal to zero for any other $j$.
            Then for each $n$ we have three cases:
            1. Suppose $\mu_n = 1/|a_n| + 1/|b_n|$. Then if $n$ is $2j$ for some integer $j$, then
            let:
            \begin{align*}
                T_{4j+1,4j}&=1/a_n & \quad T_{4j+2,4j} = 0,\\
                T_{4j+2,4j+1}&=1/b_n.
            \end{align*}
            That guarantees an equality $\Xi_{4j} = \Xi_{4j+1} = 1$.
            If $n$ is $2j+1$ then let:
            \begin{align*}
                T_{4j+2,4j+3}&=1/a_n & \quad T_{4j+2,4j+4} = 0,\\
                T_{4j+3,4j+4}&=1/b_n,
            \end{align*}
            which provides an equality $\Xi_{4j+2} = \Xi_{4j+3} = 1$.
            2. Consider the case when $\mu_n$ is equal to $(1 + |b_n|)/|d_n|$. 
            Suppose $n = 2j$ for some natural $j$.
            Here we assign
            \begin{align*}
                T_{4j+1,4j} &= b_{2j}/d_{2j} & \quad T_{4j+2,4j} = -1/d_{2j},\\
                T_{4j+2,4j+1} &= 0.
            \end{align*}
            The case $n = 2j + 1$ is essentially the same.
            3. Consider the case when $\mu_n$ is equal to $(1 + |a_n|)/|c_n|$. 
            Suppose $n = 2j + 1$ for some natural $j$.
            Here we assign
            \begin{align*}
                T_{4j+2,4j+3} &= 0 & \quad T_{4j+2,4j+4} = 1/c_{2j+1},\\
                T_{4j+3,4j+4} &= a_{2j+1}/c_{2j+1}.
            \end{align*}
            The case $n = 2j$ is essentially the same.
            All the other entries $T_{ij}$ we are setting to zero.
            The constructed operator $T$ obviously belongs to the trace class (since all the non-zero elements are summable 
            on the assumption that $\mu_n$ is a $l^1$ sequence).
            These assignments ensure that all the $\Xi_n$ are equal to $1$ for any $n > 1$.
            Since the trace of the operator $T$ is equal to
            $1$ and it annihilates all the rank one operators $f_n \otimes f^*_n$, the sufficiency is proved.
            \medskip\\
            Now assume that there exists a trace class operator $T$ with all the properties we require.
            Then obviously all the $T_{nn}, T_{n, n+1}, T_{n, n+2}$ elements are summable (simple property of
            a trace class operator matrix).
            As we know $\Xi_n$ tends to $1$ since the trace of the operator is $1$. Let us consider a sum
            $|T_{4j+1, 4j}| + |T_{4j+2,4j}| + |T_{4j+2,4j+1}|$. It depends linearly on $T_{4j+2, 4j}$ and the minimum
            is obtained on the boundary of the domain. The minimum value is greater than $\mu_{2j}/2$ whenever $j$ is
            sufficiently large.
            The second pair of equalities gives out the minimum value greater than $\mu_{2j+1}/2$ when $j$ is large,
            which shows that the stated condition is necessary for the existence of operator $T$.
        \end{proof}
        Now consider the case of the $k$-dimensional operator $T$.
        Let us put the operator $T$ as a sum of $k$ rank one operators:
        $$
            T = \sum_1^k y^s \otimes x^s,
        $$
        where $x^s, y^s \in \cal{H}$
        Therefore $T_{ij} = \sum {y^s_j x^s_i}$.
        Let us define vectors $v_n$ and $u_n$ which lie within $\mathbb{R}^k$ as follows:
        \begin{align*}
            v_{2j} &= (y^1_{4j}, y^2_{4j}, \dots ,y^k_{4j}) \quad
            &v^*_{2j} = (x^1_{4j}, x^2_{4j}, \dots ,x^k_{4j}) \\
            v_{2j+1} &= (x^1_{4j+2}, x^2_{4j+2}, \dots ,x^k_{4j+2}) \quad
            &v^*_{2j+1} = (y^1_{4j+2}, y^2_{4j+2}, \dots ,y^k_{4j+2}) \\
            u_{2j} &= (x^1_{4j+1}, x^2_{4j+1}, \dots ,x^k_{4j+1}) \quad
            &u^*_{2j} = (y^1_{4j+1}, y^2_{4j+1}, \dots ,y^k_{4j+1}) \\
            u_{2j+1} &= (y^1_{4j+3}, y^2_{4j+3}, \dots ,y^k_{4j+3}) \quad
            &u^*_{2j+1} = (x^1_{4j+3}, x^2_{4j+3}, \dots ,x^k_{4j+3}) 
        \end{align*}
        Note that the sequences $|v_n|$, $|u_n|$ belong to $l^2$. Also $u_n$ and $v_n$ are defined for all $n > 0$.
        We can rewrite the previous equations using the introduced vectors:
        \begin{align*}
            \Xi_{4j} &= a_{2j} \langle u_{2j}, v_{2j}\rangle + c_{2j} \langle v_{2j+1}, v_{2j}\rangle,\\
            \Xi_{4j + 1} &= -d_{2j} \langle v_{2j+1}, v_{2j}\rangle + b_{2j} \langle v_{2j+1}, u_{2j}\rangle,\\
            \Xi_{4j + 2} &= a_{2j+1} \langle v_{2j+1}, u_{2j+1} \rangle + c_{2j+1} \langle v_{2j+1}, v_{2j+2} \rangle,\\
            \Xi_{4j + 3} &= -d_{2j+1} \langle v_{2j+1}, v_{2j+2}\rangle + b_{2j+1} \langle u_{2j+1}, v_{2j+2} \rangle.
        \end{align*}
        Here the $\langle\cdot, \cdot\rangle$ denotes a standard scalar product in $\mathbb{R}^k$.
        \begin{note}
        Observe that in such setup we always have $\Xi_n = 0$ for any $n \leq 1$.
        \end{note}
        Since the vectors are in the $R^k$ we are able to reduce the system to
        \begin{align*}
            \Xi_{2j} &= a_{j} \langle u_{j}, v_{j} \rangle  + c_{j} \langle v_{j+1}, v_{j} \rangle,\\
            \Xi_{2j + 1} &= -d_{j} \langle v_{j+1}, v_{j} \rangle + b_{j} \langle v_{j+1}, u_{j}\rangle.
        \end{align*}
        We get that the existence of an operator $T$
        might be reduced to the existence of vectors $u_n$, $v_n$ in the $\mathbb{R}^k$ which
        respects the conditions given above.
        Given that the sequences of vectors $u_n$, $v_n$ lie in the $\mathbb{R}^k$ we might think of the scalar product as
        the usual product of the vector lengths and the cosinus of the angle between the vectors.
        \begin{prop}
            \label{k-dim-statement}
            If $\mu_n$ belongs to $l^1$ then it is possible to construct such vectors $u_n$ and $v_n$ that the equations above are true and for any $n > 1$ we have $\Xi_n = 1$.
        \end{prop}
        \begin{proof}
            The operator matrix is going to look very similar to one we built in the previous proposition.
            Again we are going to look at three possible values of the $\mu_n$.
            For each $n$ we are going to find $V_n = |v_n|$, $U_n = |u_n|$ and three angles:
            $\alpha_n$~--- is the angle between $v_n$ and $v_{n + 1}$, and
            $\beta_n$~--- is the angle between $v_n$ and $u_n$.
            $\gamma_n$~--- is the angle between $v_{n + 1}$ and $u_n$.
            Let $V_1$ be equal to one.
            We choose the $v^*_1$ vector such that $\langle v_1, v^*_1 \rangle = -1$ and set all
            the other $u^*_n$, $v^*_n$ to zero. That guarantees that the trace of the constructed operator (if it
            will fall within the trace class) is equal to $-1$.\\
            Within the construction we are going to set all the $V_n$ inductively. In order to do
            that we are going to define an auxiliary sequence $M_n \in l^2$ such that for any $n$ greater than zero
            $V_n \geq M_n$. On each step $n$ we are going to define $V_n$ and $M_{n+1}$.
            
            So let us start the construction with the setting $M_1 = 1$.\\
            \textbf{Case 1:} $\mu_n = 1/|a_n| + 1/|b_n|$\\
                Here we want to have $v_n$ orthogonal to $v_{n+1}$.
                Set
                \begin{align*}
                    V_n &:= \max\left(M_n, \frac{1}{\sqrt{\smash[b]{|a_n|}}}\right),\\
                    U_n &:= \left(\frac{1}{a_n^2 V_n^2} + \frac{1}{b_n^2 V_{n+1}^2}\right)^\frac{1}{2},\\
                    M_{n+1} &:= \frac{1}{\sqrt{\smash[b]{|b_n|}}}.
                \end{align*}
                \begin{prop}
                    The following inequalities are true:
                    \begin{align*}
                        M_{n+1} &\leq \sqrt{\mu_n},\\
                        U_n &\leq \sqrt{\mu_n},\\
                        V_n &\leq \max(\sqrt{\mu_n}, M_n).
                    \end{align*}
                    There exist such angles $\beta_n$, $\gamma_n$ that with the values $U_n$, $V_n$ defined like this the following is true:
                    \begin{align*}
                        \langle u_n, v_n \rangle &= 1/a_n,\\
                        \langle u_n, v_{n+1} \rangle &= 1/b_n,\\
                        \langle v_n, v_{n+1} \rangle &= 0.
                    \end{align*}
                \end{prop}
                \begin{proof}
                    First part of the proposition is a simple exercise.\\
                    Now we want to understand why there exist such $\beta_n$ and $\gamma_n$ that all the scalar products of
                    $v_n$, $v_{n+1}$, $u_n$ satisfy the conditions above.
                    Now observe that the chosen $U_n$, $1/(|a_n| V_n)$ and $1/(|b_n| V_{n+1})$ 
                    comprise a right triangle with $U_n$ as a hypotenuse.
                    As a result there always be such $\beta_n$ and $\gamma_n$ that
                    \begin{align*}
                        |U_n \cos{\beta_n}| &= \frac{1}{|a_n|V_n},\\
                        \left|U_n \cos{\gamma_n}\right| &= \left|U_n \sin{\beta_n}\right| = \frac{1}{|b_n|V_{n+1}}.
                    \end{align*}
                \end{proof}
            \noindent\textbf{Case 2:} $\mu_n = (1 + |a_n|)/|c_n|$\\
                Here we will have $u_n$ orthogonal to $v_n$.
                Assign:
                \begin{align*}
                    M_{n+1} &:= \max\biggl(\smash[b]{\frac{\sqrt{|a_n|}}{\sqrt{|c_n|}}}, \frac{1}{\sqrt{\smash[b]{|c_n|}}}\biggr),\\
                    V_n &:= \max\left(M_n, \frac{2}{\sqrt{\smash[b]{|c_n|}}}\right),\\
                    \alpha_n &:= \arccos{\frac{1}{c_n V_n V_{n+1}}},\\
                    \gamma_n &:= \frac{\pi}{2} \pm \alpha_n,\\
                    U_n &:= \frac{a_n}{c_n \cos{\gamma_n} V_{n+1}}.
                \end{align*}
                \begin{remark*}
                    We choose plus or minus in the expression of $\gamma_n$ in order to make the $a_n/(c_n \cos{\gamma_n})$ always positive.
                \end{remark*}
                \begin{prop}
                    The angle $\alpha_n$ is defined correctly and the following inequalities are true:
                    \begin{align*}
                        M_{n+1} &\leq \sqrt{\mu_n},\\
                        V_n &\leq \max(2\sqrt{\mu_n}, M_n),\\
                        U_n &\leq \sqrt{\mu_n}.
                    \end{align*}
                    With the values $U_n$, $V_n$, $\gamma_n$, $\alpha_n$ defined like this the following is true:
                    \begin{align*}
                        \langle u_n, v_n \rangle &= 0,\\
                        \langle u_n, v_{n+1} \rangle &= a_n/c_n,\\
                        \langle v_n, v_{n+1} \rangle &= 1/c_n.
                    \end{align*}
                \end{prop}
                \begin{proof}
                    Firstly $\alpha_n$ is defined correctly, since
                    $$
                    V_n V_{n+1} \geq V_n M_{n+1} \geq \frac{2}{\sqrt{\smash[b]{|c_n|}}} \frac{1}{\sqrt{\smash[b]{|c_n|}}}
                     = \frac{2}{|c_n|}
                    $$
                    which means that the absolute value of the arccosinus argument is always less than $1/2$. Now taking into
                    account that $|\cos{\alpha_n}|$ is less than or equal to $1/2$, we have that $|\cos{\gamma_n}| = |\sin{\alpha_n}|$ is always greater than $1/2$. It is the fact which will use later on in this proof.
                    
                    Trivially $M_{n+1} \leq \sqrt{\mu_n}$ and $V_n \leq \max(2\sqrt{\mu_n}, M_n)$. Next
                    $$
                        U_n = |U_n| \leq 2 \frac{|a_n|}{|c_n|} \frac{1}{V_{n+1}} \leq 2 \frac{\sqrt{|a_n|}}{\sqrt{|c_n|}} \leq \sqrt{\mu_n}.
                    $$
                    It is easy to notice that the angles and $U_n$ are chosen in such way that the scalar product conditions are 
                    satisfied.
                    Finally, we are able to lay out the vectors $v_n$, $u_n$ and $v_{n+1}$ with the prescribed angles 
                    since we have $\beta_n = \pi/2 = (\pi/2 \pm \alpha_n) \mp \alpha_n = \gamma_n \mp \alpha_n$,
                    where we choose the '$\mp$' sign accordingly to our choice in the expression of the $\gamma_n$ angle.
                \end{proof}
            \noindent\textbf{Case 3:} $\mu_n = (1 + |b_n|)/|d_n|$\\
                This case is almost identical to the previous one.\\
                Here we will have $u_n$ orthogonal to $v_{n+1}$.
                Assign:
                \begin{align*}
                    M_{n+1} &:= \max\biggl(\smash[t]{\frac{\sqrt{|b_n|}}{\sqrt{|d_n|}}}, \frac{1}{\sqrt{\smash[b]{|d_n|}}}\biggr),\\
                    V_n &:= \max\left(M_n, \frac{2}{\sqrt{\smash[b]{|d_n|}}}\right),\\
                    \alpha_n &:= \arccos{\frac{1}{-d_n V_n V_{n+1}}},\\
                    \beta_n &:= \frac{\pi}{2} \pm \alpha_n,\\
                    U_n &:= \frac{b_n}{d_n \cos{\beta_n} V_n}.
                \end{align*}
                \begin{remark*}
                    We choose plus or minus in the expression of $\beta_n$ in order to make the $b_n/(d_n \cos{\beta_n})$ always positive.
                \end{remark*}
                \begin{prop}
                    The angle $\alpha_n$ is defined correctly and the following inequalities are true:
                    \begin{align*}
                        M_{n+1} &\leq \sqrt{\mu_n},\\
                        V_n &\leq \max(2\sqrt{\mu_n}, M_n),\\
                        U_n &\leq \sqrt{\mu_n}.
                    \end{align*}
                    With the values $U_n$, $V_n$, $\beta_n$, $\alpha_n$ defined like this the following is true:
                    \begin{align*}
                        \langle u_n, v_n \rangle &= b_n/d_n,\\
                        \langle u_n, v_{n+1} \rangle &= 0,\\
                        \langle v_n, v_{n+1} \rangle &= -1/d_n.
                    \end{align*}
                \end{prop}
                \begin{proof}
                    Firstly $\alpha_n$ is defined correctly, since
                    $$
                    V_n V_{n+1} \geq V_n M_{n+1} \geq \frac{2}{\sqrt{\smash[b]{|d_n|}}} \frac{1}{\sqrt{\smash[b]{|d_n|}}}
                     = \frac{2}{|d_n|}
                    $$
                    which means that the absolute value of the arccosinus argument is always less than $1/2$. Now taking into
                    account that $|\cos{\alpha_n}|$ is less than or equal to $1/2$, we always have $|\cos{\beta_n}| > 1/2$.
                    
                    Trivially $M_{n+1} \leq \sqrt{\mu_n}$ and $V_n \leq \max(2\sqrt{\mu_n}, M_n)$. Next
                    $$
                        U_n = |U_n| \leq 2 \frac{|b_n|}{|d_n|} \frac{1}{V_n} \leq 2 \frac{\sqrt{|b_n|}}{\sqrt{|d_n|}} \leq \sqrt{\mu_n}.
                    $$
                    It is easy to notice that the angles and $U_n$ are chosen in such way that the scalar product conditions are 
                    satisfied.
                    Finally, we are able to lay out the vectors $v_n$, $u_n$ and $v_{n+1}$ with the prescribed angles 
                    since we have $\gamma_n = \pi/2 = (\pi/2 \pm \alpha_n) \mp \alpha_n = \beta_n \mp \alpha_n$,
                    where we choose the '$\mp$' sign accordingly to our choice in the expression of the $\beta_n$ angle.
                \end{proof}
            In each of three cases we guaranteed that $M_{n+1} \leq \sqrt{\mu_n}$ and
            that $V_n \leq \max(M_n, 2\sqrt{\mu_n})$. Hence we get that $V_n$ is bounded up to some constant by the $\max(\sqrt{\mu_{n-1}}, \sqrt{\mu_n})$. Due to all three propositions we also have an estimated bound for $U_n$: we always get
            $U_n \leq \sqrt{\mu_n}$. Thus the constructed vector lenghs sequences $V_n$ and $U_n$ belong to $l^2$.\\
            In the end let us choose such $v_1^*$ that $\langle v_1, v_1^*\rangle$ is equal to $-1$.
            It is possible since we earlier set $M_1 = 1$ which means that the vector $v_1$ is not trivial, because it has
            a length greater than zero.
            Additionally we set all the other $v^*_n$ and $u^*_n$ to zero in order to guarantee that for any $n > 1$ we have $\Xi_n$ equal to $1$. The trace of such operator is obviously equal to $-1$. The proposition is proven.
        \end{proof}
        The proof of the theorem is finished with the proof of the statements~\ref{inf-dim-statement} and~\ref{k-dim-statement}.
    \end{proof}
    
    
\medskip
% E-mail: aapyshkin@gmail.com
\bigskip
\begin {thebibliography}{20}
    \bibitem{azoff}
    E.~\!Azoff, H.~\!Shehada,
    \emph{Algebras generated by mutually orthogonal idempotent operators}.
    J. Oper. Theory, 29 (1993), 2, 249--267.
    \bibitem{bbb} 
    A. Baranov, Yu. Belov, A. Borichev,                                       
    \emph{Hereditary completeness for systems of exponentials and reproducing kernels},
    Adv. Math., 235 (2013), 1, 525--554.
    \bibitem{bbb1}
    A. Baranov, Yu. Belov, A. Borichev, 
    \emph{Spectral synthesis in de Branges spaces},
    Geom. Funct. Anal. (GAFA), 25 (2015), 2, 417--452.
    \bibitem{ad_preprint}
    A.D.~\!Baranov, D.V.~\!Yakubovich,
    \emph{Completeness and spectral synthesis of nonselfadjoint one-dimensional
    perturbations of selfadjoint operators}.
    arXiv:1212.5965 [math.FA]
    \bibitem{katavolos}
    A.~\!Katavolos, M.~\!Lambrou, M.~\!Papadakis,
    \emph{On some algebras diagonalized by $M$-bases of $\ell^2$}.
    Integr. Equat. Oper. Theory, 17 (1993), 1, 68--94.
    %\bibitem{wermer}
    %J.~\!Wermer,
    %\emph{On invariant subspaces of normal operators}.
    %Proc. Amer. Math. Soc., 3(1952), 2, 270--277.
    \bibitem{larson}
    D.~\!Larson, W.~\!Wogen,
    \emph{Reflexivity properties of $T\bigoplus0$}.
    J. Funct. Anal., 92 (1990), 448--467.
    %\bibitem{rotfeld}
    %В.В.~\!Пеллер,
    %\emph{Операторы Ганкеля и их приложения}.
    %Издательство РХД, Ижевск(2005).
    %N.K.~\!Nikol'skii,
    %\emph{Complete extensions of Volterra operators},
    %Izv. Akad. Nauk SSSR Ser. Mat 33(1969), 1349--1355. (Russian)

\end{thebibliography}
\vspace{1em}
\noindent{\bf Keywords:} complete minimal system, biorthogonal system, hereditary completeness, strong M-basis, summation method.

\end{document}