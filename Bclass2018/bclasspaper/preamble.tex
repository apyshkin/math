\usepackage{cmap} % тоже для кодировки
%\usepackage[cp866]{inputenc}
\usepackage[T2A]{fontenc}
\usepackage[utf8]{inputenc} % любая желаемая кодировка
\usepackage[english]{babel}
\usepackage[pdftex,unicode]{hyperref}
\usepackage{amsmath}
\usepackage{amssymb}
\usepackage{amsthm}
\usepackage{relsize}
\usepackage{amsfonts}
\usepackage{graphicx}
\usepackage[normalem]{ulem}
\usepackage{extsizes}
\usepackage{float}
\usepackage{bbold}
\usepackage{dsfont}
\usepackage{calc}
\usepackage{bm}
\usepackage{tocloft}
%\renewcommand{\cfttoctitlefont}{\hspace{0.38\textwidth} \bfseries}
%\renewcommand{\cftbeforetoctitleskip}{-1em}
%\renewcommand{\cftaftertoctitle}{\mbox{}\hfill \\ \mbox{}\hfill{\footnotesize Стр.}\vspace{-.5em}}
%\providecommand{\cftchapfont}{\normalsize\bfseries \sectionname}
%\renewcommand{\cftsecfont}{\hspace{31pt}}
%\renewcommand{\cftsubsecfont}{\hspace{11pt}}
%\providecommand{\cftbeforechapskip}{1em}
%\renewcommand{\cftparskip}{-1mm}
%\renewcommand{\cftdotsep}{1}
%\setcounter{tocdepth}{2} % задать глубину оглавления - до subsection включительно

\usepackage{titlesec}
\sloppy
%\titleformat{\section}
%{\normalsize\bfseries}
%{\thesection}
%{1em}{}

%\titleformat{\subsection}
%{\normalsize\bfseries}
%{\thesubsection}
%{1em}{}

% Настройка вертикальных и горизонтальных отступов
%\titlespacing*{\chapter}{0pt}{-30pt}{8pt}
%\titlespacing*{\section}{\parindent}{*4}{*4}
%
%\linespread{1.3}

\newlength{\widecommentlength}
\setlength{\widecommentlength}{5in}
% \newcommand{\widecommentbox}[2]{\def#1##1{\strut\newline\noindent\colorbox{#2}{\linespread{1}\parbox{.95\textwidth}{\small ##1}}\newline}}
\usepackage{pgfplots}
\newcommand{\widecommentbox}[3]{\def#1##1{\strut\newline\noindent\colorbox{#3}{\linespread{1}\parbox{.95\textwidth}{\small {\bf [#2]} ##1}}\newline}}
\def\commentsep{\noindent\dotfill}

% To temporarily omit all comments, enable these two lines:
% \renewcommand{\widecommentbox}[3]{\def#1##1{}}
% \let\commentsep\relax

\widecommentbox{\alex}{AP}{green!20!white}
\widecommentbox{\ad}{AD}{red!20!white}

\usepackage{enumitem}
\usepackage{setspace}

\makeatletter
%\@addtoreset{theorem}{section}
%\@addtoreset{lemma}{section}
\@addtoreset{prop}{section}
\makeatother

\theoremstyle{definition}
%\newcommand{\sectionbreak}{\clearpage}
\usepackage[square,numbers,sort&compress]{natbib}
\usepackage{mathtools}
\renewcommand{\bibnumfmt}[1]{#1.\hfill} % нумерация источников в самом списке - через точку
% \renewcommand{\bibsection}{\section*{Список литературы}} % заголовок специального раздела
\setlength{\bibsep}{0pt}
\newcommand*{\Scale}[2][4]{\scalebox{#1}{\ensuremath{#2}}}%

%\titleformat{\section}[block]{\Large\bfseries\centering}{}{1em}{}
%\titleformat{\subsection}[block]{\large\bfseries\centering}{}{1em}{}
\renewcommand{\cal}[1]{\mathcal{#1}}
\renewcommand{\leq}{\leqslant}
\renewcommand{\geq}{\geqslant}
\renewcommand{\phi}{\varphi}
\newtheorem{theorem}{Theorem}
\newtheorem{prop}{Proposition}
\newtheorem{lemma}{Lemma}
\newtheorem{corol}{Corollary}
\newtheorem{definition}{Definition}
\newtheorem*{definition*}{Definition}
\newtheorem{example}{Example}
\theoremstyle{remark}
\newtheorem{remark}{Remark}
\newtheorem*{remark*}{Remark}
\newtheorem*{note}{Note}
\newcommand\inner[2]{\langle #1, #2 \rangle}
\newcommand\bigmatrixzero{\raisebox{-0.25\height}{\textnormal{\Huge 0}}}
\newcommand\bigzero{\makebox(10, 10){\text{\Huge 0}}}
\newcommand{\seq}[1]{\{{#1}_n\}_{n=1}^\infty}
\newcommand{\fsys}{\mathfrak{F}}
\newcommand{\fstarsys}{\mathfrak{F^{*}}}
\newcommand{\wt}{\mathrm{\hat{w}}}
\newcommand{\wtp}{\mathrm{w}}
\newcommand{\depth}{\operatorname{depth}}
\newcommand{\flow}{\mathcal{\hat{F}}}
\newcommand{\flowpos}{\mathcal{F}}
\newcommand{\preflow}{\mathcal{\tilde{F}}}
\newcommand{\flowposn}[1]{\mathcal{F}_{#1}}
\newcommand{\flown}{\mathcal{\hat{F}}_{n}}
\newcommand{\flowsgn}{\cal{\tilde{F}}}
\newcommand{\source}{\mathit{source}}
\newcommand{\sink}{\mathit{sink}}
\newcommand{\init}{init}
\newcommand{\ter}{ter}
\newcommand{\ein}{in}
\newcommand{\eout}{out}
\newcommand{\eback}{\mathbf{back}}
\newcommand{\efor}{\mathbf{forward}}
\renewcommand{\root}{\mathbf{r}}
\newcommand{\scal}[2]{\langle {#1}, {#2} \rangle}
\newcommand{\net}{\Delta}
\newcommand{\onet}{\vec{\Delta}}
\newcommand{\gpaths}{\cal{P}_{G}}
\newcommand{\gnpaths}{\cal{P}_n}
\newcommand{\gstar}{G^{*}}
\newcommand{\gfi}{\varphi_{G}}
\newcommand{\vspan}[1]{span\left(#1\right)}
\newcommand{\phistar}{\phi_{*}}
\numberwithin{remark}{section}
\numberwithin{theorem}{section}
\numberwithin{prop}{section}
\numberwithin{equation}{section}
\numberwithin{lemma}{section}
