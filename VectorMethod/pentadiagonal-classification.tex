\section{Pentadiagonal example}
  \label{section:pentadiagonal}
  In this section we explore another vector system $\fsys$ and its biorthogonal system $\fstarsys$ defined as follows:
  \begin{equation}
    \label{eq:5system}
    \begin{aligned}
      &\mathbf{f_{4j}} = e_{4j} \quad
      \mathbf{f^*_{4j}} = e_{4j} + d_{2j - 1} e_{4j-2} - b_{2j-1} e_{4j-1} + a_{2j} e_{4j+1} + c_{2j} e_{4j+2}\\
      &\mathbf{f_{4j+1}} = -a_{2j} e_{4j} + e_{4j+1} \quad
      \mathbf{f^*_{4j+1}} = e_{4j+1} + b_{2j} e_{4j+2},\\
      &\mathbf{f_{4j+2}} = e_{4j+2} + d_{2j} e_{4j} - b_{2j} e_{4j+1} + a_{2j+1} e_{4j+3} + c_{2j+1} e_{4j+4}\quad
      \mathbf{f^*_{4j+2}} = e_{4j+2},\\
      &\mathbf{f_{4j+3}} = e_{4j+3} + b_{2j+1} e_{4j+4}\quad
      \mathbf{f^*_{4j+3}} = -a_{2j+1} e_{4j+2} + e_{4j+3},
    \end{aligned}
  \end{equation}
    where the real coefficients $a_n$, $b_n$, $c_n$, $d_n$ are equal to zero whenever $n < 0$, and satisfy the equality
      $c_n + d_n = a_n b_n$ for any $n \geq 0$.
  \begin{prop}
    The given system is an $M$-basis.
  \end{prop}
  \begin{proof}
    The equality $c_n + d_n = a_n b_n$ guarantees the bi\-orthogonality,
      while the completeness of $\fsys$ and $\fstarsys$ is easy to check.
  \end{proof}

  We prove a theorem similar to Theorem~\ref{thm:katavolos}, though we do not investigate the case $k = 1$ in this section.
  %\section{Main result}
    \begin{theorem}
      The given system is $k$ point dense for some (equivalently any) $k > 1$ if and only if the sequence
      \[
        \mu_n = \min\left(\frac{1}{\lvert a_n \rvert} + \frac{1}{\lvert b_n \rvert}, \frac{1 + \lvert b_n\rvert}{\lvert d_n\rvert}, \frac{1 + \lvert a_n\rvert}{\lvert c_n\rvert}\right)
      \]
        does not belong to $\ell^1$.
    \end{theorem}
    \begin{proof}
      In order to investigate the density properties we repeat the reasoning from Section~\ref{sec:preliminaries}.
      Presume that $\Xi_n$ are defined as it is described in Section~\ref{sec:preliminaries}.

      Thus, for any $j \geq 0$ we have
      \begin{align*}
        \Xi_{4j} &= a_{2j} T_{4j+1, 4j} + c_{2j} T_{4j+2, 4j},\\
        \Xi_{4j + 1} &= -d_{2j} T_{4j+2, 4j} + b_{2j} T_{4j+2, 4j+1},\\
        \Xi_{4j + 2} &= a_{2j+1} T_{4j+2, 4j+3} + c_{2j+1} T_{4j+2, 4j+4},\\
        \Xi_{4j + 3} &= -d_{2j+1} T_{4j+2, 4j+4} + b_{2j+1} T_{4j+3, 4j+4},
      \end{align*}
        where $T_{ij}$ stands for $\langle Te_j, e_i \rangle$.
      First of all we want to investigate the conditions of the rank one density property for $\fsys$.
      \begin{prop}
        \label{prop:inf-dim}
        The following statements are equivalent:
        \begin{enumerate}
          \item The system $\fsys$ is not rank one dense;
          %\item There exists an operator $T$ with the trace equal to $-1$ such that $\inner{Tf_n}{f_n^*} = 0$ for any $n \geq 0$;
          \item There exists an operator $T$ such that $TrT = -1$ and for any $n \geq 0$ one has $\Xi_n = -\sum_{s=0}^n \inner{Te_s}{e_s}$;
          \item The sequence $\left\{\mu_n\right\}_{n=1}^\infty$ belongs to $\ell^1$.
        \end{enumerate}
      \end{prop}
      \begin{proof}
        The equivalence of the first two statements is due to Proposition~\ref{prop:reformulation}.
        We are going to prove the equivalence between the last two statements.

        Assume that $\mu_n \in \ell^1$; we are going construct the required operator $T$.
        Let $T_{00}$ be equal to $-1$, and $T_{jj}$ be equal to zero for any $j > 0$.
        Next we consider three cases for each $n \geq 0$.

        \noindent\textbf{Case 1.}
        Suppose $\mu_n = 1/|a_n| + 1/|b_n|$.
        For $n=2j$ we set:
        \[
          T_{4j+1,4j}=1/a_n, \quad T_{4j+2,4j} = 0, \quad T_{4j+2,4j+1}=1/b_n.
        \]
        That guarantees the equality $\Xi_{2n} = \Xi_{2n+1} = 1$.
        For $n=2j+1$ we set:
        \[
          T_{4j+2,4j+3}=1/a_n, \quad T_{4j+2,4j+4} = 0, \quad T_{4j+3,4j+4}=1/b_n,
        \]
        which provides the equality $\Xi_{2n} = \Xi_{2n+1} = 1$.

        \noindent\textbf{Case 2.}
        Assume $\mu_n = (1 + |b_n|)/|d_n|$.
        For $n=2j$ we set
        \[
          T_{4j+1,4j} = b_{2j}/d_{2j}, \quad T_{4j+2,4j} = -1/d_{2j}, \quad T_{4j+2,4j+1} = 0.
        \]
        Again, we have $\Xi_{2n} = \Xi_{2n+1} = 1$.
        For $n = 2j + 1$ we set:
        \[ 
          T_{4j+2,4j+3}=b_{2j+1}/d_{2j+1},  \quad T_{4j+2,4j+4} = -1/d_{2j+1}, \quad T_{4j+3,4j+4}=0,
        \]

        The third case $\mu_n = (1 + |a_n|)/|c_n|$ is left to the reader.
        \medskip

        All the other entries $T_{ij}$ we set to zero.
        These equalities ensure that $\Xi_n = 1 = -\sum_{s=0}^n T_{ss}$ for any $n \geq 0$.

        The constructed operator $T$ belongs to the trace class since the non-zero operator matrix entries are summable
          due to the assumption that $\left\{\mu_n\right\} \in \ell^1$.
        Since the trace of $T$ is equal to $-1$, the sufficiency is proved.

        Now assume that there exists a trace class operator $T$ in the annihilator of $R_1(\cal{A})$ with the trace equal to $-1$.
        We will prove that $\mu_{2j}$ is a summable sequence, the odd case is analogous.

        Since $T$ is in the trace class, the near-diagonal operator matrix elements $T_{nn}, T_{n, n+1}, T_{n, n+2}$ are absolutely summable.
        Then $\nu_n = \lvert T_{nn} \rvert + \lvert T_{n, n + 1} \rvert + \lvert T_{n, n+2} \rvert$ belongs to $\ell^1$.
        It can be easily checked that if for some $n$ one of $a_n$,$b_n$,$c_n$,$d_n$ is equal to zero, then $\nu_n \geq \mu_n/2$.

        For any $n = 2j$ consider the linear function
        \begin{align*}
          g_{n}(x) = \Big\lvert \frac{\Xi_{2n} - c_{n} x}{a_{n}} \Big\rvert +
                   \lvert x \rvert +
                   \Big\lvert \frac{\Xi_{2n+1} + d_{n} x}{b_{n}} \Big\rvert.
        \end{align*}
        Obviously, we have $\nu_{n} = g_{n}(T_{2n+2, 2n})$.
        %\begin{align*}
          %S_{2j} = \Big\lvert \frac{\Xi_{4j} - c_{2j} T_{4j+2, 4j}}{a_{2j}} \Big\rvert +
                   %\Big\lvert T_{4j+2, 4j} \Big\rvert +
                   %\Big\lvert \frac{\Xi_{4j+1} + d_{2j} T_{4j+2, 4j}}{b_{2j}} \Big\rvert.
        %\end{align*}
        This function is piecewise linear, so its minimum is attained in the breakpoints.
        The breakpoints are zero, $y_n = \Xi_{2n}/c_n$ and $z_n = -\Xi_{2n+1}/d_n$.
        Obviously, we have $g_n(0) \geq \mu_n/2$.
        Consider the set $N_1 \subset \mathbb{N}_{{even}}$, such that for any $n \in N_1$ we have $g_n$ attains its minimum in
          the point $\Xi_{2n}/c_n$.
        Thus for any $n\in N_1$ we have $\nu_n \geq g_n(y_n)$.
        We have
        \[
          g_n(y_n) = \Big\lvert \Xi_{2n}/c_n \Big\rvert +
                   \Big\lvert \frac{\Xi_{2n+1} + d_{n} \Xi_{2n}/c_n}{b_{n}} \Big\rvert.
        \]
        Since $\nu_n$ is summable and $\nu_n \geq g_n(y_n) \geq 0.5/|c_n|$ for any $n \in N_1$
          we deduce that $\sum_{n \in N_1} |c_n|^{-1} < \infty$.

        We denote $g_n(y_n) - \Big\lvert \Xi_{2n}/c_n \Big\rvert$ with $G_n$.
        Trivially, $G_n$ is summable.
        Let $\Delta_n$ stand for the difference $\Xi_{2n+1} - \Xi_{2n}$.
        Then
        \[
          G_n = \Big\lvert \frac{c_n \Xi_{2n+1} + d_{n} \Xi_{2n}}{c_n b_n} \Big\rvert
            = \Big\lvert \frac{c_n \Delta_n + (c_n + d_n) \Xi_{2n}}{c_n b_n} \Big\rvert
            = \Big\lvert \frac{c_n \Delta_n + a_n b_n\Xi_{2n}}{c_n b_n} \Big\rvert.
        \]
        Notice that
          $\big\lvert G_n - \Xi_{2n}\lvert a_n/c_n\rvert \big\rvert \leq \lvert \Delta_n/b_n \rvert$.
        Consider $N_2 = \{n\in N_1 \mid 0.5 \leq \lvert b_n \rvert\}$ and $N_3 = N_1 \setminus N_2$.
        Since $\Xi_n$ has a finite limit then $\sum_{n \in N_2} |a_n|/|c_n| < \infty$.
        Hence, $\sum_{n \in N_2} \mu_n < \infty$.

        Presume that $\sum_{n \in N_3} |a_n|/|c_n| = \infty$.
        We have $\lvert |b_n| G_n - (\Xi_{2n}|c_n + d_n|)/|c_n|\rvert \leq \lvert \Delta_n \rvert$.
        Then because both $|b_n| G_n$ and $\Delta_n$ are summable for $n \in N_2$ we see that
          $|c_n + d_n| / |c_n|$ is summable.
        Consequently, it is less then $0.5$ when $n$ is sufficiently big, and so $|d_n| \geq 0.5|c_n|$.
        Hence $\sum_{n \in N_3} 1/|d_n| < \infty$, and since for $n \in N_3$ one has $\lvert b_n \rvert \leq 0.5$,
          $\sum_{n\in N_3} \mu_n \leq \sum_{n \in N_3} 1/\lvert d_n \rvert + \lvert b_n / d_n \rvert < \infty$.

        The latter shows that the $\left\{\mu_n\right\}_{n=1}^\infty$ is a summable sequence.
      \end{proof}
      Now consider the case of the $k$-dimensional operator $T$, we are going to reproduce the reasoning of Section~\ref{sec:preliminaries}.
      We may view the operator $T$ as a sum of $k$ rank one operators: $T = \sum_{s=1}^k y^s \otimes x^s$, where $x^s, y^s \in \cal{H}$.
      Let us define vectors $v_n$ and $u_n$ in $\mathbb{R}^k$ for $n \geq 0 $ as follows:
      \begin{align*}
        v_{2j} &= (y^1_{4j}, y^2_{4j}, \dots ,y^k_{4j}) \quad
        &v^*_{2j} = (x^1_{4j}, x^2_{4j}, \dots ,x^k_{4j}) \\
        v_{2j+1} &= (x^1_{4j+2}, x^2_{4j+2}, \dots ,x^k_{4j+2}) \quad
        &v^*_{2j+1} = (y^1_{4j+2}, y^2_{4j+2}, \dots ,y^k_{4j+2}) \\
        u_{2j} &= (x^1_{4j+1}, x^2_{4j+1}, \dots ,x^k_{4j+1}) \quad
        &u^*_{2j} = (y^1_{4j+1}, y^2_{4j+1}, \dots ,y^k_{4j+1}) \\
        u_{2j+1} &= (y^1_{4j+3}, y^2_{4j+3}, \dots ,y^k_{4j+3}) \quad
        &u^*_{2j+1} = (x^1_{4j+3}, x^2_{4j+3}, \dots ,x^k_{4j+3})
      \end{align*}
      Note that the sequences $\lvert v_n \rvert$, $\lvert u_n \rvert$ belong to $\ell^2$.

      We can rewrite the previous equations using the introduced vectors.
      \begin{align*}
        \Xi_{4j} &= a_{2j} \langle u_{2j}, v_{2j}\rangle + c_{2j} \langle v_{2j+1}, v_{2j}\rangle,\\
        \Xi_{4j + 1} &= -d_{2j} \langle v_{2j+1}, v_{2j}\rangle + b_{2j} \langle v_{2j+1}, u_{2j}\rangle,\\
        \Xi_{4j + 2} &= a_{2j+1} \langle v_{2j+1}, u_{2j+1} \rangle + c_{2j+1} \langle v_{2j+1}, v_{2j+2} \rangle,\\
        \Xi_{4j + 3} &= -d_{2j+1} \langle v_{2j+1}, v_{2j+2}\rangle + b_{2j+1} \langle u_{2j+1}, v_{2j+2} \rangle,
      \end{align*}
        where $\langle\cdot, \cdot\rangle$ denotes the scalar product in $\mathbb{R}^k$.
      Now we will have the following simplified expressions for $\Xi_n$ instead.
      \begin{align*}
        \Xi_{2j} &= a_{j} \langle u_{j}, v_{j} \rangle  + c_{j} \langle v_{j+1}, v_{j} \rangle,\\
        \Xi_{2j + 1} &= -d_{j} \langle v_{j+1}, v_{j} \rangle + b_{j} \langle v_{j+1}, u_{j}\rangle.
      \end{align*}
      To summarize, Proposition~\ref{prop:kreformulation} applies to this set-up as well.

      Next, we are going to analyze the necessary condition of $k$ point density property for $\fsys$.
      \begin{prop}
        \label{prop:2pd}
        If $\left\{\mu_n\right\}_{n=1}^\infty$ belongs to $\ell^1$ then it is possible to construct
          the vectors $u_n, v_n \in \mathbb{R}^2$ such that for any $n \geq 0$ we have $\Xi_n = 1$.
      \end{prop}
      \begin{corol}
        \label{corol:2density}
        If $\fsys$ is $k$ point dense for any $k \geq 2$ then $\sum_{n=1}^\infty \lvert\mu_n\rvert = \infty$.
      \end{corol}
      \begin{proof}[Proof of the corollary]
        Assume the converse: $\{\mu_n\}_{n=1}^\infty \in \ell^1$.

        We apply the proposition and get the vectors $u_n$ and $v_n$.
        Now without loss of generality we can assume that $u_0 \neq 0$.
        Then consider $u^*_0$ so that $\inner{u_0}{u^*_0} = -1$ and set all $u^*_k$ ($k > 0$), $v^*_k$ to zero.
        Since trace of the resulting operator $T$ is equal to $\sum_{s=0}^\infty \inner{u_s}{u^*_s} + \inner{v_s}{v^*_s}$,
          Proposition~\ref{prop:kreformulation} implies that $\fsys$ is not two point dense.
        Trivially when $\fsys$ is not two point dense, it is also not $k$ point dense for any $k \geq 2$.
      \end{proof}
      \begin{proof}[Proof of Proposition~\ref{prop:2pd}]
        %The operator matrix will look very similar to one we built in the previous proposition.
        Here again we are going to look at three possible values of the $\mu_n$.
        For each $n \geq 0$ we are going to find the vector lengths $V_n = |v_n|$, $U_n = |u_n|$ and three angles:
          $\alpha_n$ which stands for the angle between the vectors $v_n$ and $v_{n + 1}$,
          $\beta_n$ which denotes the angle between $v_n$ and $u_n$,
          and $\gamma_n$ standing for the angle between $v_{n + 1}$ and $u_n$.
        We will write out the vector lengths and define the angles and we will prove that the corresponding vectors could be settled in $\mathbb{R}^2$.
        %We choose the vector $v^*_0$ such that $\langle v_0, v^*_0 \rangle = -1$ and set all
          %the other vectors $u^*_n$, $v^*_n$ to $\vec{0}$.
        %That guarantees that the trace of the constructed operator (if it belongs to the trace class) is equal to $-1$.

        Within the construction we are going to set all $V_n$ step by step.
        In order to do that we are going to define an auxiliary sequence $\{M_n\}_{n=0}^\infty \in \ell^2$ such that $V_n \geq M_n$ for any $n \geq 0$.
        On each step $n$ we are going to define $V_n$ and $M_{n+1}$.
        We start by setting $M_0 = V_0 = 1$.

        \noindent\textbf{Case 1.} Suppose $\mu_n = 1/|a_n| + 1/|b_n|$.

          Here we want to have $v_n$ orthogonal to $v_{n+1}$.
          We set:
          \begin{align*}
              V_n &= \max\left(M_n, \frac{1}{\sqrt{\smash[b]{|a_n|}}}\right),\\
              U_n &= \sqrt{\frac{1}{a_n^2 V_n^2} + \frac{1}{b_n^2 V_{n+1}^2}},\\
              M_{n+1} &= \frac{1}{\sqrt{\smash[b]{|b_n|}}}.
          \end{align*}
          \begin{prop}
            The following inequalities are true.
            \begin{align*}
              M_{n+1} &\leq \sqrt{\mu_n},\\
              U_n &\leq \sqrt{\mu_n},\\
              V_n &\leq \max(\sqrt{\mu_n}, M_n).
            \end{align*}
            There exist angles $\beta_n$, $\gamma_n$ such that:
            \begin{equation}
              \label{eqn:case1}
              \begin{aligned}
                \langle u_n, v_n \rangle &= 1/a_n,\\
                \langle u_n, v_{n+1} \rangle &= 1/b_n,\\
                \langle v_n, v_{n+1} \rangle &= 0.
              \end{aligned}
            \end{equation}
          \end{prop}
          \begin{proof}
            First part of the proposition is trivial.
            Now we want to understand why there exist such $\beta_n$ and $\gamma_n$ that all the scalar products of
              $v_n$, $v_{n+1}$, $u_n$ satisfy the conditions~\eqref{eqn:case1}.

            Observe that the chosen $U_n$, $\frac{1}{|a_n| V_n}$ and $\frac{1}{|b_n| V_{n+1}}$
              make up a right triangle with $U_n$ as a hypotenuse.
            As a result there always be such $\beta_n$ and $\gamma_n$ that
            \begin{align*}
              |U_n \cos{\beta_n}| &= \frac{1}{|a_n|V_n},\\
              \left|U_n \cos{\gamma_n}\right| &= \left|U_n \sin{\beta_n}\right| = \frac{1}{|b_n|V_{n+1}}.
            \end{align*}
          \end{proof}
        \noindent\textbf{Case 2.} Assume $\mu_n = (1 + |a_n|)/|c_n|$.

          Here we will have $u_n$ orthogonal to $v_n$.
          Assign:
          \begin{align*}
            M_{n+1} &= \max\biggl(\smash[b]{\frac{\sqrt{|a_n|}}{\sqrt{|c_n|}}}, \frac{1}{\sqrt{\smash[b]{|c_n|}}}\biggr),\\
            V_n &= \max\left(M_n, \frac{2}{\sqrt{\smash[b]{|c_n|}}}\right),\\
            \alpha_n &= \arccos{\frac{1}{c_n V_n V_{n+1}}},\\
            \gamma_n &= \frac{\pi}{2} \pm \alpha_n,\\
            U_n &= \frac{a_n}{c_n \cos{\gamma_n} V_{n+1}}.
          \end{align*}
          \begin{remark*}
            We choose plus or minus in the expression of $\gamma_n$ in order to make the $a_n/(c_n \cos{\gamma_n})$ always positive.
          \end{remark*}
          \begin{prop}
            The angle $\alpha_n$ is defined correctly and the following inequalities are true:
            \begin{align*}
                M_{n+1} &\leq \sqrt{\mu_n},\\
                V_n &\leq \max(2\sqrt{\mu_n}, M_n),\\
                U_n &\leq \sqrt{\mu_n}.
            \end{align*}
            With $U_n$, $V_n$, $\gamma_n$, $\alpha_n$ defined like this, we have
            \begin{equation}
              \label{eqn:case2}
              \begin{aligned}
                \langle u_n, v_n \rangle &= 0,\\
                \langle u_n, v_{n+1} \rangle &= a_n/c_n,\\
                \langle v_n, v_{n+1} \rangle &= 1/c_n.
              \end{aligned}
            \end{equation}
          \end{prop}
          \begin{proof}
            Firstly, $\alpha_n$ is defined correctly since
            \[
              V_n V_{n+1} \geq V_n M_{n+1} \geq \frac{2}{\sqrt{\smash[b]{|c_n|}}} \frac{1}{\sqrt{\smash[b]{|c_n|}}} = \frac{2}{|c_n|},
            \]
              which means that the absolute value of the inverse cosine argument is always less than $1/2$.
            Taking into account that $|\cos{\alpha_n}| \leq 1/2$, notice that $|\cos{\gamma_n}| = |\sin{\alpha_n}|$ is always greater than $1/2$.
            It is the fact which we need later.

            Trivially, $M_{n+1} \leq \sqrt{\mu_n}$ and $V_n \leq \max(2\sqrt{\mu_n}, M_n)$.
            Next we set
            \[
              U_n = |U_n| \leq 2 \frac{|a_n|}{|c_n|} \frac{1}{V_{n+1}} \leq 2 \frac{\sqrt{|a_n|}}{\sqrt{|c_n|}} \leq \sqrt{\mu_n}.
            \]
            It is easy to notice that the angles and $U_n$ are chosen in such way that the scalar product conditions~\eqref{eqn:case2} are satisfied.
            Finally, we are able to lay out the vectors $v_n$, $u_n$ and $v_{n+1}$ with the prescribed angles
              since we have $\beta_n = \pi/2 = (\pi/2 \pm \alpha_n) \mp \alpha_n = \gamma_n \mp \alpha_n$,
              where we choose the '$\mp$' sign accordingly to our choice in the expression of the $\gamma_n$ angle.
          \end{proof}
        \noindent\textbf{Case 3.} Suppose $\mu_n = (1 + |b_n|)/|d_n|$.

          This case is almost identical to the previous one.
          Here we will have $u_n$ orthogonal to $v_{n+1}$.
          Assign:
          \begin{align*}
            M_{n+1} &= \max\biggl(\smash[t]{\frac{\sqrt{|b_n|}}{\sqrt{|d_n|}}}, \frac{1}{\sqrt{\smash[b]{|d_n|}}}\biggr),\\
            V_n &= \max\left(M_n, \frac{2}{\sqrt{\smash[b]{|d_n|}}}\right),\\
            \alpha_n &= \arccos{\frac{1}{-d_n V_n V_{n+1}}},\\
            \beta_n &= \frac{\pi}{2} \pm \alpha_n,\\
            U_n &= \frac{b_n}{d_n \cos{\beta_n} V_n}.
          \end{align*}
          \begin{remark*}
            We choose plus or minus in the expression of $\beta_n$ in order to make the $b_n/(d_n \cos{\beta_n})$ always positive.
          \end{remark*}
          \begin{prop}
              The angle $\alpha_n$ is defined correctly and the following inequalities are true:
              \begin{align*}
                M_{n+1} &\leq \sqrt{\mu_n},\\
                V_n &\leq \max(2\sqrt{\mu_n}, M_n),\\
                U_n &\leq \sqrt{\mu_n}.
              \end{align*}
              With the values $U_n$, $V_n$, $\beta_n$, $\alpha_n$ defined like this the following is true:
              \begin{equation}
                \label{eqn:case3}
                \begin{aligned}
                  \langle u_n, v_n \rangle &= b_n/d_n,\\
                  \langle u_n, v_{n+1} \rangle &= 0,\\
                  \langle v_n, v_{n+1} \rangle &= -1/d_n.
                \end{aligned}
              \end{equation}
          \end{prop}
          \begin{proof}
            Firstly, $\alpha_n$ is defined correctly since
            \[
              V_n V_{n+1} \geq V_n M_{n+1} \geq \frac{2}{\sqrt{\smash[b]{|d_n|}}} \frac{1}{\sqrt{\smash[b]{|d_n|}}} = \frac{2}{|d_n|},
            \]
              which means that the absolute value of the inverse cosine argument is always less than $1/2$.
            Taking into account that $|\cos{\alpha_n}| \leq 1/2$, we always have $|\cos{\beta_n}| > 1/2$.

            Trivially, we have $M_{n+1} \leq \sqrt{\mu_n}$ and $V_n \leq \max(2\sqrt{\mu_n}, M_n)$.
            Next,
            \[
              U_n = |U_n| \leq 2 \frac{|b_n|}{|d_n|} \frac{1}{V_n} \leq 2 \frac{\sqrt{|b_n|}}{\sqrt{|d_n|}} \leq \sqrt{\mu_n}.
            \]
            Again, the angles and $U_n$ are chosen in such way that the scalar product conditions~\eqref{eqn:case3} are satisfied.
            Finally, we are able to lay out the vectors $v_n$, $u_n$ and $v_{n+1}$ with the prescribed  analyze the necessary condition of $k$ point density property for $\fsys$.
              since we have $\gamma_n = \pi/2 = (\pi/2 \pm \alpha_n) \mp \alpha_n = \beta_n \mp \alpha_n$,
              where we choose the '$\mp$' sign accordingly to our choice in the expression of the $\beta_n$ angle.
          \end{proof}
        In each of three cases we guaranteed that $M_{n+1} \leq \sqrt{\mu_n}$ and that $V_n \leq \max(M_n, 2\sqrt{\mu_n})$.
        Hence, we get that $V_n$ is bounded up to some constant by $\max(\sqrt{\mu_{n-1}}, \sqrt{\mu_n})$.
        Due to the three propositions above, for any $n \geq 0$ we have $U_n \leq \sqrt{\mu_n}$.
        Thus the constructed sequences $V_n$ and $U_n$ belong to $\ell^2$.
      \end{proof}
      Now due to Corollary~\ref{corol:2density} we get that the two point density of $\fsys$ implies the divergence of $\sum_{n=1}^\infty \mu_n$,
        which in turn is equivalent to the rank one density property of $\fsys$ (see~\ref{prop:inf-dim}).
      Since rank one density implies $k$ point density for any $k$, the theorem is proved.
    \end{proof}
