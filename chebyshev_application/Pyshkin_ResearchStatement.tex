\documentclass[a4paper,12pt]{article}
\usepackage{cmap} % тоже для кодировки
\usepackage[T2A]{fontenc}
\usepackage[utf8]{inputenc} % любая желаемая кодировка
\usepackage[english, russian]{babel}
\usepackage[pdftex,unicode]{hyperref}
\usepackage{amsmath}
\usepackage{amssymb}
\usepackage{amsthm}
\usepackage{amsfonts}
\usepackage{graphicx}
\usepackage[normalem]{ulem}
\usepackage{extsizes}
%\usepackage{pscyr}
\usepackage{float}
\usepackage{bbold}
\usepackage{dsfont}
\usepackage{calc}
\usepackage [a4paper,% other options: a3paper, a5paper, etc
  left=2cm,
  right=2cm,
  top=2cm,
  bottom=2cm,
]{geometry}

\usepackage{tocloft}
\renewcommand{\cfttoctitlefont}{\hspace{0.38\textwidth} \bfseries}
\renewcommand{\cftbeforetoctitleskip}{-1em}
\renewcommand{\cftaftertoctitle}{\mbox{}\hfill \\ \mbox{}\hfill{\footnotesize Стр.}\vspace{-.5em}}
\providecommand{\cftchapfont}{\normalsize\bfseries \sectionname}
\renewcommand{\cftsecfont}{\hspace{31pt}}
\renewcommand{\cftsubsecfont}{\hspace{11pt}}
\providecommand{\cftbeforechapskip}{1em}
\renewcommand{\cftparskip}{-1mm}
\renewcommand{\cftdotsep}{1}
\newcommand{\cspan}[1]{\overline{span}\left(#1\right)}
%\setcounter{tocdepth}{2} % задать глубину оглавления — до subsection включительно

\usepackage{titlesec}

%\newcommand{\empline}{\mbox{}\newline}
%\newcommand{\likechapterheading}[1]{ 
  %\begin{center}
    %\textbf{#1}
  %\end{center}
  %\empline
%}
%\makeatletter
%\renewcommand{\@dotsep}{2}
%\newcommand{\l@likechapter}[2]{{\bfseries\@dottedtocline{0}{0pt}{0pt}{#1}{#2}}}
%\makeatother
%\newcommand{\likechapter}[1]{    
  %\likechapterheading{#1}    
  %\addcontentsline{toc}{likechapter}{#1}
%}

%\titleformat{\chapter}
%{\bfseries\thechapter.}
%{8pt}
%{\bfseries}{}

\sloppy
\titleformat{\section}
{\normalsize\bfseries}
{\thesection}
{1em}{}

\titleformat{\subsection}
{\normalsize\bfseries}
{\thesubsection}
{1em}{}

% Настройка вертикальных и горизонтальных отступов
%\titlespacing*{\chapter}{0pt}{-30pt}{8pt}
\titlespacing*{\section}{\parindent}{*4}{*4}
\titlespacing*{\subsection}{\parindent}{*4}{*4}
\linespread{1.2}

\newlength{\widecommentlength}
%\setlength{\widecommentlength}{5in}
%%\newcommand{\widecommentbox}[2]{\def#1##1{\strut\newline\noindent\colorbox{#2}{\linespread{1}\parbox{.95\textwidth}{\small ##1}}\newline}}
\usepackage{pgfplots}
\newcommand{\widecommentbox}[3]{\def#1##1{\strut\newline\noindent\colorbox{#3}{\linespread{1}\parbox{.95\textwidth}{\small {\bf [#2]} ##1}}\newline}}
\def\commentsep{\noindent\dotfill}

% To temporarily omit all comments, enable these two lines:
% %\renewcommand{\widecommentbox}[3]{\def#1##1{}}
% %\let\commentsep\relax

\widecommentbox{\alex}{Леша}{green!20!white}
\widecommentbox{\ad}{АД}{red!20!white}

\usepackage{enumitem}
\usepackage{setspace}

\makeatletter
%\@addtoreset{theorem}{section}
%\@addtoreset{lemma}{section}
\@addtoreset{prop}{section}
\makeatother

\newcommand{\sectionbreak}{\clearpage}
\usepackage[square,numbers,sort&compress]{natbib}
\renewcommand{\bibnumfmt}[1]{#1.\hfill} % нумерация источников в самом списке — через точку
\renewcommand{\bibsection}{\section*{Список литературы}} % заголовок специального раздела
\setlength{\bibsep}{0pt}

%\titleformat{\section}[block]{\Large\bfseries\centering}{}{1em}{}
%\titleformat{\subsection}[block]{\large\bfseries\centering}{}{1em}{}
\renewcommand{\cal}[1]{\mathcal{#1}}
\renewcommand{\leq}{\leqslant}
\renewcommand{\geq}{\geqslant}
\renewcommand{\phi}{\varphi}
\newtheorem{theorem}{Теорема}
\newtheorem{prop}{Утверждение}
\newtheorem{prop_under_lemma}{Утверждение}
\newtheorem{definition}{Определение}
\newtheorem{lemma}{Лемма}
\newtheorem{corol}{Следствие}
\newtheorem{remark}{Замечание}
\newtheorem*{note}{Примечание}
\newcommand\bigmatrixzero{\raisebox{-0.25\height}{\textnormal{\Huge 0}}}
\newcommand\bigzero{\makebox{0, 0}{\textnormal{\Huge 0}}}
\newcommand{\system}[1]{\{{#1}_k\}_{k=1}^\infty}

\numberwithin{prop_under_lemma}{lemma}

%\textwidth 165mm
\textheight 220mm
%\renewcommand\baselinestretch{1}
%\topmargin -16mm \oddsidemargin 0pt \evensidemargin 0pt

\begin{document}

\title{Проект исследований <<Наследственная полнота и аппроксимация тождественного оператора в 
  различных операторных топологиях на гильбертовом пространстве>>.}
\author{Пышкин А.В.}
\date{}
\maketitle

В своей дипломной работе я исследовал свойства полноты и наследственной полноты
(сильной М-базисности) систем собственных векторов для некоторых классов компактных
несамосопряженных операторов. Систему векторов называют {\it наследственно полной} (или {\it сильным $M$-базисом}), если
любой вектор $x$ в гильбертовом пространстве $\mathcal{H}$ можно аппроксимировать
линейными комбинациями членов ее ряда Фурье. То есть, если $\{\phi_j\}_{j=1}^\infty$, $\{\psi_j\}_{j=1}^\infty$~---
биортогональные полные системы векторов в $\mathcal{H}$, то для всякого вектора $x \in \mathcal{H}$ верно
$x \in \cspan{\langle\psi_j, x\rangle \phi_j},$
где $\cspan{\cdot}$ обозначает замыкание линейной оболочки.

В частности, был исследован пример равномерно минимальной полной последовательности
с полной биортогональной последовательностью, не обладающей свойством наследственной полноты, приведенный в статьях~\cite{azoff},~\cite{argyroslambrou},~\cite{katavolos}.
\begin{theorem}[см.~\cite{azoff}]
  \label{thm_seq}
  Пусть $\{e_j\}_{j=1}^\infty$~--- ортонормированный базис $\mathcal{H}$. Зададим $\{\phi_j\}$, $\{\psi_j\}$ при $j > 0$:
  \begin{equation}
    \label{main-system}
    \begin{aligned}
      &\phi_1 = e_1 + a_2 e_2, \qquad &\phi_{2j}=e_{2j}, \quad&j > 0,&\\
      &\phi_{2j-1}=-a_{2j-1}e_{2j-2} + e_{2j-1} + a_{2j}e_{2j}, \qquad &\makebox[5em]{} \quad&j > 1,&\\
      &\psi_{2j}=-a_{2j}e_{2j-1}+e_{2j}+a_{2j+1}e_{2j+1}, \qquad &\psi_{2j-1}=e_{2j-1}, \quad&j > 0&
    \end{aligned}
  \end{equation}
  для некоторых вещественных $a_k$.

  Последовательности $\{\phi_j\}$, $\{\psi_j\}$ биортогональны друг другу и полны в $\cal{H}$. Они
  не являются наследственно полными тогда и только тогда, когда обе последовательности
  \begin{equation}
    \label{sequences}
    \begin{aligned}
      &\lambda_k = \frac{a_2 a_4\cdots a_{2k}}{a_3 a_5\cdots a_{2k+1}}\qquad
      \mu_k = \frac{a_3 a_5 \cdots a_{2k - 1}}{a_2 a_4 \cdots a_{2k}}
    \end{aligned}
  \end{equation}
  принадлежат пространству $\ell^2$.
\end{theorem}
В дипломной работе было приведено упрощенное доказательство этой теоремы, а также получены некоторые новые
  свойства вышеописанной системы векторов.
В аспирантуре я продолжал исследование свойств усиленной полноты различных систем векторов
  в гильбертовом пространстве.
В статье Катаволоса, Ламбру и Пападакиса~\cite{katavolos} был предложен более общий подход
  к вопросам наследственной полноты.
Былa рассмотрена банахова алгебра $\cal{A}=\{B\in\cal{B}(\cal{H}) \mid B \phi_n \in [\phi_n]\},$
  где $[\phi_n]$~--- линейное подпространство, порожденное вектором $\phi_n$.
Обозначив за $\cal{R}_1(\cal{A})$ подалгебру $\cal{A}$ операторов ранга один, сформулируем теорему, приведенную в~\cite{katavolos}.
\begin{theorem}[Katavolos et al.]
  \label{thm-katavolos}
  Следующие условия равносильны:
  \begin{enumerate}[label=\textnormal{\textbf{\alph*}}]
    \item \label{linsum} Подалгебра $\cal{R}_1(\cal{A})$ замкнутa в сильной операторной топологии в $\cal{A}$,
    \item Для любых $x, y \in \cal{H}$ и $\epsilon > 0$ существует такой $R\in\cal{R}_1(\cal{A})$,
      что \newline $\|Rx - x\|\leq \epsilon$ и $\|Ry - y\|\leq \epsilon$,
    \item Последовательность $\{1/a_n\}$ не суммируема.
  \end{enumerate}
\end{theorem}
Рассмотрим подалгебру операторов $\cal{R}_1(\cal{A})$.
Легко видеть, что $\cal{R}_1(\cal{A})=\cspan{\phi_n \otimes \psi_n}$.
Заметим, что наследственная полнота~--- это свойство приближения тождественного оператора на гильбертовом пространстве $\cal{H}$ в одной точке 
  линейными комбинациями одномерных операторов вида $\phi_n \otimes \psi_n$.
Тогда второй пункт теоремы~\ref{thm-katavolos} есть нe что иное, как свойство приближения тождественного оператора сразу в двух точках 
  линейными комбинациями одномерных операторов такого же вида.
Первый пункт является более сильным свойством, утверждая, что тождественный оператор возможно приблизить в сильной операторной топологии одномерными операторами.
Этот свойство эквивалентно наличию линейного метода суммирования для ряда Фурье минимальной системы векторов $\{\phi_j\}_{j=1}^\infty$.

%Таким образом, теорема утверждает, что для системы~\eqref{main-system} слабый второй пункт теоремы~\ref{thm-katavolos} 
  %о приближении тождественного оператора в двух точках влечет более сильное свойство (первый пункт).
Таким образом, из теорем~\ref{thm_seq} и~\ref{thm-katavolos} следует, что возможно построение такой наследственно полной системы,
  для которой не существует линейного метода суммирования ее ряда Фурье (пункт~\ref{linsum} теоремы~\ref{thm-katavolos}).
Такую систему можно получить, например, если положить параметры $a_k = 2^k$ в системе~\eqref{main-system}.

Этот пример предоставляет ответ на вопрос, поставленный в статье Лонгстафа~\cite{longstaff} в 1976 году.
Лонгстаф использовал более абстрактную формулировку задачи, в частности, предметом его рассмотрения являлись решетки подпространств гильбертового пространства $\cal{H}$.
В нашем же случае решетка подпространств порождена одномерными подпространствами $[\phi_n]$.
Этот вопрос был так или иначе затронут в статьях~\cite{erdos},~\cite{longstaff},~\cite{laurielongstaff}.
Более подробно история исследования этого вопроса приводится в обзорной статье~\cite{review}.

В конечном счете в начале 90-х годов была опубликована статья~\cite{larson}, в которой и был предложен пример~\eqref{main-system}.
Тот факт, что этот пример дает отрицательный ответ на вопрос Лонгстафа, приводится в статьях~\cite{argyroslambrou} и~\cite{larson}.
В статьях~\cite{azoff},~\cite{katavolos} также проводился детальный анализ примера~\eqref{main-system}.

Нас интересует существование подобных примеров, заданных в базисе гильбертова пространства матрицами с некоторой дополнительной структурой.
Также интересно было бы выяснить, для всех ли систем существует равносильность из теоремы~\ref{thm-katavolos} или 
  все же существуют системы, для которых, например, существует аппроксимация тождественного оператора в двух точках, но не в трех.

Доказательство теоремы~\ref{thm-katavolos}, предложенное в статье~\cite{katavolos}, было достаточно сложным.
Мною был найден альтернативный метод доказательства подобных утверждений,
  опирающееся на элементарные свойства методов суммирования рядов~\cite{my1}.
Этот метод упростил доказательство теоремы~\ref{thm-katavolos}.

Также были построены некоторые конечнодиагональные примеры векторных систем типа~\eqref{main-system}, для которых были доказаны утверждения наподобие теоремы~\ref{thm-katavolos}.
Наконец, был построен класс конечнодиагональных примеров, представляющих из себя обобщение системы~\eqref{main-system}, для которых была доказана теорема вида~\ref{thm-katavolos},
  использую теорию счетных сетей.
В частности, последний результат показал, что систему векторов, предлагающую аппроксимацию в двух точках, но не в трех, невозможно найти среди конечнодиагональных векторных систем
  <<простого>> вида.
В данный момент по этому результату готовится публикация.

Таким образом, на текущий момент основным направлением моей научной работы является построение новых примеров биортогональных систем векторов 
  в гильбертовых пространствах и исследование вышеприведенных свойств для таких систем.
Также приоритетным направлением является поиск наследственно полных систем, для которых аппроксимация в двух точках влечет наличие линейного метода суммирования.
  
До сих пор мной рассматривались системы векторов в абстрактном гильбертовом пространстве.
Однако впоследствии возможно также изучение свойства усиленной полноты для специальных систем векторов в различных функциональных пространствах,
  прежде всего в пространствах аналитических функций.
Такое направление поиска мотивировано недавними результатами А. Баранова и Д. Якубовича~\cite{ad} и А. Баранова, Ю. Белова и А. Боричева \cite{bbb}.
В этих работах особую роль играют системы воспроизводящих ядер в пространствах целых функций и, в частности, в пространстве Пэли-Винера.
Отметим, что существование в пространстве Пэли-Винера наследственно полной системы ядер без свойства приближения тождественного оператора является сложной открытой задачей.

\small
\addcontentsline{toc}{section}{Список литературы}
\begin{thebibliography}{20}
  \bibitem{my1}
    A.~\!Pyshkin,
    \emph{Summation methods for Fourier series with respect to the Azoff-Shehada system,}
    Investigations on linear operators and function theory.
    Part 43, Zap. Nauchn. Sem. POMI, 434, POMI, St. Petersburg, 2015, 116--125;
    J. Math. Sci. 215:5 (2016), 617--623.

  \bibitem{argyroslambrou}
    S.~\!Argyros, M.~\!Lambrou and W.E~\!Longstaff,
    \emph{Atomic Boolean Subspace Lattices and Applications to the Theory of Bases},
    Memoirs. Amer. Math. Soc., No. 445 (1991).

  \bibitem{azoff}
    E.~\!Azoff, H.~\!Shehada,
    \emph{Algebras generated by mutually orthogonal idempotent operators},
    J. Oper. Theory, 29 (1993), 2, 249--267.

  \bibitem{larson}
    D.~\!Larson, W.~\!Wogen,
    \emph{Reflexivity properties of $T\bigoplus0$},
    J. Funct. Anal., 92 (1990), 448--467.

  \bibitem{review}
    J.A~\!Erdos,
    \emph{Basis theory and operator algebras},
    In: A. Katavolos (ed.), Operator Algebras and Application, Kluwer Academic Publishers, 1997, 209--223.

  \bibitem{katavolos}
    A.~\!Katavolos, M.~\!Lambrou and M.~\!Papadakis,
    \emph{On some algebras diagonalized by $M$-bases of $\ell^2$},
    Integr. Equat. Oper. Theory, 17 (1993), 1, 68--94.

  \bibitem{erdos}
    J.A~\!Erdos,
    \emph{Operators of finite rank in nest algebras},
    J. London Math. Soc., 43 (1968), 391--397.

  \bibitem{laurielongstaff}
    C.~\!Laurie, W.~\!Longstaff,
    \emph{A note on rank one operators in reflexive algebras},
    Proc. Amer. Math. Soc., 89 (1983), 293--297.

  \bibitem{longstaff}
    W.E~\!Longstaff,
    \emph{Operators of rank one in reflexive algebras},
    Canadian J. Math., 27 (1976), 19--23.

  \bibitem{ad}
    A.D.~\!Baranov, D.V.~\!Yakubovich,
    \emph{Completeness and spectral synthesis of nonselfadjoint one-dimensional perturbations of selfadjoint operators},
    Advances in Mathematics, 302 (2016), 740--798.

  \bibitem{bbb}
    A.D.~\!Baranov, Y.S.~\!Belov, A.A.~\!Borichev,
    \emph{Hereditary completeness for systems of exponentials and reproducing kernels},
    Advances in Mathematics, 235 (2013), 525--554.
  %N.K.~\!Nikol'skii,
  %\emph{Complete extensions of Volterra operators},
  %Izv. Akad. Nauk SSSR Ser. Mat 33(1969), 1349--1355. (Russian)

\end{thebibliography}
\end{document}
