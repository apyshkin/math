\documentclass[12pt]{amsart}
\usepackage{cmap}
%\usepackage[cp866]{inputenc}
\usepackage[T2A]{fontenc}
\usepackage[utf8]{inputenc}
\usepackage[ngerman, english]{babel}
\usepackage[pdftex,unicode]{hyperref}
\usepackage{amsmath}
\usepackage{amssymb}
\usepackage{amsthm}
\usepackage{verbatim}
\usepackage{relsize}
\usepackage{amsfonts}
\usepackage{graphicx}
\usepackage[normalem]{ulem}
\usepackage{extsizes}
\usepackage{float}
\usepackage{bbold}
\usepackage{dsfont}
\usepackage{calc}
\usepackage{bm}
%\usepackage{tocloft}
%\renewcommand{\cfttoctitlefont}{\hspace{0.38\textwidth} \bfseries}
%\renewcommand{\cftbeforetoctitleskip}{-1em}
%\renewcommand{\cftaftertoctitle}{\mbox{}\hfill \\ \mbox{}\hfill{\footnotesize Стр.}\vspace{-.5em}}
%\providecommand{\cftchapfont}{\normalsize\bfseries \sectionname}
%\renewcommand{\cftsecfont}{\hspace{31pt}}
%\renewcommand{\cftsubsecfont}{\hspace{11pt}}
%\providecommand{\cftbeforechapskip}{1em}
%\renewcommand{\cftparskip}{-1mm}
%\renewcommand{\cftdotsep}{1}
%\setcounter{tocdepth}{2} % задать глубину оглавления - до subsection включительно

%\usepackage{titlesec}
%\sloppy
%\titleformat{\section}
%{\normalsize\bfseries}
%{\thesection}
%{1em}{}

%\titleformat{\subsection}
%{\normalsize\bfseries}
%{\thesubsection}
%{1em}{}

% Настройка вертикальных и горизонтальных отступов
%\titlespacing*{\chapter}{0pt}{-30pt}{8pt}
%\titlespacing*{\section}{\parindent}{*4}{*4}
%
%\linespread{1.3}

\newlength{\widecommentlength}
\setlength{\widecommentlength}{5in}
% \newcommand{\widecommentbox}[2]{\def#1##1{\strut\newline\noindent\colorbox{#2}{\linespread{1}\parbox{.95\textwidth}{\small ##1}}\newline}}
\usepackage{pgfplots}
\newcommand{\widecommentbox}[3]{\def#1##1{\strut\newline\noindent\colorbox{#3}{\linespread{1}\parbox{.95\textwidth}{\small {\bf [#2]} ##1}}\newline}}
\def\commentsep{\noindent\dotfill}

% To temporarily omit all comments, enable these two lines:
% \renewcommand{\widecommentbox}[3]{\def#1##1{}}
% \let\commentsep\relax

\widecommentbox{\alex}{AP}{green!20!white}
\widecommentbox{\ad}{AD}{red!20!white}


%\makeatletter
%\@addtoreset{theorem}{section}
%\@addtoreset{lemma}{section}
%\@addtoreset{prop}{section}
%\makeatother

\usepackage{enumitem}
%\usepackage{setspace}
%\newcommand{\sectionbreak}{\clearpage}
\usepackage[square,numbers,sort&compress]{natbib}
\usepackage{mathtools}
\renewcommand{\bibnumfmt}[1]{#1.\hfill} % нумерация источников в самом списке - через точку
% \renewcommand{\bibsection}{\section*{Список литературы}} % заголовок специального раздела
\setlength{\bibsep}{0pt}
\newcommand*{\Scale}[2][4]{\scalebox{#1}{\ensuremath{#2}}}%

%\titleformat{\section}[block]{\Large\bfseries\centering}{}{1em}{}
%\titleformat{\subsection}[block]{\large\bfseries\centering}{}{1em}{}
\newcommand{\cal}[1]{\mathcal{#1}}
\renewcommand{\leq}{\leqslant}
\renewcommand{\geq}{\geqslant}
\renewcommand{\phi}{\varphi}
\newtheorem{theorem}{Theorem}
\newtheorem*{theorem*}{Theorem}
\newtheorem{prop}{Proposition}
\newtheorem{lemma}{Lemma}
\newtheorem{corol}{Corollary}
\theoremstyle{definition}
\newtheorem{definition}{Definition}
\newtheorem*{definition*}{Definition}
\newtheorem{example}{Example}
\theoremstyle{remark}
\newtheorem{remark}{Remark}
\newtheorem*{remark*}{Remark}
\newtheorem*{note}{Note}
\newcommand\inner[2]{\langle #1, #2 \rangle}
\newcommand\bigmatrixzero{\raisebox{-0.25\height}{\textnormal{\Huge 0}}}
\newcommand\bigzero{\makebox(10, 10){\text{\Huge 0}}}
\newcommand{\seq}[1]{\{{#1}_n\}_{n=1}^\infty}
\newcommand{\fsys}{\mathfrak{F}}
\newcommand{\fstarsys}{\mathfrak{F^{*}}}
\newcommand{\wt}{\mathrm{\hat{w}}}
\newcommand{\wtp}{\mathrm{w}}
\newcommand{\len}{\mathfrak{L}}
\newcommand{\depth}{\operatorname{depth}}
\newcommand{\flow}{\mathcal{\hat{F}}}
\newcommand{\flowpos}{\mathcal{F}}
\newcommand{\preflow}{\mathcal{F^{*}}}
\newcommand{\flowposn}[1]{\mathcal{F}_{#1}}
\newcommand{\flown}{\cal{\hat{F}}_{n}}
\newcommand{\flowsgn}{\cal{\hat{F}}}
\newcommand{\source}{\mathbf{s}}
\newcommand{\sink}{\mathbf{t}}
\newcommand{\init}{init}
\newcommand{\ter}{ter}
\newcommand{\ein}{in}
\newcommand{\eout}{out}
\newcommand{\eback}{\mathbf{back}}
\newcommand{\efor}{\mathbf{forward}}
\renewcommand{\root}{\mathbf{r}}
\newcommand{\scal}[2]{\langle {#1}, {#2} \rangle}
\newcommand{\net}{\Delta}
\newcommand{\onet}{\vec{\Delta}}
\newcommand{\gpaths}{\cal{P}_{G}}
\newcommand{\gnpaths}{\cal{P}_n}
\newcommand{\gstar}{G^{*}}
\newcommand{\gfi}{\varphi_{G}}
\newcommand{\vspan}[1]{span\left(#1\right)}
\newcommand{\cspan}[1]{\overline{span}\left(#1\right)}
\newcommand{\phistar}{\phi_{*}}
\numberwithin{remark}{section}
\numberwithin{theorem}{section}
\numberwithin{prop}{section}
\numberwithin{equation}{section}
\numberwithin{lemma}{section}

\newtheoremstyle{case}{}{}{}{}{}{:}{ }{}
\theoremstyle{case}
\newtheorem{case}{Case}

\begin{document}
\title{On $k$ point density problem for band-diagonal $M$-bases}
\author{Alexey Pyshkin}
\begin{abstract}
  In the early 1990s the works of Larson, Wogen and Argyros, Lambrou, Longstaff
    disclosed an example of a strong tridiagonal $M$-basis that was not rank one dense.
  Later Katavolos, Lambrou and Papadakis studied the $k$ point density property of this example.
  In this paper we present a new method for the $k$ point density property analysis of band-diagonal
    $M$-bases.
\end{abstract}
\keywords{biorthogonal system, $M$-basis, rank one density, two point density}
\thanks{The author was supported by RFBR (the project~16-01-00674).}
\maketitle

\section{Introduction}
  \subsection{Density properties}
    Consider a complete minimal vector system $\fsys = \seq{f}$ in the infinite-dimensional real Hilbert space $\cal{H}$.
    Suppose that $\cal{H}$ has an orthonormal basis $\{e_j\}_{j=0}^\infty$.
    The sequence is called \emph{minimal} if none of its elements can be approximated by the linear combinations of the others.
    The system $\fsys$ is minimal when and only when it possesses a unique biorthogonal system $\fstarsys$.
    We call the minimal system $\fsys$ \emph{band-diagonal} if there is $L \in \mathbb{N}$ such that $\inner{f_k}{e_l} = \inner{f^*_k}{e_l} = 0$
      whenever $\lvert k - l \rvert > L$.
    We say that $\fsys$ is an $M$-basis if $\fstarsys$ is complete as well.
    Let ${\cal{A} = \{T\in B(\cal{H}): Tf_n = \lambda_n f_n,  \text{ for some } \lambda_n \in \mathbb{R}, n \geq 0\}}$ be an operator algebra
      and $R_1(\cal{A})$ be an algebra generated by the rank one operators of $\cal{A}$.
    %We denote by $\lattice$ the lattice of invariant subspaces for operators in $\cal{A}$.

    We are interested in the following properties of the algebra $\cal{A}$.
    %\begin{definition}[one point density property]
      %\label{1pd}
      %We say that $\fsys$ is \emph{one point dense} if for any $x \in \cal{H}$ and $\varepsilon > 0$
        %there exists such $T\in \cal{A}$ that $||Tx - x|| < \varepsilon$.
    %\end{definition}
    \begin{definition}[$k$ point density property]
      \label{kpd}
      We say that the algebra $\cal{A}$ is \emph{$k$ point dense} if for any $x_1, x_2,\dots x_k \in \cal{H}$ and $\varepsilon > 0$
        there exists such $T\in \cal{A}$ that $||Tx_s - x_s|| < \varepsilon$ for any $1 \leq s \leq k$.
    \end{definition}
    The definition for $k=1$ is equivalent to $\fsys$ being a \emph{strong $M$-basis} (see~\cite{katavolos}):
      the system $\fsys$ is called a \emph{strong $M$-basis} if for any $x\in\cal{H}$ we have $x \in \overline{span}\big(\inner{x}{f^*_n}f_n\big)$, where
      $\overline{span}$ denotes a closed linear span.
    \begin{definition}[rank one density property]
      \label{r1d}
      We say that the algebra $\cal{A}$ is \emph{rank one dense} if the unit ball of the rank one subalgebra $R_1(\cal{A})$
        is dense in the unit ball of $\cal{A}$ in the strong operator topology.
    \end{definition}
    By abuse of notation, we say that $\fsys$ is $k$ point dense (rank one dense)
      when the corresponding algebra $\cal{A}$ is $k$ point dense (rank one dense).

    Notice that the rank one density property implies the $k$ point density property for any $k$.
    %Conversely, there is an easy compactness argument which shows that $\fsys$ is rank one dense if it is
    %  $k$ point dense for any $k > 0$.

  \subsection{Motivation}
    %It is well known that the last definition is equivalent to $R_1(\cal{A})$ being dense
      %in $\cal{A}$ in the ultraweak (or $\sigma$-weak) topology.
    %Rank one density property can be considered as a generalization of the notion of \emph{linear summation method} for the system $\fsys$.
    %We say that the system $\fsys$ admits a linear summation method if there exist $\alpha_{kn} \in \mathbb{C}, n \in \mathbb{N}, 0 < k < k_n$
      %such that for any $x \in \cal{H}$,
      %\[
        %\lim_{n \to \infty} \sum_k \alpha_{kn} \inner{x}{f^*_k}f_k = x.
      %\]
    %Clearly, the existence of the linear summation method implies the rank one density property.

    %Notice that the rank one density property~\eqref{r1d} implies the two point density property~\eqref{kpd}, and~\eqref{kpd}
      %in its turn implies the one point density property~\eqref{1pd}.
    %But is it true that $\eqref{1pd}$ implies $\eqref{r1d}$ or that~\eqref{kpd} implies~\eqref{r1d}?

    %It is now a well-known fact that for a subspace lattice generated from an $M$-basis the complete distributivity of its lattice is equivalent to
      %the property~\eqref{1pd} (see~\cite{argyroslambrou} for a proof).
    %In the same paper the following question was raised: does the complete distributivity of $\lattice$ imply the rank one density property?
    %The answer was already known to be positive in the case of a totally ordered lattice~\cite{erdos}, and
      %Longstaff proved it for a finite-dimensional Hilbert space in~\cite{longstaff}.
    %Later Laurie and Longstaff discovered that the rank one density holds in the case of commutative subspace lattices~\cite{laurielongstaff}.
    Long\-staff in~\cite{longstaff} studied abstract subspace lattices and corresponding operator algebras.
    In that paper Longstaff raised an important question: does one point density property always implies the rank one density property?

    The solution remained unknown until Larson and Wogen showed that the answer is negative (see~\cite{larson}).
    They constructed an example of a vector system $\fsys$ such that it is one point dense but does not possess the rank one density property.
    \begin{example}[Larson--Wogen system $\fsys_{LW}$ parameterized with real $a_k$]
      \label{lw-sys}
      For any $j \geq 0$ we define
      \begin{align*}
        &f_{2j+1}=-a_{2j+1}e_{2j} + e_{2j+1} + a_{2j+2}e_{2j+2}, \qquad &f_{2j}=e_{2j},\\
        &f^*_{2j}=-a_{2j}e_{2j-1} + e_{2j} + a_{2j+1}e_{2j+1}, \qquad &f^*_{2j+1}=e_{2j+1},
      \end{align*}
      where $a_n$ are nonzero real numbers for any $n > 0$ and $a_0 = 0$.
    \end{example}
    The construction presented by Larson and Wogen was remarkably simple and elementary,~--- notice that the matrices corresponding to
      the vectors $\{f_j\}_{j=0}^\infty$ and $\{f^*_j\}_{j=0}^\infty$ are both tridiagonal.
    Afterwards this example was also studied in~\cite{argyroslambrou} (see Addendum) and by Azoff and Shehada in~\cite{azoff}.
    Finally, Katavolos, Lambrou and Papadakis in~\cite{katavolos} performed a deep analysis of the density properties
      of this vector system and deduced that for $\fsys_{LW}$ one point density property does not imply rank point density property~\eqref{r1d}.
    Moreover, they showed that for such system two point density is equivalent to rank one density.

    We want to consider band-diagonal systems similar to the one regarded by Larson and Wogen and to determine the exact conditions
      for the $k$ point density property of such vector systems.
    It seems that in the general case it is rather easy to find the parameter conditions under which the system is rank one dense;
      on the other hand, it is not clear how to study $k$ point density property.
    In this paper we present a new technique for analysis of $k$ point density property of a band-diagonal vector system.

    In the next section we will gather some basic facts and outline the main idea of the paper.
    In Section~\ref{section:lw-sys} we perform the analysis for the Larson--Wogen example providing the simpler proof of Theorems 2.1 and 2.2 in~\cite{katavolos}.
    In Section~\ref{section:pentadiagonal} we prove a similar theorem for one pentadiagonal system.
\section{Preliminaries}
  \label{sec:preliminaries}
  Suppose that $\fsys=\{f_n\}_{n=0}^\infty$ is an arbitrary band-diagonal $M$-basis and $\{f^*_n\}_{n=0}^\infty$ is its biorthogonal sequence.
  In this section we establish several facts about this system.
  \begin{prop}
  %It means that $Tr(TR) = 0$ for each $R \in \cal{R}_1(\cal{A})$ which is equivalent to $\inner{Tf_n}{f_n^*} = 0$ for any $n$.
    The system $\fsys$ is rank one dense whenever each trace class operator $T$,
      such that $\inner{Tf_n}{f_n^*} = 0$ for any $n \geq 0$, has a zero trace.
  \end{prop}
  \begin{proof}
    It is well known that the rank one density definition is equivalent to $R_1(\cal{A})$ being dense
      in $\cal{A}$ in the ultraweak (or $\sigma$-weak) topology (see~\cite{katavolos}, Theorem 2.2).
  \end{proof}
  \begin{prop}
    The system $\fsys$ is $k$ point dense whenever each $k$-dimensional operator $T$,
      such that $\inner{Tf_n}{f_n^*} = 0$ for any $n \geq 0$, has a zero trace.
  \end{prop}
  \begin{proof}
    In the paper~\cite{katavolos} authors proved the proposition for $k = 2$.
    For greater $k$s the same reasoning works.
  \end{proof}
  For an arbitrary linear operator $T$ we can define the differences between
    the partial sums of the Fourier series using the system $\fsys$ and
    partial sums of the canonical Fourier series (using the orthonormal basis $e_k$).
  \begin{equation}
    \label{eq:xi}
    \Xi_n = -\sum_{m=0}^n \inner{Te_m}{e_m} + \sum_{m=0}^n \inner{Tf_m}{f_m^*},
  \end{equation}
    where the $\langle \cdot, \cdot\rangle$ denotes a standard scalar product in $\cal{H}$.
  It appears that $\Xi_n$ takes a concise form for a finite-band system $\fsys$, and it is much easier to study $\Xi_n$
    than, for example, $\inner{Tf_s}{f_s^*}$.

  \begin{prop}
    \label{prop:reformulation}
    Notice that $T$ is a trace class operator annihilating the subalgebra $R_1(\cal{A})$ if and only if
      for any $n \geq 0$ one has $\Xi_n = -\sum_{m=0}^n \inner{Te_m}{e_m}$.
  \end{prop}
    We will work with this formulation in the following sections.

  Now assume that the operator $T$ has a finite rank.
  In that case we write $T$ as a finite sum $T = \sum_{s=1}^k y^s \otimes x^s$,
    where $x^s, y^s \in \cal{H}$.

  Let us define vectors $v_n$ and $u_n$ which lie within $\mathbb{R}^k$ as follows:
  \begin{align*}
    v_n &= (x^1_n, x^2_n, \dots, x^k_n),\\
    u_n &= (y^1_n, y^2_n, \dots, y^k_n),
  \end{align*}
  where $x^s_n = \inner{x^s}{e_j}$.
  Since $\inner{Te_k}{e_l} = \inner{u_k}{v_l}$ for any $k$ and $l$, we can rewrite $\Xi_n$ in terms of
    the scalar products of $\{u_n\}_{n=0}^\infty$ and $\{v_n\}_{n=0}^\infty$.
  In turn it means that the condition from Proposition~\ref{prop:reformulation} $\Xi_n = -\sum_{m=0}^n \inner{Te_m}{e_m}$
    can be rewritten in terms of the scalar products of $\{u_n\}_{n=0}^\infty$ and $\{v_n\}_{n=0}^\infty$.

  Hence, the existence of $T$ might be reduced to the existence of
    the vectors $u_n$, $v_n$ in $\mathbb{R}^k$ (such that the sequences $\{\lvert u_n\rvert\}$, $\{\lvert v_n\rvert\}$ are both square summable)
    which satisfy the condition given in Proposition~\ref{prop:reformulation}.
  Thus, instead of looking for $k$ Hilbert elements $x^s$ and $y^s$ we might look for an infinite sequence
    of $k$-dimensional vectors $v_n$ and $u_n$ such that $\lvert v_n \rvert$, $\lvert u_n \rvert$ belong to $\ell^2$.
  Obviously we can also write the trace of $T$ in terms of $u_n$, $v_n$.
  \begin{align*}
    Tr T &= \sum_{s=1}^k \langle y^s, x^s \rangle = \sum_{s=1}^k \sum_{n=0}^\infty y^s_n x^s_n =\\
         &= \sum_{n=0}^\infty \sum_{s=1}^k y^s_n x^s_n = \sum_{n=0}^\infty \langle u_n, v_n \rangle.
  \end{align*}

  Thus we have just found the following reformulation for the $k$ point density property.
  \begin{prop}
    \label{prop:kreformulation}
    There exists a $k$-dimensional operator $T$ which annihilates $R_1(\cal{A})$
      such that $Tr\, T \neq 0$ if and only if
      there exist vectors $u_n$, $v_n \in \mathbb{R}^k$ such that
      $\{\lvert v_n \rvert\}$, $\{\lvert u_n \rvert\} \in \ell^2$,
      $\sum_{n=0}^\infty \inner{u_n}{v_n} \neq 0$, and the following holds:
      \begin{equation}
        \label{eq:thm}
        \Xi_n = -\sum_{m=0}^n \inner{u_m}{v_m},
      \end{equation}
      for any $n \geq 0$.
  \end{prop}
  Essentially, the last theorem asserts that the $k$ point density property can viewed as a
    possibility of laying out the sequence of vectors in $\mathbb{R}^k$ which is constricted with
    a series of relations~\eqref{eq:thm}.
\section{Classification for the Larson--Wogen $M$-basis}
  \label{section:lw-sys}
  In this section we study Example~\eqref{lw-sys} of Larson--Wogen vector system $\fsys_{LW}$.
  Namely, we prove a theorem similar to Theorem 2.2 of~\cite{katavolos}.
  Up until now there existed two different techniques in studying the $k$ point density for $k=1$ (strong $M$-basisness) and for $k\geq2$.
  Here we demonstrate a universal method for the analysis of the $k$ point density property.
  \begin{theorem}[\cite{katavolos}, Theorem 2.2]
    \label{thm:katavolos}
    The sequences $\{f_j\}_{j=0}^\infty$, $\{f^*_j\}_{j=0}^\infty$ are biorthogonal
      and both are complete in $\cal{H}$ (so $\fsys_{LW}$ is an $M$-basis).
    Moreover, the following is true.
    \begin{itemize}
      \item  The system~\eqref{lw-sys} is one point dense (a strong $M$-basis) if and only if the sequence
        \begin{equation}
          \mu_n = \frac{a_{n-1} a_{n-3} \dots}{a_{n} a_{n-2} \dots }
        \end{equation}
        does not belong to $\ell^2$.
      \item The system is $k$ point dense ($k > 1$) if and only if the sequence $\{1/a_n\}_{n=1}^\infty$ does not belong to $\ell^1$.
    \end{itemize}
  \end{theorem}
  \begin{proof}
    Consider a $k$-dimensional operator $T = \sum_{s=1}^k y^s \otimes x^s$ which annihilates $R_1(\cal{A})$.
    We reproduce the logic from the previous section and define $\Xi_n$, $u_n$, $v_n$ exactly in the way it is described there.
    For the given $M$-basis $\fsys = \fsys_{LW}$ we calculate $\Xi_n$ precisely.
    \begin{align*}
      \Xi_{2n-1} &= a_{2n}T_{2n - 1, 2n},\\
      \Xi_{2n} &= a_{2n + 1}T_{2n + 1, 2n},
    \end{align*}
      where $T_{ij} = \inner{Te_j}{e_i}$.
    %\begin{remark}
      %If such operator $T$ exists and has a non-zero trace, then the sequence $1/a_k$ is summable.
    %\end{remark}
    %\begin{proof}
      %Suppose the trace is equal to $-1$ without loss of generality.
      %Then $\Xi_k$ tend to $1$ with $k$ going to infinity. Since $T_{k-1, k}$ as well as $T_{k, k-1}$
        %is a summable sequence for any trace operator $T$, we get the required condition.
    %\end{proof}

    We deduce new expressions for $\Xi_n$ via the vectors $u_n$ and $v_n$,
      also defined in the previous section.
    %Next we express $\Xi_n$ using the scalar product of the vectors $u_n$ and $v_n$, defined in the previous section.
    \begin{equation}
      \begin{aligned}
        \Xi_{2n-1} &= a_{2n} \sum_{s=1}^k y^s_{2n} x^s_{2n-1} = a_{2n} \inner{u_{2n}}{v_{2n-1}},\\
        \Xi_{2n} &= a_{2n+1} \sum_{s=1}^k y^s_{2n} x^s_{2n+1} = a_{2n+1} \inner{u_{2n}}{v_{2n+1}},
      \end{aligned}
    \end{equation}
      where $\langle\cdot, \cdot\rangle$ denotes the scalar product in $\mathbb{R}^k$.

    Now we introduce the sequences of vectors $w_n$ and $w^*_n$.
    \begin{align*}
      w_{2n} &= u_{2n} \quad w^*_{2n} = v_{2n},\\
      w_{2n+1} &= v_{2n+1} \quad w^*_{2n+1} = u_{2n+1}.
    \end{align*}
    Proposition~\ref{prop:kreformulation} asserts that $k$ point density is equivalent to the existence of
      such $k$-dimensional vectors $w_m$, $w^*_m$ lying in $\ell^2(\mathbb{R}^k)$ which comply with the equation
    \begin{equation}
      \label{eq:vector2}
      \sum_{m=0}^n \inner{w_m}{w^*_m} = -a_{n+1} \inner{w_m}{w_{m+1}},
    \end{equation}
      for any $m \geq 0$, such that $\sum_{m=0}^\infty \inner{w_m}{w^*_m} \neq 0$.

    In what follows we prove that the conditions in the previous proposition can be simplified.
    \begin{prop}
      \label{prop:reformulation-lw}
      Such $k$-dimensional vectors $w_s$, $w^*_s$ do exist if and only if there exists such $r_s$ in $\ell^2(\mathbb{R}^k)$
        satisfying the following.
      \begin{equation}
        \label{eq:vector3}
        a_{n+1} \inner{r_n}{r_{n+1}} = 1,
      \end{equation}
      for any $n \geq 0$.
    \end{prop}
    \begin{proof}
      Suppose we found such $r_n$.
      Then we solve~\eqref{eq:vector2} by putting $w^*_s$ to zero for any $s > 0$, $w_s$ to $r_s$ and
        choose the vector $w^*_0$ so that $\inner{w_0}{w^*_0} = -1$.

      Now we prove the converse.
      Suppose we found such $w_n$ that~\eqref{eq:vector2} holds.
      Given that the vectors $w_n$ lie in the $\mathbb{R}^k$, we rewrite the scalar product as
        the product of the vector lengths and the cosine of the angle between the vectors.
      Namely, we define $W_n = \lvert w_n\rvert$ and real $\theta_n$ that
        $\inner{w_{n}}{w_{n+1}} = W_n W_{n+1} \cos{\theta_n}.$

      The sequence $\Xi_n = -\sum_0^n \inner{w_m}{w^*_m}$ has a non-zero limit, so let us
        find the largest $N > 0$ such that $\Xi_N = 0$.
      Then we can modify the original sequence by setting $w_n$, $w^*_n$ to zero for any $0 \leq n \leq N$ so that~\eqref{eq:vector2}
        still holds.
      Therefore, without loss of generality we can assume that $\Xi_n \neq 0$ for any $n \geq 0$.
      %\begin{prop}
        %For a given system $k$ point density is equivalent to the existence such $0 < a'_n \leq a_n$ that
        %\[
          %\nu_n = \frac{a'_{n-1} a'_{n-3} \dots}{a'_{n} a'_{n-2} \dots }
        %\]
        %belongs to $\ell^2$.
      %\end{prop}
      Setting $a'_n = a_n \cos{\theta_n}$ we see that the sequence
      \[
        W_n = \frac{\Xi_{n-1}/a'_n}{\Xi_{n-2}/a'_{n-1}} \cdot \frac{\Xi_{n-3}/a'_{n-2}}{\Xi_{n-4}/a'_{n-3}} \cdots
      \]
        belongs to $\ell^2$.
      Now since $\Xi_n = -\sum_0^n \inner{w_s}{w^*_s}$, we discover that
      \[
        \frac{\Xi_n}{\Xi_{n-1}} = 1 + \eta_n,
      \]
        where $\eta_n \in \ell^1$.
      Thus the product of such $(1 + \eta_s)$ fractions is bounded by some constant above.
      It follows that
      \[
        W^\#_n = \frac{1/a'_n}{1/a'_{n-1}} \cdot \frac{1/a'_{n-2}}{1/a'_{n-3}} \cdots
      \]
        belongs to $\ell^2$.
      Now we set $r_n$ to $\frac{W^\#_n}{W_n}w_n$, and then~\eqref{eq:vector2} holds since
      \[
        a_{n+1} \inner{r_n}{r_{n+1}} = a_{n+1} \frac{1/a'_{n+1}}{\Xi_n/a'_{n+1}} \inner{w_n}{w_{n+1}} = 1.
      \]
      Since $\lvert r_n \rvert = \lvert W^\#_n \rvert$ and $\{\lvert W^\#_n \rvert\}$ belongs to $\ell^2$,
        then $r_n \in \ell^2(\mathbb{R}^k)$ as well.
    \end{proof}

    Now we are ready to finish the proof.
    First we prove the theorem for the case $k=1$.
    \begin{prop}
      The system $\fsys$ is one point dense if and only if $\mu_n$ does not belong to $\ell^2$.
    \end{prop}
    \begin{proof}
      We apply Propositions~\ref{prop:kreformulation} and~\ref{prop:reformulation-lw}.
      The case $k=1$ has all the vectors $w_k$, $w^*_k$ lying on the same line.
      The formula for the multidimensional case reduces to the expression of $\mu_n$ whenever we set all the angles to zero $\theta_n = 0$,
        namely the lengths of the vectors $w_k$ are precisely $\mu_n$.
    \end{proof}

    And now we consider the higher dimensions.
    We finish the proof of the theorem by considering the case $k > 1$.
    \begin{prop}
      The system $\fsys$ is $k$ point dense ($k > 1$) if and only if the sequence $\{1/a_n\}_{n=1}^\infty$ does not belong to $\ell^1$.
    \end{prop}
    \begin{proof}
      According to Proposition~\ref{prop:reformulation-lw}, the $k$ point density is equivalent to the non-existence of
        a sequence of vectors $w_n$ in the $\ell^2(\mathbb{R}^k)$ which satisfy $a_n \inner{w_n}{w_{n-1}} = 1$.
      Obviously if there are such vectors $w_n$ then $1/a_n$ belongs to $\ell^1$.

      Conversely, suppose $1/a_n$ belongs to $\ell^1$.
      Then consider $W_n$ defined as follows:
      \[
        W_n = \max(a_n^{-\frac{1}{2}}, a^{-\frac{1}{2}}_{n+1}).
      \]
      It is easy to see that $W_n$ belongs to $\ell^2$.

      Notice that the product $W_nW_{n-1} \geq 1/a_n$, and so it is always possible to choose the angle $\theta_n$ so that
      \[
        a_n \langle w_n, w_{n-1} \rangle = a_n W_n W_{n-1}\cos{\theta_n} = 1.
      \]
      Now we have defined the lengths for $w_n$ and the angles between two consecutive vectors $w_{n-1}$, $w_n$.
      Obviously, for any $k \geq 2$ we are able to choose the corresponding vectors $w_n$.
    \end{proof}
  \end{proof}
  %\begin{remark}
    %The difference between two cases $k=1$ and $k\geq 2$ is in $\mathbb{R}^1$ any two vectors
      %have $0$ or $\pi$ angle between them.
  %\end{remark}
  % \begin{remark}
  %     Right here we can deduce the necessary and sufficient condition for the $k$-completeness property (for $k>1$) having the condition
  %     for the strong approximation property discovered earlier.
  %     As we remember $f_n$ approximates strongly iff the $a_n^{-1}$ is summable. Also we know as a fact that
  %     a system approximates strongly iff for any $k$ it is $k$-complete. Taking the last proposition into account we
  %     understand that the $k$-completeness condition is the same for all $k > 1$ and is identical to the strong approximation condition: $a_n^{-1}$ must be summable.
  %o  \end{remark}

\section{Pentadiagonal example}
  \label{section:pentadiagonal}
  In this section we explore the vector system $\fsys$ and its biorthogonal system $\fstarsys$ defined as follows:
  \begin{equation}
    \label{main-system}
    \begin{aligned}
      &\mathbf{f_{4j}} = e_{4j} \quad
      \mathbf{f^*_{4j}} = e_{4j} + d_{2j - 1} e_{4j-2} - b_{2j-1} e_{4j-1} + a_{2j} e_{4j+1} + c_{2j} e_{4j+2}\\
      &\mathbf{f_{4j+1}} = -a_{2j} e_{4j} + e_{4j+1} \quad
      \mathbf{f^*_{4j+1}} = e_{4j+1} + b_{2j} e_{4j+2},\\
      &\mathbf{f_{4j+2}} = e_{4j+2} + d_{2j} e_{4j} - b_{2j} e_{4j+1} + a_{2j+1} e_{4j+3} + c_{2j+1} e_{4j+4}\quad
      \mathbf{f^*_{4j+2}} = e_{4j+2},\\
      &\mathbf{f_{4j+3}} = e_{4j+3} + b_{2j+1} e_{4j+4}\quad
      \mathbf{f^*_{4j+3}} = -a_{2j+1} e_{4j+2} + e_{4j+3},
    \end{aligned}
  \end{equation}
    where the real coefficients $a_n$, $b_n$, $c_n$, $d_n$ are equal to zero whenever $n < 0$, and satisfy the equality
      $c_n + d_n = a_n b_n$ for any $n \geq 0$.
  \begin{prop}
    The given system is an $M$-basis.
  \end{prop}
  \begin{proof}
    The equality $c_n + d_n = a_n b_n$ guarantees the bi\-orthogonality,
      while the completeness of $\fsys$ and $\fstarsys$ is easy to check.
  \end{proof}

  \section{Main result}
    \begin{theorem}
      The given system is NOT $k$ point dense for some (equivalently any) $k > 1$ if and only if the sequence
      \[
        \mu_n = \min\left(\frac{1}{|a_n|} + \frac{1}{|b_n|}, \frac{1 + |b_n|}{|d_n|}, \frac{1 + |a_n|}{|c_n|}\right)
      \]
        belongs to $\ell^1$.
    \end{theorem}
    \begin{proof}
      %Let us consider a $k$-dimensional operator $T$ such that 
        %$Tr(TR) = 0$ for each $R \in \cal{R}_1(\cal{A})$ which essentially means that
        %$\langle Tf_n, f_n^* \rangle = 0$ for any $n > 0$. 
      Notice that the partial sums of the Fourier series for the given system are somehow close to the
        partial sums of the canonical Fourier series (using the orthonormal basis $e_n$).
      Consider the following differences
      \[
        \Xi_n = \sum_0^n \langle Tf_s, f_s^* \rangle - \sum_0^n \langle Te_s, e_s \rangle,
      \]
        where $\langle \cdot, \cdot\rangle$ denotes the inner product in $\cal{H}$.
      For any $j \geq 0$ we have
      \begin{align*}
        \Xi_{4j} = a_{2j} T_{4j+1, 4j} + c_{2j} T_{4j+2, 4j},\\
        \Xi_{4j + 1} = -d_{2j} T_{4j+2, 4j} + b_{2j} T_{4j+2, 4j+1},\\
        \Xi_{4j + 2} = a_{2j+1} T_{4j+2, 4j+3} + c_{2j+1} T_{4j+2, 4j+4},\\
        \Xi_{4j + 3} = -d_{2j+1} T_{4j+2, 4j+4} + b_{2j+1} T_{4j+3, 4j+4},
      \end{align*}
      where $T_{ij}$ stands for $\langle Te_j, e_i \rangle$.
      \begin{prop}
        \label{inf-dim-statement}
        There exists an operator $T$ with the trace equal to $-1$ such that $\langle Tf_n, f_n^*\rangle = 0$ for any $n \geq 0$
          if and only if the sequence $\left\{\mu_n\right\}$ belongs to $\ell^1$.
      \end{prop}
      \begin{proof}
        Assume that $\mu_n \in \ell^1$.
        We will then construct a required operator $T$.
        Let $T_{00}$ be equal to $-1$, and $T_{jj}$ be equal to zero for any $j > 0$.

        Consider three cases for each $n \geq 0$.

        \noindent\textbf{Case 1.}
        Suppose $\mu_n = 1/|a_n| + 1/|b_n|$.
        For $n=2j$ we set:
        \begin{align*}
          T_{4j+1,4j}&=1/a_n & \quad T_{4j+2,4j} = 0,\\
          T_{4j+2,4j+1}&=1/b_n.
        \end{align*}
        That guarantees an equality $\Xi_{2n} = \Xi_{2n+1} = 1$.
        For $n=2j+1$ we set:
        \begin{align*}
          T_{4j+2,4j+3}&=1/a_n & \quad T_{4j+2,4j+4} = 0,\\
          T_{4j+3,4j+4}&=1/b_n,
        \end{align*}
        which provides an equality $\Xi_{2n} = \Xi_{2n+1} = 1$.

        \noindent\textbf{Case 2.}
        Assume $\mu_n = (1 + |b_n|)/|d_n|$. 
        For $n=2j$ we set
        \begin{align*}
          T_{4j+1,4j} &= b_{2j}/d_{2j} & \quad T_{4j+2,4j} = -1/d_{2j},\\
          T_{4j+2,4j+1} &= 0.
        \end{align*}
        Again, we have $\Xi_{2n} = \Xi_{2n+1} = 1$.
        The case $n = 2j + 1$ is left to the reader.

        \noindent\textbf{Case 3.}
        Suppose $\mu_n = (1 + |a_n|)/|c_n|$. 
        For $n = 2j + 1$ we assign:
        \begin{align*}
          T_{4j+2,4j+3} &= 0 & \quad T_{4j+2,4j+4} = 1/c_{2j+1},\\
          T_{4j+3,4j+4} &= a_{2j+1}/c_{2j+1}.
        \end{align*}
        The case $n = 2j$ is analogous.
        \medskip
        All the other entries $T_{ij}$ we set to zero.
        These assignments ensure that $\Xi_n = 1$ for any $n \geq 0$.

        The constructed operator $T$ belongs to the trace class since all the non-zero elements are summable 
          due to the assumption that $\left\{\mu_n\right\} \in \ell^1$.
        Notice that $T$ annihilates all the rank one operators $f^*_n \otimes f_n$ due to the equality $\Xi_n = 1$.
        Since the trace of the operator $T$ is equal to $-1$, the sufficiency is proved.

        \medskip
        Now assume that there exists a trace class operator $T$ with the trace equal to $-1$, which annihilates all the rank one operatos $f^*_n \otimes f_n$.
        Then all the operator matrix elements $T_{nn}, T_{n, n+1}, T_{n, n+2}$ belong to $\ell^1$ (simple property of a trace class operator).
        As we know, $\Xi_n$ tends to $1$, since the trace of the operator is $-1$.
        Look at the sum $S_{2j} = |T_{4j+1, 4j}| + |T_{4j+2,4j}| + |T_{4j+2,4j+1}|$.
        Observe that $S_{2j}$ is a linear function if considered as a function of $T_{4j+2, 4j}$.
        Its minimum is attained on the boundary of the domain, thus it is greater than $\mu_{2j}/2$ whenever $j$ is sufficiently large.
        The second pair of equalities gives out the minimum value greater than $\mu_{2j+1}/2$ when $j$ is large,
          which shows that the stated condition is necessary for the existence of the operator $T$.
        %\alex{TODO elaborate}
      \end{proof}
      Now consider the case of the $k$-dimensional operator $T$.
      Let us view the operator $T$ as a sum of $k$ rank one operators:
      \[
        T = \sum_1^k y^s \otimes x^s,
      \]
        where $x^s, y^s \in \cal{H}$.
      Let us define vectors $v_n$ and $u_n$ for $n \geq 0 $ which lie in $\mathbb{R}^k$ as follows:
      \begin{align*}
        v_{2j} &= (y^1_{4j}, y^2_{4j}, \dots ,y^k_{4j}) \quad
        &v^*_{2j} = (x^1_{4j}, x^2_{4j}, \dots ,x^k_{4j}) \\
        v_{2j+1} &= (x^1_{4j+2}, x^2_{4j+2}, \dots ,x^k_{4j+2}) \quad
        &v^*_{2j+1} = (y^1_{4j+2}, y^2_{4j+2}, \dots ,y^k_{4j+2}) \\
        u_{2j} &= (x^1_{4j+1}, x^2_{4j+1}, \dots ,x^k_{4j+1}) \quad
        &u^*_{2j} = (y^1_{4j+1}, y^2_{4j+1}, \dots ,y^k_{4j+1}) \\
        u_{2j+1} &= (y^1_{4j+3}, y^2_{4j+3}, \dots ,y^k_{4j+3}) \quad
        &u^*_{2j+1} = (x^1_{4j+3}, x^2_{4j+3}, \dots ,x^k_{4j+3}) 
      \end{align*}
      Note that the sequences $|v_n|$, $|u_n|$ belong to $\ell^2$.
      We can rewrite the previous equations using the introduced vectors.
      \begin{align*}
        \Xi_{4j} &= a_{2j} \langle u_{2j}, v_{2j}\rangle + c_{2j} \langle v_{2j+1}, v_{2j}\rangle,\\
        \Xi_{4j + 1} &= -d_{2j} \langle v_{2j+1}, v_{2j}\rangle + b_{2j} \langle v_{2j+1}, u_{2j}\rangle,\\
        \Xi_{4j + 2} &= a_{2j+1} \langle v_{2j+1}, u_{2j+1} \rangle + c_{2j+1} \langle v_{2j+1}, v_{2j+2} \rangle,\\
        \Xi_{4j + 3} &= -d_{2j+1} \langle v_{2j+1}, v_{2j+2}\rangle + b_{2j+1} \langle u_{2j+1}, v_{2j+2} \rangle.
      \end{align*}
      Here $\langle\cdot, \cdot\rangle$ denotes the scalar product in $\mathbb{R}^k$.
      %\begin{note}
        %Observe that in such setup we always have $\Xi_n = 0$ for any $n \leq 1$.
      %\end{note}
      Now we will study the simplified system instead.
      \begin{align*}
        \Xi_{2j} &= a_{j} \langle u_{j}, v_{j} \rangle  + c_{j} \langle v_{j+1}, v_{j} \rangle,\\
        \Xi_{2j + 1} &= -d_{j} \langle v_{j+1}, v_{j} \rangle + b_{j} \langle v_{j+1}, u_{j}\rangle.
      \end{align*}
      To summarize, we obtained that the existence of an operator $T$ could be reduced to the existence of vectors $u_n$, $v_n$ in the $\mathbb{R}^k$
        which satisfy the conditions given above.
      Given that the sequences of vectors $u_n$, $v_n$ lie in the $\mathbb{R}^k$, we might think of the scalar product as
        the product of the vector lengths and the cosinus of the angle between the vectors.
      \begin{prop}
        \label{k-dim-statement}
        If $\left\{\mu_n\right\}$ belongs to $\ell^1$ then it is possible to construct the vectors $u_n, v_n \in \mathbb{R}^2$ such that
          the equations above are true and for any $n \geq 0$ we have $\Xi_n = 1$.
      \end{prop}
      \begin{corol*}
        For any $k > 1$ there exists a $k$-dimensional operator $T$ with the trace equal to $-1$ such that $\langle Tf_n, f_n^*\rangle = 0$ for any $n \geq 0$
          if and only if the sequence $\left\{\mu_n\right\}$ belongs to $\ell^1$.
      \end{corol*}
      \begin{proof}
        %The operator matrix will look very similar to one we built in the previous proposition.
        Here again we are going to look at three possible values of the $\mu_n$.
        For each $n \geq 0$ we are going to find the vector lengths $V_n = |v_n|$, $U_n = |u_n|$ and three angles:
          $\alpha_n$ which stands for the angle between the vectors $v_n$ and $v_{n + 1}$,
          $\beta_n$ which denotes the angle between $v_n$ and $u_n$,
          and $\gamma_n$ standing for the angle between $v_{n + 1}$ and $u_n$.
        We will write out the vector lengths and define the angles and we will prove that the corresponding vectors could be settled in $\mathbb{R}^2$.
        %We choose the vector $v^*_0$ such that $\langle v_0, v^*_0 \rangle = -1$ and set all
          %the other vectors $u^*_n$, $v^*_n$ to $\vec{0}$.
        %That guarantees that the trace of the constructed operator (if it belongs to the trace class) is equal to $-1$.

        Within the construction we are going to set all $V_n$ step by step.
        In order to do that we are going to define an auxiliary sequence $\{M_n\}_{n=0}^\infty \in \ell^2$ such that $V_n \geq M_n$ for any $n \geq 0$.
        On each step $n$ we are going to define $V_n$ and $M_{n+1}$.
        We start by setting $M_0 = V_0 = 1$.

        \noindent\textbf{Case 1.} Suppose $\mu_n = 1/|a_n| + 1/|b_n|$.

          Here we want to have $v_n$ orthogonal to $v_{n+1}$.
          We set:
          \begin{align*}
              V_n &= \max\left(M_n, \frac{1}{\sqrt{\smash[b]{|a_n|}}}\right),\\
              U_n &= \sqrt{\frac{1}{a_n^2 V_n^2} + \frac{1}{b_n^2 V_{n+1}^2}},\\
              M_{n+1} &= \frac{1}{\sqrt{\smash[b]{|b_n|}}}.
          \end{align*}
          \begin{prop}
            The following inequalities are true.
            \begin{align*}
              M_{n+1} &\leq \sqrt{\mu_n},\\
              U_n &\leq \sqrt{\mu_n},\\
              V_n &\leq \max(\sqrt{\mu_n}, M_n).
            \end{align*}
            There exist such angles $\beta_n$, $\gamma_n$ that with the values $U_n$, $V_n$ defined like this the following is true:
            \begin{align*}
              \langle u_n, v_n \rangle &= 1/a_n,\\
              \langle u_n, v_{n+1} \rangle &= 1/b_n,\\
              \langle v_n, v_{n+1} \rangle &= 0.
            \end{align*}
          \end{prop}
          \begin{proof}
            First part of the proposition is trivial.
            Now we want to understand why there exist such $\beta_n$ and $\gamma_n$ that all the scalar products of
            $v_n$, $v_{n+1}$, $u_n$ satisfy the conditions above.
            Now observe that the chosen $U_n$, $1/(|a_n| V_n)$ and $1/(|b_n| V_{n+1})$ 
              comprise a right triangle with $U_n$ as a hypotenuse.
            As a result there always be such $\beta_n$ and $\gamma_n$ that
            \begin{align*}
              |U_n \cos{\beta_n}| &= \frac{1}{|a_n|V_n},\\
              \left|U_n \cos{\gamma_n}\right| &= \left|U_n \sin{\beta_n}\right| = \frac{1}{|b_n|V_{n+1}}.
            \end{align*}
          \end{proof}
        \noindent\textbf{Case 2.} Assume $\mu_n = (1 + |a_n|)/|c_n|$.

          Here we will have $u_n$ orthogonal to $v_n$.
          Assign:
          \begin{align*}
            M_{n+1} &= \max\biggl(\smash[b]{\frac{\sqrt{|a_n|}}{\sqrt{|c_n|}}}, \frac{1}{\sqrt{\smash[b]{|c_n|}}}\biggr),\\
            V_n &= \max\left(M_n, \frac{2}{\sqrt{\smash[b]{|c_n|}}}\right),\\
            \alpha_n &= \arccos{\frac{1}{c_n V_n V_{n+1}}},\\
            \gamma_n &= \frac{\pi}{2} \pm \alpha_n,\\
            U_n &= \frac{a_n}{c_n \cos{\gamma_n} V_{n+1}}.
          \end{align*}
          \begin{remark*}
            We choose plus or minus in the expression of $\gamma_n$ in order to make the $a_n/(c_n \cos{\gamma_n})$ always positive.
          \end{remark*}
          \begin{prop}
            The angle $\alpha_n$ is defined correctly and the following inequalities are true:
            \begin{align*}
                M_{n+1} &\leq \sqrt{\mu_n},\\
                V_n &\leq \max(2\sqrt{\mu_n}, M_n),\\
                U_n &\leq \sqrt{\mu_n}.
            \end{align*}
            With the values $U_n$, $V_n$, $\gamma_n$, $\alpha_n$ defined like we have
            \begin{align*}
              \langle u_n, v_n \rangle &= 0,\\
              \langle u_n, v_{n+1} \rangle &= a_n/c_n,\\
              \langle v_n, v_{n+1} \rangle &= 1/c_n.
            \end{align*}
          \end{prop}
          \begin{proof}
            Firstly, $\alpha_n$ is defined correctly since
            \[
              V_n V_{n+1} \geq V_n M_{n+1} \geq \frac{2}{\sqrt{\smash[b]{|c_n|}}} \frac{1}{\sqrt{\smash[b]{|c_n|}}} = \frac{2}{|c_n|}
            \]
            which means that the absolute value of the arccosinus argument is always less than $1/2$.
            Now taking into account that $|\cos{\alpha_n}| \leq 1/2$, notice that $|\cos{\gamma_n}| = |\sin{\alpha_n}|$ is always greater than $1/2$.
            It is the fact which we need later.
            
            Trivially, $M_{n+1} \leq \sqrt{\mu_n}$ and $V_n \leq \max(2\sqrt{\mu_n}, M_n)$.
            Next we set
            \[
              U_n = |U_n| \leq 2 \frac{|a_n|}{|c_n|} \frac{1}{V_{n+1}} \leq 2 \frac{\sqrt{|a_n|}}{\sqrt{|c_n|}} \leq \sqrt{\mu_n}.
            \]
            It is easy to notice that the angles and $U_n$ are chosen in such way that the scalar product conditions are 
              satisfied.
            Finally, we are able to lay out the vectors $v_n$, $u_n$ and $v_{n+1}$ with the prescribed angles 
              since we have $\beta_n = \pi/2 = (\pi/2 \pm \alpha_n) \mp \alpha_n = \gamma_n \mp \alpha_n$,
              where we choose the '$\mp$' sign accordingly to our choice in the expression of the $\gamma_n$ angle.
          \end{proof}
        \noindent\textbf{Case 3.} Suppose $\mu_n = (1 + |b_n|)/|d_n|$.

          This case is almost identical to the previous one.
          Here we will have $u_n$ orthogonal to $v_{n+1}$.
          Assign:
          \begin{align*}
            M_{n+1} &= \max\biggl(\smash[t]{\frac{\sqrt{|b_n|}}{\sqrt{|d_n|}}}, \frac{1}{\sqrt{\smash[b]{|d_n|}}}\biggr),\\
            V_n &= \max\left(M_n, \frac{2}{\sqrt{\smash[b]{|d_n|}}}\right),\\
            \alpha_n &= \arccos{\frac{1}{-d_n V_n V_{n+1}}},\\
            \beta_n &= \frac{\pi}{2} \pm \alpha_n,\\
            U_n &= \frac{b_n}{d_n \cos{\beta_n} V_n}.
          \end{align*}
          \begin{remark*}
            We choose plus or minus in the expression of $\beta_n$ in order to make the $b_n/(d_n \cos{\beta_n})$ always positive.
          \end{remark*}
          \begin{prop}
              The angle $\alpha_n$ is defined correctly and the following inequalities are true:
              \begin{align*}
                M_{n+1} &\leq \sqrt{\mu_n},\\
                V_n &\leq \max(2\sqrt{\mu_n}, M_n),\\
                U_n &\leq \sqrt{\mu_n}.
              \end{align*}
              With the values $U_n$, $V_n$, $\beta_n$, $\alpha_n$ defined like this the following is true:
              \begin{align*}
                \langle u_n, v_n \rangle &= b_n/d_n,\\
                \langle u_n, v_{n+1} \rangle &= 0,\\
                \langle v_n, v_{n+1} \rangle &= -1/d_n.
              \end{align*}
          \end{prop}
          \begin{proof}
            Firstly, $\alpha_n$ is defined correctly since
            \[
              V_n V_{n+1} \geq V_n M_{n+1} \geq \frac{2}{\sqrt{\smash[b]{|d_n|}}} \frac{1}{\sqrt{\smash[b]{|d_n|}}} = \frac{2}{|d_n|},
            \]
              which means that the absolute value of the arccosinus argument is always less than $1/2$. Now taking into
              account that $|\cos{\alpha_n}|$ is less than or equal to $1/2$, we always have $|\cos{\beta_n}| > 1/2$.
            
            Trivially, we have $M_{n+1} \leq \sqrt{\mu_n}$ and $V_n \leq \max(2\sqrt{\mu_n}, M_n)$.
            Next
            \[
              U_n = |U_n| \leq 2 \frac{|b_n|}{|d_n|} \frac{1}{V_n} \leq 2 \frac{\sqrt{|b_n|}}{\sqrt{|d_n|}} \leq \sqrt{\mu_n}.
            \]
            It is easy to notice that the angles and $U_n$ are chosen in such way that the scalar product conditions are satisfied.
            Finally, we are able to lay out the vectors $v_n$, $u_n$ and $v_{n+1}$ with the prescribed angles 
              since we have $\gamma_n = \pi/2 = (\pi/2 \pm \alpha_n) \mp \alpha_n = \beta_n \mp \alpha_n$,
              where we choose the '$\mp$' sign accordingly to our choice in the expression of the $\beta_n$ angle.
          \end{proof}
        In each of three cases we guaranteed that $M_{n+1} \leq \sqrt{\mu_n}$ and
          that $V_n \leq \max(M_n, 2\sqrt{\mu_n})$.
        Hence, we get that $V_n$ is bounded up to some constant by $\max(\sqrt{\mu_{n-1}}, \sqrt{\mu_n})$.
        Due to the three propositions above, for any $n \geq 0$ we have $U_n \leq \sqrt{\mu_n}$.
        Thus the constructed sequences $V_n$ and $U_n$ belong to $\ell^2$.

        In the end we choose the vector $v_0^*$ such that $\langle v_0, v_0^*\rangle$ is equal to $-1$.
        It is possible since the vector $v_0$ is not trivial.
        Furthermore, we set all the other $v^*_n$ and $u^*_n$ to the zero vector in order to guarantee that for any $n \geq 0$ we have $\Xi_n = 1$.
        The trace of the constructed operator is obviously equal to $-1$.
      \end{proof}
      The proof of the theorem is finished with the proof of the statements~\ref{inf-dim-statement} and~\ref{k-dim-statement}.
    \end{proof}


\section{Acknowledgements}
  %This is part of author's Ph.D. thesis, written under the supervision of Anton Baranov at the PDMI RAS.
  The author gratefully acknowledges the many helpful suggestions of Anton Baranov during the preparation of the paper.

\begin {thebibliography}{20}
  \bibitem{argyroslambrou}
    S.~\!Argyros, M.~\!Lambrou and W.E~\!Longstaff,
    \emph{Atomic Boolean Subspace Lattices and Applications to the Theory of Bases},
    Memoirs. Amer. Math. Soc., No. 445 (1991).

  \bibitem{azoff}
    E.~\!Azoff, H.~\!Shehada,
    \emph{Algebras generated by mutually orthogonal idempotent operators},
    J. Oper. Theory, 29 (1993), 2, 249--267.

  \bibitem{larson}
    D.~\!Larson, W.~\!Wogen,
    \emph{Reflexivity properties of $T\bigoplus0$},
    J. Funct. Anal., 92 (1990), 448--467.

  \bibitem{review}
    J.A~\!Erdos,
    \emph{Basis theory and operator algebras},
    In: A. Katavolos (ed.), Operator Algebras and Application, Kluwer Academic Publishers, 1997, pp. 209--223.

  \bibitem{katavolos}
    A.~\!Katavolos, M.~\!Lambrou and M.~\!Papadakis,
    \emph{On some algebras diagonalized by $M$-bases of $\ell^2$},
    Integr. Equat. Oper. Theory, 17 (1993), 1, 68--94.

  \bibitem{erdos}
    J.A~\!Erdos,
    \emph{Operators of finite rank in nest algebras},
    J. London Math. Soc., 43 (1968), 391--397.

  \bibitem{laurielongstaff}
    C.~\!Laurie, W.~\!Longstaff,
    \emph{A note on rank one operators in reflexive algebras},
    Proc. Amer. Math. Soc., 89 (1983), 293 - 297.

  \bibitem{longstaff}
    W.E~\!Longstaff,
    \emph{Operators of rank one in reflexive algebras},
    Canadian J. Math., 27 (1976), 19--23.

  \bibitem{raney}
    G.N.~\!Raney,
    \emph{Completely distributive complete lattices},
    Proc. Amer. Math. Soc. 3 (1952), 677-680.

  \bibitem{bbb}
    A.~\!Baranov, Y.~\!Belov and A.~\!Borichev,
    \emph{Hereditary completeness for systems of exponentials and reproducing kernels},
    Adv. Math., 235 (2013), 1, 525--554.

  \bibitem{bbb1}
    A.~\!Baranov, Y.~\!Belov and A.~\!Borichev,
    \emph{Spectral synthesis in de Branges spaces},
    Geom. Funct. Anal. (GAFA), 25 (2015), 2, 417--452.

  \bibitem{ad_preprint}
    A.D.~\!Baranov, D.V.~\!Yakubovich,
    \emph{Completeness and spectral synthesis of nonselfadjoint one-dimensional
    perturbations of selfadjoint operators},
    Advances in Mathematics, 302 (2016), 740-798;


% \bibitem{nikolski}
    %N.K.~\!Nikol'skii,
    %\emph{Complete extensions of Volterra operators},
    %Izv. Akad. Nauk SSSR Ser. Mat 33(1969), 1349--1355. (Russian)
\end{thebibliography}

\end{document}
